% ==================================================================================================================================
% Introduction

\minitoc  % Affiche la table des matières pour ce chapitre








\begin{definition}[Vecteur Aléatoire]
    Une vecteur aléatoire à $n$ composantes est une al=pplication $V$ d'un espace probabilisé $(\Omega, \mathcal{A}, P)$ dans $\R^n$ telle que l'image réciproque $V^{-1}(B)$ de tout borélien de $\R^n$ soit un élément de la tribu de $\Omega$.
\end{definition}

\begin{theorem}[Fubini]
    Si $ f : \R^2 \rightarrow \R $ est intégrable sur $[a, b] \times [c, d]$ alors pour presque tout $x \in [a, b]$, la fonction partielle $ y \mapsto f(x, y)$ est intégrable sur $[c, d]$ et 
        \[ \iint_{[a, b] \times [c, d]} f(x, y) dx \; dy = \int_{a}^{b} \left( \int_{c}^{d} f(x, y) dy \right) dx \]
\end{theorem}

\begin{definition}[Densité Conjointe]
    Soit $ f : \R^2 \rightarrow \R $ une application intégrable telle que $f \geq 0$ et 
        \[ \iint_{\R^2} f(x, y) dx \; dy = 1 \]
    Alors la loi $\mathcal{L}(X, Y)$ du couple $(X, Y)$ est absolument continue de densité conjointe $f$ si, pour tout borélien $B$, on a :
        \[ P((X, Y) \in B) = \iint_B f(x, y) dx \; dy \]
    Si deux densité sont égales presque partout, alors elles définissent la même loi de probabilité. Donc si on modifie une densité sur un ensemble négligeable, elle définit toujours la même loi.
\end{definition}













\section*{Démonstrations}

- intersection quelconque de tribu est une tribu
- limite et continuité de la fonction de répartition


