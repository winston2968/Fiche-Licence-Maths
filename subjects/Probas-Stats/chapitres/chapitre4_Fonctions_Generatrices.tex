% ==================================================================================================================================
% Introduction

\minitoc  % Affiche la table des matières pour ce chapitre

\section*{Définition Générale}

On considère une variable aléatoire discrète entière positive non nulle $X$ qui prend ses valeurs dans $\N$ et pour tout $n$ dans $\N$ on pose 
$p_n = P(X = n)$, on a donc, d'après la définition d'une probabilité $ \sum_{n=0}^{\infty} p_n = 1 $

\begin{definition}[Fonction Génératrice]
    On appelle fonction génératrice de $X$ la fonction :
        \[ \boxed{ g_X(z) = \sum_{n=0}^{\infty} P(X = n) z^n} \] 
    C'est une série entière de la variable $z \in \C$.    
\end{definition}

\begin{remark}
    Le rayon de convergence cette série est \underline{supérieur} à $1$.
\end{remark}

\section*{Propriétés}

\begin{itemize}
    \item $g_X$ est continue sur $ \overline{D(0, 1)}$ et $\mathcal{C}^\infty$ sur $D(0, 1)$.
    \item pour $|z| > 1$ on a $g_X(z) = E(z^X)$
    \item la fonction génératrice caractérise la loi 
        \[ \text{i.e. } g_X = g_Y \text{ sur un voisinnage de 0} \Longrightarrow \mathcal{L}(X = \mathcal{L}(Y) \] 
    \item si $E(X)$ existe alors $E(X) = g_X(1)$ 
    \item si $E(X^k), k \in \N$ existe alors $g_X^{(k)}(z) = E(X(X-1)(X-2) \dots (X - k + 1))$ 
        
        En particulier, si $E(X^2)$ existe, alors 
            \[ g_X^{\prime \prime}(1) = E(X(X-1)) = E(X^2) - E(X) \] 
        Donc $ V(X) = g_X^{\prime \prime}(1) + g_X^{\prime}(1) - (g_X^{\prime}(1))^2 $
\end{itemize}


\section*{Somme de variables aléatoires entières positives indépendantes}

\begin{theorem}
    Si $X$ et $Y$ sont indépendantes, alors 
        \[ g_{X + Y} = g_X g_Y \] 
\end{theorem}

\begin{proof}
    On a :
        \[ g_X(z) = \sum_{n\in \N} p_n z^n, \quad g_Y(z) = \sum_{n \in \N} q_n z^n, \quad g_{X+Y}(z) = \sum_{n \in \N} r_n z^n \] 
    D'après la formule des probabilités totales :
        \[ r_n = P(X + Y = n) = \sum_{k = 0}^{n} P(X = k) P(Y = n-k) = \sum_{k=0}^{n} p_k q_{n_k} \]
    et 
    \begin{align*}
        g_{X +Y}(z) &= \sum_{n = 0}^{\infty} \sum_{k = 0}^{n} p_k q_{n-k} z^n = \sum_{n = 0}^{\infty} \sum_{k = 0}^{n} p_k q_{n-k} z^k z^{n-k} \\
        &= \Biggl( \sum_{i = 0}^{\infty} p_i z^i \Biggr) \Biggl( \sum_{j = 0}^{\infty} q_j z^j \Biggr) \\
        &= g_X g_Y
    \end{align*}
   
\end{proof}
