% ==================================================================================================================================
% Introduction

\minitoc  % Affiche la table des matières pour ce chapitre

Dans le chapitre précédent, nous avons étudié l'estimation ponctuelle, qui consiste à proposer une 
valeur unique pour un paramètre inconnu d'une population à partir d’un échantillon. 
Cependant, une estimation ponctuelle est sujette à une certaine incertitude, car elle dépend du 
choix de l’échantillon observé.

Pour quantifier cette incertitude, on introduit la notion d’intervalle de confiance. 
Contrairement aux estimateurs ponctuels, un intervalle de confiance ne donne pas une valeur unique 
du paramètre, mais une plage de valeurs dans laquelle le paramètre a une forte probabilité de se situer.

L’objectif de ce chapitre est d’étudier les principes fondamentaux des intervalles de confiance, 
leur construction et leur interprétation dans le cadre des statistiques inférentielles.


% ==================================================================================================================================
% Premières Définitions

\section{Premières Définitions}


Commençons par bien définir la notion d'intervalle de confiance... 

\begin{definition}[Intervalle de Confiance]
    Soit $\phi$ un paramètre inconnu d'une population (une moyenne, une proportion, etc..). 
    Un intervalle de confiance au niveau $1 - \alpha$ est un intervalle $[ L(X); U(X)]$ définit 
    à partir d'un échantillon $X = (X_1, \dots, X_n)$, tel que :
        \[ \myP(L(X) \leqslant \theta \leqslant U(X)) = 1 - \alpha \] 
    où $\alpha$ est le \textbf{risque} ou \textbf{niveau de signification} et $1-\alpha$ 
    est le \textbf{niveau de confiance} de l'intervalle. En général, on choisit $0.95$ ou $0.99$. 
\end{definition}

Cette notion de risque signifie que si l'on répétait l'échantillonnage un grand nombre de fois, 
alors dans $100(1 - \alpha) \%$ des cas, l'intervalle contiendrai la vraie valeur de $\theta$. 

\vspace{0.3cm}

En revanche, pour un intervalle donné, on ne peut pas affirmer que $\theta$ appartient à cet 
intervalle avec une probabilité $1 - \alpha$, car $\theta$ est une quantité fixe, non aléatoire. 
Ce point est fondamental dans l’interprétation des intervalles de confiance.


% ==================================================================================================================================
% Intervalles de confiance d'une moyenne 

\section{Intervalle de confiance d'une moyenne}

\subsection{Principe}

Soit $\mu$ la moyenne d'une population et une échantillon de cette population $X = (X_1, \dots, x_n)$ de 
taille $n \in \N$. 
Ici, l'objectif est de construire un intervalle $[L,U]$ tel que: 
    \[ \myP(L \leqslant \mu \leqslant U) = 1 - \alpha \] 

Pour construire cet intervalle, nous allons utiliser un estimateur ponctuel 
de la moyenne théorique, la moyenne empirique :
    \[ \overline{X} = \frac{1}{n} \left( \sum_{i=1}^{n} X_i \right) \] 

La dispersion des valeurs autour de $\mu$ est caractérisée par l'écart-type $\sigma$ et la taille de l'échantillon $n$. 
Grâce au théorème central limite, la distribution de $\overline{X}$ suit approximativement une loi normale si $n$ est 
suffisament grand. 

\subsection{Premier Cas : si l'écart-type est connu}

Si l'écart-type de la population $\sigma$ est connu, alors $\overline{X}$ suit une loi normale de moyenne $\mu$ et 
d'écart-type $\frac{\sigma}{\sqrt{n}}$. On a donc : 
    \[ \overline{X} \sim \mathcal{N}\left(\mu, \frac{\sigma}{\sqrt{n}}\right) \] 
On peut donc construire l'intervalle de confiance de $\mu$ suivant : 
    \[ \boxed{IC_\mu := \left[ \overline{X} \pm z_{\alpha/2} \frac{\sigma}{\sqrt{n}}\right]} \] 
où $ z_{\alpha/2}$ est le quantille de la loi normale tel que :
    \[ \myP(Z \leqslant z_{\alpha/2}) = 1 - \frac{\alpha}{2} \] 

\subsection{Second Cas : si l'écart-type est inconnu}

Si l'écart-type de la population $\sigma$ est connu, on doit l'estimer par l'écart-type 
empirique de l'échantillon :
    \[ S = \sqrt{\frac{1}{n-1} \sum_{i=1}^{n} (X_i - \overline{X})^2 } \] 

Dans ce cas, la statistique suivante suit une loi de Student à $n-1$ degrés de liberté : 
    \[ T = \frac{\overline{X}  \mu}{S/\sqrt{n}} \sim t_{n-1} \] 

On peut alors construire l'intervalle de confiance suivant pour $\mu$ : 
    \[ \boxed{ IC_\mu := \left[\overline{X} \pm t_{\frac{\alpha}{2},n-1} \frac{S}{\sqrt{n}}\right] } \] 
où $t_{\frac{\alpha}{2},n-1}$ est le quantille de la loi de Student à $n-1$ degrés de liberté. 

\subsection{Interprétation et tableau récapitulatif}

\begin{itemize}
    \item Si l'on répétait de nombreuses fois l'expérience d'échantillonnage, et que l'on construisait à chaque fois 
    l'intervalle alors dans $100 (1 - \alpha) \%$ des cas $\mu$ serait dans l'intervalle de confiance. 
    \item \textbf{Attention : } Cela ne veut pas dire que $\mu$ a une probabilité de $1 - \alpha$ d'être dans l'intervalle 
    puisque $\mu$ n'est pas une variable aléatoire, c'est une constante inconnue. 
    \item En étudiant la construction des intervalles, on peut remarquer que plus l'échantillon est grand, plus l'intervalle 
    est petit. En théorie, on peut donc encadrer $\mu$ de façon aussi fine que l'on souhaite. 
\end{itemize}


\begin{table}[h]
    \centering
    \renewcommand{\arraystretch}{2} % Augmente l'espacement vertical
    \setlength{\tabcolsep}{4pt} % Réduction de l'espacement des colonnes
    \begin{tabular}{|c|c|c|c|}
        \hline
        \textbf{Taille} & \textbf{$\sigma$ connu ?} & \textbf{Statistique} & \textbf{Distribution} \\
        \hline
        $n > 30$ & Oui & $\displaystyle Z = \frac{\bar{X} - \mu}{\sigma / \sqrt{n}}$ & Normale $\mathcal{N}(0,1)$ \\
        \hline
        $n > 30$ & Non & $\displaystyle T = \frac{\bar{X} - \mu}{S / \sqrt{n}}$ & Approximativement $\mathcal{N}(0,1)$ \\
        \hline
        $n < 30$ & Oui & $\displaystyle Z = \frac{\bar{X} - \mu}{\sigma / \sqrt{n}}$ & Normale $\mathcal{N}(0,1)$ \\
        \hline
        $n < 30$ & Non & $\displaystyle T = \frac{\bar{X} - \mu}{S / \sqrt{n}}$ & Student ($n-1$ ddl) \\
        \hline
    \end{tabular}
    \caption{Synthèse et conseils d'application pour l'IC de $\mu$}
    \label{tab:IC_synthese}
\end{table}

% ==================================================================================================================================
% Intervalle de confiance d'une proportion

\section{Intervalle de confiance d'une proportion}

On considère maintenant un échantillon $X = (X_1, \dots, X_n), n \in \N$ de fréquence $f$ d'un paramètre observé. 
On souhaite déterminer la proporition $p$ de ce même paramètre dans la population où l'échantillon a été prélevé. 
Pour cela, nous allons construire un intervalle de confiance pour obtenir un encadrement plus ou moins précis 
de la proportion (exacte) recherchée. 

\vspace{0.3cm}

On suppose ici que $n > 30$. Alors la distribution de la proportion observée $f$ peut être approchée par une loi normale 
d'après le théorème central limite. 

\vspace{0.3cm}

Ainsi l'intervalle de confiance pour $p$ à un niveau de confiance $1 - \alpha$ est donné par :
    \[ \boxed{IC_p := \left[ f \pm z_{\alpha/2} \sqrt{\frac{f(1-f)}{n}}\right]} \] 
où : 
\begin{itemize}
    \item $z_{\alpha/2}$ est le quantile de la loi normale standard. 
    \item $\sqrt{\frac{f(1-f)}{n}} $ représente l'écart-type de $f$. 
\end{itemize}

% ==================================================================================================================================
% Taille d'un échantillon 

\section{Taille d'un échantillon}

Depuis le début du chapitre, on construit des intervalles de confiance pour des moyennes et des proportions. 
Or on remarque bien que la précision de cet intervalle dépend énormément de la taille de l'échantillon prélevé 
dans la population. 
On pourrait alors se demander, si, en fonction des caractéristiques de la population et d'une précision voulue, 
on ne pourrait pas directement déterminer la taille minimale de l'échantillon à prélever. 

\subsection{Taille de l'échantillon pour estimer une moyenne}

Avant de réaliser l'échantillonnage, on va poser une précision $d \in \R$, telle que :
    \[ \mu = \overline{x} \pm d \] 

on a donc, d'après la formule de l'intervalle précédent : 
    \[ \boxed{ d = z_\alpha \frac{\sigma}{\sqrt{n}} \iff n = \left(z_\alpha \frac{\sigma}{d}\right)^2 } \] 


\subsection{Taille de l'échantillon pour estimer une proportion}

Avant de réaliser l'échantillonnage, on va poser une précision $d \in \R$ telle que 
    \[ p = f \pm d \] 

La taille de l'échantillon est donnée par : 
    \[ \boxed{n = z_\alpha^2 \frac{f(1-f)}{d^2}} \] 
