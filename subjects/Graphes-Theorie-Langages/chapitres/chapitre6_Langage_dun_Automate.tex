

\minitoc % Ajoute le sommaire local ici


% ==================================================================================================================================
% Introduction 

Maintenant que nous savons bien manipuler les automates fini déterministes et non déterministes, il serait utile, 
à partir d'un automate de pouvoir déterminer le langage qu'il reconnaît. Pour cela nous auront besoin de définir 
les systèmes d'équations au langages, d'introduire le Lemme d'Arden. Nous finirons par énoncer le théorème de Kleene 
caractérisant les langages automatiques. 

% ==================================================================================================================================
% Langage d'un automate fini

\section{Langage d'un automate fini}

\begin{definition}[Langage d'arrivée]
    Soit $ \mathcal{A} = \{Q, \Sigma, T, q_0, A\}$ un automate fini. On définit le langage d'arrivée à l'état $ q \in Q$, 
    noté $ L_q$ l'ensemble des mots de $ \Sigma^*$ dont la lecture par $ \mathcal{A}$ débute par $q_0$ et finit en $q$. 
\end{definition}

On peut alors définir la langage d'un automate comme :

\begin{proposition}
    Soit $ \mathcal{A} = \{Q, \Sigma, T, q_0, A\}$ un automate fini. Soit $\{L_q \; | \; i \in A \}$ 
    l'ensemble des langages d'arrivés aux états acceptants de l'automate $ \mathcal{A}$. On a alors l'égalité suivante : 
        \[ L( \mathcal{A}) = \sum_{q \in A} L_q \]
    Autrement dit, le langage de $ \mathcal{A}$ est la somme de tous ses langages d'arrivé aux états acceptants. 
\end{proposition}

\begin{definition}[Système d'équations aux langages]
    Soient un ensemble $X_1, \dots, X_n$ de langages sur un même alphabet $\Sigma$. 
    Un système d'équations aux langages est un ensemble d'équations de la forme : 
        \[ X_i = \sum_{j=1}^{n} a_{ij} X_j + b_j \quad \forall i \in \llbracket 1, n \rrbracket \] 
    où $ \forall i,j \in \llbracket 1, n \rrbracket, a_{ij} \in \Sigma, b_i \in \Sigma^*$
\end{definition}

\begin{proposition}
    Soit $ \mathcal{A} = \{Q, \Sigma, T, q_0, A\}$ un automate fini.
    A partir des définitions, on peut donc représenter le langage d'un automate par un système d'équations aux langages. 
    Elles associent à chaque $L_q$ une équation de langages dérivant de l'automate. 
    
    \vspace{0.2cm}
    
    Ainsi si un état $q$ a des transitions vers d'autres états selon les lettres $a \in \Sigma$ et si $q$ est un état 
    acceptant alors l'équation associée à $L_q$ est de la forme 
        \[ 
            \begin{cases}
                L_q = \sum_{(q,a,q') \in T} a L_{q'} + \{\varepsilon\}  \quad \text{si q est un état initial} \\ 
                L_q = \sum_{(q,a,q') \in T} a L_{q'} \quad \text{sinon}
            \end{cases}
        \] 
\end{proposition}

Il nous faut maintenant être capable de résoudre ces équations pour déterminer les $L_q$ et ainsi le langage de l'automate. 


% ==================================================================================================================================
% Lemme d'Arden

\section{Lemme d'Arden}

Le Lemme d'Arden permet de résoudre de telles équations. Il fut démontré en 1961 par Dean N. Arden. Voici son énoncé : 

\begin{lemma}[Arden]
    Soient $A$ et $B$ deux langages. Le langage 
        \[ \boxed{ L = A^*B } \]
    est le plus petit langage (pour l'inclusion ensembliste) qui est solution de l'équation 
        \[ \boxed{ (E) : X = (AX) \cup B } \] 
    De plus, si $A$ ne contient pas le mot vide $\varepsilon$, $A^*B$ est l'unique solution de cette équation. 
\end{lemma}

Ainsi, puisque chacune des équations aux langages vues précédement sont de la forme $ L = AL + B$, on peut donc 
résoudre de tels systèmes et calculer explicitement le langage d'un automate. 

\begin{example}

    Déterminons le langage de l'automate suivant :

    \begin{center}
        \begin{tikzpicture}[shorten >=1pt, node distance=3cm, on grid, auto]
            \node[state, initial] (1) {$1$};
            \node[state, accepting, right of=1] (2) {$1$};
            \node[state, accepting, right of=2] (3) {$3$};

            \path[->]
            (1) edge[loop below] node{$a$} (1)
            (1) edge[above] node{$b$} (2)
            (2) edge[loop below] node{$a$} (2) 
            (2) edge[above] node{$b$} (3) 
            (3) edge[loop below] node{$a$} (3);
        \end{tikzpicture}
    \end{center}

    Les langages d'arrivés vérifient donc le système d'équations aux langages suivant :
    \[ 
        \begin{cases}
            L_1 = L_1 a + \varepsilon \\ 
            L_2 = L_1 b + L_2 a \\ 
            L_3 = L_2 b + L_3 b 
        \end{cases}
    \] 
    Que l'on résout progressivement grâce au Lemme d'Arden : 
    \[ 
        \begin{cases}
            L_1 = \varepsilon a^* = a^* \\ 
            L_2 = L_2 a + a^* b \\ 
            L_3 = L_3 a + L_2 b 
        \end{cases}
        \quad \iff \quad 
        \begin{cases}
            L_1 = a^* \\ 
            L_2 = a^*ba^* \\ 
            L_3 = L_3a + a^*ba^* 
        \end{cases} 
    \quad \iff \quad 
        \begin{cases}
            L_1 = a^* \\ 
            L_2 = a^*ba^* \\ 
            L_3 = a^*ba^* b a^* \\ 
        \end{cases}
    \]

    D'où : 
        \begin{align*}
            L( \mathcal{A}) &= a^*ba^* + a^*ba^* b a^* \\ 
            &= a^*ba^*(\varepsilon + ba^*) 
        \end{align*}
    Donc $ \mathcal{A}$ reconnaît les mots qui contiennent 1 ou 2 b. 
\end{example}

Cet algorithme de résolution est très utile en terme pratique mais il apporte aussi une précisions supplémentaire sur 
les langages automatiques. En effet, à partir d'un automate (i.e langage automatique), on peut écrire le langage 
reconnu comme une expression régulière. D'où le résultat suivant : 

\begin{prop}[Langages Automatiques]
    Tout langage automatique peut être décrit par une expression régulière. 
\end{prop}

On en déduit donc le théorème de Kleene établissant définitivement le lien entre les langages réguliers et automatique : 

\begin{theorem}[Kleene]
    Les langages réguliers sont les mêmes que les langages automatiques. 
\end{theorem}




