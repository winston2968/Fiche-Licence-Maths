

\minitoc % Ajoute le sommaire local ici


% ==================================================================================================================================
% Introduction 

Précédement, nous avons vu que les langages automatiques sont de très bonnes propriétés. Ils sont stables 
pour la plupart des opérations définies sur les langages. De plus, le théorème de Kleene nous a permis d'établir 
le lien direct entre langages automatiques et langages réguliers. 
La simplicité de la représentation sagitale des automates permet de les implémenter facilement algorithmiquement. 
Il sont, de plus, facile à manipuler à la main et permettent de rapidement "voir" les langages reconnus. 

Cependant, cette simplicité a un certain coût, celui de ne pas pouvoir reconnaître des langages "compliqués", notamment
ceux où il faut "compter" les lettres. Un automate fini déterministe ne peut donc pas reconnaître le langage composé 
d'autant de $a$ que de $b$. 
On va donc chercher à indroduire une nouvelle théorie, celle des \textbf{grammaires formelles} qui nous permettra
de reconnaître de tels langages. 


% ==================================================================================================================================
% Grammaires Formelles

\section{Grammaires Algébriques}

\subsection{Contexte et définition}

Les grammaires formelles ont initialement été développés par des linguistes, notamment Noam Chomsky en 1955. 
L'objectif était de développer un méthode systématique de traduction entre différentes langues. 
Ils se sont alors heurtés au problème des mêmes mots qui admettent plusieurs traductions en fonction du contexte de la 
phrase et n'ont pas pu aboutir leur oeuvre. 

\vspace{0.3cm}

Or en informatique, pour l'étude de la syntaxe de langages de programmation, le problème du contexte ne se pose pas. 
Leur théorie a donc été récupérée pour la vérification syntaxique. 

\vspace{0.3cm}

L'idée est donc de représenter un langage \emph{récursivement} par un ensemble de règles de production composées 
d'un axiome de départ et de différentes règles de productions ou de réécriture. 

Nous utilisons souvent cette approche pour la gestion de types en Caml en définissant tes types récurisivement. 

\begin{definition}[Grammaire Algébrique]
    Une grammaire formelle est un quadruplet 
        \[ G = (\Sigma, V, S, P) \] 
    où 
    \begin{itemize}
        \item $\Sigma$ est un \emph{alphabet terminal} dont chaque élément ne peut se réécrire plus simplement. 
        \item $V$ est \emph{l'alphabet auxiliaire} (disjoint de $\Sigma$) composé de variables, qui ne 
        peuvent pas non plus se réécrire. 
        \item $S$ est la variable de départ, appelé axiome. 
        \item $P$ est un ensemble de règles dites \emph{de production} ou de réécriture du type 
            \[ X \longrightarrow w \quad X \in V \text{ et } w \in \left( V \cup \Sigma \right)^* \] 
    \end{itemize}
\end{definition}

Par convention, on notera toujours les variables en majuscule et les éléments terminaux en minuscules. 
En pratique, on regroupera plusieurs réécritures d'une même variable sur la même ligne en les séparant par 
des barres verticales de la forme :
    \[ X \longrightarrow w_1 | w_2 | \dots | w_p \iff 
        \begin{cases}
            X \longrightarrow w_1 \\ 
            X \longrightarrow w_2 \\ 
            \vdots \\ 
            W \longrightarrow w_p 
        \end{cases}
    \] 


\subsection{Réécriture d'un mot et langages algébriques}

L'objectif d'une grammaire algébrique, vous l'aurez compris, est de réécrire un mot récursivement jusqu'à arriver à des 
éléments terminaux. 

\begin{definition}[Réécriture d'un mot]
    Soit $ G = (\Sigma, V, S, P)$ une grammaire formelle. Soient $u,v \in \left( V \cup \Sigma \right)^*$ deux mots. 
    On dit que \emph{$u$ peut se réécrire en $v$ en une étape} et on note :
        \[ u \vdash v \] 
    si il existe des décompositions de $u$ et $v$ en 
        \[ u = u_1 X u_2 \text{ et } v = u_1 w u_2 \] 
    et que $G$ contient la règle de production :
        \[ X \longrightarrow w \] 
\end{definition}

Plus généralement, on peut définir la réécriture en plusieurs étapes de la forme : 

\begin{definition}[Réécriture (2)]
    Soit $ G = (\Sigma, V, S, P)$ une grammaire algébrique. Soient $u,v \in \left( V \cup \Sigma \right)^*$ deux mots.
    On dit que \emph{$u$ peut se réécrire en $v$} ou que \emph{$v$ dérive en $u$} en un nombre quelconque de fois si 
    il existe $u_1, \dots, u_p \in \left( V \cup \Sigma \right)^*$ tels que 
        \[ u \vdash u_1 \vdash u_2 \vdash \dots \vdash u_p \vdash v \] 
    On note alors 
        \[ u \overset{*}{\vdash} v \] 
\end{definition}

On peut maintenant définir les langages engendrés par des grammaires et les langages algébriques, le coeur de ce 
chapitre. 

\begin{definition}[Langage Engendré]
    La \emph{langage engendré} par une grammaire algébrique $ G = (\Sigma, V, S, P)$ est l'ensemble des mots de $\Sigma^*$ 
    qui dérivent de l'axiome $S$ en un nombre quelconque d'étapes. On le note, comme pour les automates, $L(G)$. 
\end{definition}

\begin{definition}[Langage Algébrique]
    Un langage engendré par une grammaire est appelé \emph{langage algébrique}
    (d'où l'appelation de \emph{grammaire algébrique}).
\end{definition}


\subsection{Arbre de dérivation d'un mot}

\begin{definition}[Arbre de dérivation d'un mot]
    Soit $ G = (\Sigma, V, S, P)$ une grammaire algébrique. On appelle l'arbre de dérivation de $w \in Sigma^*$ l'arbre dont :
    \begin{itemize}
        \item La racine est $S$ 
        \item Tous les sommets intérieurs appartiennent à $V$ 
        \item Toutes les feuilles appartiennent à $\Sigma \cup \{\varepsilon\}$ 
        \item Si un sommet intérieur $X$ a pour fils $X_1, \dots, X_p$ alors la règle 
            \[ X \longrightarrow X_1 | \dots | X_p \in P \] 
        \item Le mot obtenu en visitant les feuilles de l'arbre par un parcours profondeur préfixe de l'arbre 
            est un mot de $L(G)$
    \end{itemize}
\end{definition}

\begin{definition}[Grammaire Ambiguë]
    Soit $G$ une grammaire. On dit que $G$ est ambiguë s'il existe un mot $w_ \in L(G)$ possédant
    deux arbres de dérivation différents. 
\end{definition}

En pratique une telle grammaire est pas très utilisée. En effet, en informatique, il ne serait pas très pratique 
de pouvoir compiler un code en deux expressions différentes d'un autre langage. On ne saurait pas laquelle choisir. 
Il faut que la dérivation puisse se faire de façon unique.

\begin{example}[Grammaire et Arbre de dérivation]
    Soit $ \Sigma = \{a,b\}$ . 
    Soit la grammaire algébrique $G$ définie par les règles suivantes telles que $ V = \{S\}$ : 
        \[  P : 
            \begin{cases}
                S \longrightarrow aSa \\ 
                S \longrightarrow SbS \\ 
                S \longrightarrow \varepsilon
            \end{cases}
            \iff 
            \begin{cases}
                S \longrightarrow aSa \; | \; bSb \; | \;  \varepsilon
            \end{cases}
        \] 
    Soit $abaaba \in G$ on a alors l'arbre de dérivation suivant pour ce mot : 
    \begin{center}
        \begin{forest}
            [S
                [a]
                [S 
                    [b]
                    [S 
                        [a]
                        [S [$\varepsilon$]]
                        [a]
                    ]
                    [b]
                ]
                [a]
            ]
            \end{forest}
    \end{center}
    Cette grammaire reconnaît bien les palindrômes pairs. 
\end{example}

\begin{remark}
    Lors de la dérivations de mots par une grammaire, on remarque qu'il est plus facile que les règles de dérivation 
    possèdent des traces initiales ou finales uniques telles que les $a$ et les $b$. 
    Elles permettent d'identifier plus facilement les règles à utiliser pour les dérivations. 
\end{remark}


\subsection{Grammaires Régulières}

Nous allons ici faire le lien entre les deux modèles présentés précedement, les automates fini et les grammaires. 
Nous allons ainsi définir les \emph{grammaires régulières} qui permettent de représenter les automates fini déterministes sous 
la forme que nous venons d'introduire. 

\begin{prop}[Représentation d'un langage automatique]
    Soit $ \mathcal{A} = (Q, \Sigma, T, q_0, A)$ un automate fini déterministe. 
    Le langage $L$ reconnu par cet automate peur être engendré par la grammaire : 
        \[ G = (Q, \Sigma, q_0, P) \] 
    dont les variables auxiliaires sont les états de l'automate et où $P$ est l'ensemble 
    des productions de la forme :
        \[ q \longrightarrow x. T(q,x) \quad \text{où } q \in Q \text{ et } x \in \Sigma \] 
        \[ q \longrightarrow \varepsilon \quad \text{ si } q \in A \] 
\end{prop}

On peut donc représenter facilement n'importe quel langage automatique par une grammaire. 
D'où le théorème suivant. 

\begin{theorem}[Langage Automatique et Grammaire]
    Tout langage automatique (i.e reconnaissable par un automate fini) est algébrique (i.e reconnaissable par une grammaire formelle). 
    
    \vspace{0.2cm}

    L'ensemble des langages automatiques est même strictement inclus dans l'ensemble des langages algébriques. 
    Autrement dit, certains langages sont reconnaissables par une grammaire formelle mais pas par un automate. 
\end{theorem}

On définit ainsi les grammaires régulières. 

\begin{definition}[Grammaire Régulière]
    Une \emph{grammaire régulière} est une grammaire formelle dont toutes les règles de production de $P$ sont de la forme :
        \[ X \longrightarrow a.Y \quad \text{ou } X \longrightarrow \varepsilon \]
    où $X,Y \in V$ et $a \in \Sigma$.  
\end{definition}

Une grammaire régulière est donc conçue de façon à "laisser des traces" explicites de la structure des 
mots pour faciliter les dérivations. De même que précédement, on peut passer d'une grammaire régulière à un automate 
fini déterministe. 

\begin{proposition}[Représentation d'une grammaire régulière]
    Soit $G$ une grammaire régulière. Soit $L$ le langage reconnu par $G$.
    L'automate fini déterministe reconnaissante aussi $L$ est : 
        \[ \mathcal{A} = (V,\Sigma, T, q_0 = S, A) \] 
    dont les états sont les variables auxiliaires de $G$ et dont les transitions sont définies par : 
        \[ q' = T(q,x) \quad \text{si} \quad q \longrightarrow x q' \in P \] 
    et dont les états acceptants sont définis par :
        \[ q \in A \quad \text{si} \quad q \longrightarrow \varepsilon \in P \]  
\end{proposition}

\begin{remark}
    Grâce à cette propriété, les langages automatiques (réguliers) sont donc exactement les langages reconnus par des grammaires régulières. 
    D'où le nom...
\end{remark}

\begin{example}[Construction d'une grammaire régulière]
    Définissons le langage $L$ reconnaissant les mots contenant un nombre pair de $a$ et impair de $b$. 
    Alors ce langage est reconnu par l'AFn suivant : 
    \begin{center}
        \begin{tikzpicture}[shorten >=1pt, node distance=3cm, on grid, auto]
            \node[state, initial] (1) {$1$};
            \node[state, accepting, right of=1] (2) {$2$};
            \node[state, below of=1] (3) {$3$};
            \node[state, below of=2] (4) {$4$};

            \path[->]
            (1) edge[above, bend right] node{$b$} (2)
            (2) edge[above, bend right] node{$b$} (1)
            (2) edge[right, bend right] node{$a$} (4) 
            (4) edge[right, bend right] node{$a$} (2) 
            (1) edge[right, bend right] node{$a$} (3)
            (3) edge[right, bend right] node{$a$} (1)
            (3) edge[above, bend right] node{$b$} (4)
            (4) edge[above, bend right] node{$b$} (3);
        \end{tikzpicture}
    \end{center}
    
    D'après la propriété précédente, $L$ est reconnu par la grammaire régulière $G = (\Sigma, V, S, P)$
    où $V = \{S,A,B,C\}$ et les états sont représentés par :
        \[ 
            \begin{cases}
                1 \longrightarrow S \\ 
                2 \longrightarrow A \\ 
                3 \longrightarrow B \\ 
                4 \longrightarrow C
            \end{cases} \] 
    On peut ensuite déterminer les règles de production à partir du voisinnage sortant 
    de chaque état de l'automate. De plus, puisque $A$ est un état acceptant, on y rajouter $\varepsilon$. 
    \[ P : 
            \begin{cases}
                S \longrightarrow bA \; | \; aB \\ 
                A \longrightarrow bS \; | \; aC \; | \; \varepsilon \\ 
                B \longrightarrow aS \; | \; cB \\ 
                C \longrightarrow aA \; | \; bB 
            \end{cases} \] 
\end{example}


% ==================================================================================================================================
% Forme Normale de Chomsky 

\section{Forme Normale de Chomsky}

Tout comme les automates, on va chercher à simplifier les grammaires. 
Or ici, pour un langage donné il n'existe pas de forme minimale de grammaire qui l'engendre.

On va donc chercher à simplifier les grammaires dans le but d'obtenir des formes dites normales 
pour réduire le nombre de dérivations à faire pour un mot donné. L'idée est de ramener les arbres de dérivation 
à des arbres binaires donc ekes dérivations sont seulement de deux formes. 
On pourra donc calculer directement la profondeur de l'arbre de dérivation de n'importe quel mot du langage
engendré en fonction de son nombre de caractères. 

Pour cela, il faut définir un certain nombre de règles qui serviront à cette simplification. 
Commençons par un règle très simple : 

\begin{definition}[Règle 0]
    On peut toujours éliminer une règle de la forme $ X \longrightarrow X$. 
\end{definition}

\subsection{Règle 1 : Suppression des epsilon-productions}

On va chercher ici à supprimer toutes les $\varepsilon$ productions qui ne produisent rien dans la 
dérivation d'un mot et prennent beaucoup de temps et de place à exécuter. 

\begin{definition}[Règle 1 : Suppression des espilon-productions]
    Soit $G$ une grammaire formelle. On définit l'algorithme suivant pour 
    supprimer toutes les $\varepsilon$-productions de $ G$ en une grammaire équivalente : 
    \begin{enumerate}
        \item On cherche récursivement toutes les variables dont $\varepsilon$ dérive
        (i.e toutes les variables qui peuvent nous donner $\varepsilon$ à la fin). 
        \item On supprimer toutes les règles de la forme $X \longrightarrow \varepsilon$. 
        \item Pour toutes les variables $X$ de la forme $X \longrightarrow w$ 
        on rajoute toutes les productions $X \longrightarrow u$ avec $ u \not = u$ et $u$ est 
        obtenu à partir de $w$ en remplaçant une ou plusieurs variables indentifiées en 1. 
    \end{enumerate}
\end{definition}

\begin{example}
    Soit $G$ une grammaire d'alphabet $\Sigma = \{a,b\}$ et $V = \{S,A,B\}$ tel que :
        \[ P : 
            \begin{cases}
                S \longrightarrow AB | aS | A \\ 
                A \longrightarrow Ab | \varepsilon \\ 
                B \longrightarrow B | AS 
            \end{cases}
        \] 
    On cherche $G'$ telle sans $\varepsilon$-productions telle que :
        \[ L(G') \cup \{\varepsilon\} = L(G) \] 
    D'après la règle 1, toutes les variables produisent une $\varepsilon$-production. 
    On obtient donc la grammaire équivalente à $\varepsilon$-production près :
        \[ P' : 
            \begin{cases}
                S \longrightarrow AB | A | B | aS | a \\ 
                A \longrightarrow Ab | b \\ 
                B \longrightarrow AS | A | S 
            \end{cases} \] 
\end{example}

\subsection{Règle 2 : Élimination des cycles}

Dans les arbres de dérivation, les cycles peuveut conduire à des dérivations infinies. 
On va donc chercher à les supprimer. 

\newpage 

\begin{definition}[Règle 2 : Élimination des cycles]
    Soit $G$ une grammaire formelle. On définit l'algorithme suivant pour 
    supprimer tous les cycles de $ G$ en une grammaire équivalente. 
    Soit un cycle de la forme : 
        \[ X_1 \longrightarrow X_{n-1} \longrightarrow X_1 \] 
    Alors on remplace dans $P$ toutes les variables $X_i \forall i \in \llbracket 1, n-1 \rrbracket$ 
    par $X_1$. 
\end{definition}

\begin{example}
    En reprennant l'exemple précédent : 
    \[ P' : 
            \begin{cases}
                S \longrightarrow AB | A | B | aS | a \\ 
                A \longrightarrow Ab | b \\ 
                B \longrightarrow AS | A | S 
            \end{cases} \] 
    On détecte un seul cycle : $ S \longrightarrow B \longrightarrow S$. 
    On applique donc l'algorithme pour obtenir : 
    \[ P'' : 
            \begin{cases}
                S \longrightarrow AS | A | aS | a \\ 
                A \longrightarrow Ab | b \\ 
            \end{cases} \] 
\end{example}

\subsection{Règle 3 : Suppression des changements de variable}

Les changement de variable dans les arbres de dérivation font perdre du temps. 
En effet, ils augmentent la profondeur de l'arbre de dérivation sans produire de lettre. 
On va donc chercher à les supprimer avec la règle 3. 

\begin{definition}[Règle 3 : Suppression des changements de variable]
    Soit $G$ une grammaire formelle. Soit une dérivation de la forme $ A \longrightarrow B \longrightarrow C$. 
    Alors on peut la remplacer en $A \longrightarrow C$. 
\end{definition}

\begin{example}
    En reprenant l'exemple précédent, on peut supprimer les changements de variable : 
    \[ P'' : 
            \begin{cases}
                S \longrightarrow AS | A | aS | a \\ 
                A \longrightarrow Ab | b \\ 
            \end{cases} \] 
    On a donc les règles de productions suivantes en remplaçant :
    \[ P''' : 
            \begin{cases}
                S \longrightarrow AS | Ab | b | aS | a \\ 
                A \longrightarrow Ab | b 
            \end{cases} \] 
\end{example}


\subsection{Forme Normale de Chomsky}

Pour l'instant, on ne peut pas encore déterminer la profondeur de l'arbre de dérivation d'un mot. 
Il faut donc définir une forme, dite Normale de Chomsky, qui va permettre cette estimation. 

\begin{definition}[Forme Normale de Chomsky]
    Soit $ G = (\Sigma, V, S, P)$ une grammaire formelle. 
    Supposons que $G$ ne contienne ni $\varepsilon$-production, si changements de variables, 
    ni cycles. On définit alors la forme normale de Chomsky de ses règles de production $P$ comme ses 
    mêmes règles de production où seules deux formes sont autorisées : 
    \begin{itemize}
        \item Les productions de lettres de la forme $ X \longrightarrow a$ 
        \item Les dédoublements de variables de la forme $ X \longrightarrow AB$ 
    \end{itemize}
\end{definition}

\begin{example}
    En reprenant l'exemplement précédent, on obtient la forme normale de Chomsky suivante : 
    \[ \overset{\sim}{P} : 
        \begin{cases}
            S \longrightarrow AS | AY | b | XS | a \\ 
            A \longrightarrow AY | b \\ 
            X \longrightarrow a \\ 
            Y \longrightarrow b 
        \end{cases} \] 
\end{example}

On en déduit donc le théorème suivant :

\begin{theorem}[Majoration de la profondeur]
    Soit $w \in L(G)$ où $G$ est une grammaire formelle sous forme normale de Chomsky. 
    Notons $p$ la profondeur de l'arbre de dérivation de $w$ dans cette grammaire $g$. 
    On a alors l'inégalité suivante : 
        \[ \boxed{p \leqslant 2 |w| -1 } \] 
\end{theorem}
