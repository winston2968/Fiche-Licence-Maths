\minitoc % Ajoute le sommaire local ici

% ==================================================================================================================================
% Introduction 

Comme abordé dans le chapitre précédent, on cherche une méthode pratique et efficace pour déterminer si un mot appartient 
à un langage ou pas. On veut donc un modèle qui soit d'une part très pratique mathématiquement pour nous permettre de 
démontrer des choses dessus mais aussi facilemet implémentable algorithmiquement. 

Alerte Spoiler : de solides connaissances en théorie des graphes seront plus qu'utiles...

% ==================================================================================================================================
% Automates fini déterministes

\section{Automates fini déterministes}

\subsection{Définition et représentation}

\begin{definition}[Automate fini déterministe]
    Un Automate Fini Déterministe est un quintuplet :
        \[ \boxed{ \mathcal{A} = (\Sigma, Q, T, q_0, A) } \] 
    où :
    \begin{itemize}
        \item $\Sigma$ est un alphabet 
        \item $Q$ est un ensemble fini d'états (souvent une partie finie de $\N$)
        \item $T : Q \times \Sigma \longrightarrow Q$ est une application qui, à un état et une lettre associe un autre état. 
        \item $q_0$ un état initial
        \item $A \subseteq Q$ les états acceptants
    \end{itemize}
\end{definition}

On représentera ainsi un automate fini déterministe de plusieurs façons en fonction de son utilisation :
\begin{itemize}
    \item \textbf{Mathématique : } $\mathcal{A} = (\Sigma, Q, T, q_0, A) $
    \item \textbf{Table de Transition : } Elle va permettre de trouver rapidement les différents types d'états. 
    \item \textbf{Sagitale : } Sous forme de graphe 
    \item \textbf{En Python : } Nous représenterons les automates finis déterministes sous la forme de quintuplet aussi. 
\end{itemize}

Regardons en détail ces différentes représentations au travers d'un exemple. 

\begin{example}
    Soit $\mathcal{A} = (\Sigma, Q, T, q_0, A)$ un automate fini déterministe. 

    \textbf{Mathématiquement} nous avons :
    \begin{itemize}
        \item $Q = \{1,2,3\}$ 
        \item $\Sigma = \{a,b\}$ 
        \item $q_0 = 1$ 
        \item $A = \{3\}$ 
    \end{itemize}
    
    La représentation \textbf{sagitale}  de notre automate sera :
        
        \begin{center}
            \begin{figure}[h]
                \centering
                \begin{tikzpicture}[shorten >=1pt, node distance=2cm, on grid, auto]
    
                    % Définitions des styles d'états
                    \node[state, initial] (1) {1};
                    \node[state, right=of 1] (2) {2};
                    \node[state, accepting, right=of 2] (3) {3};
                
                    % Définition des transitions
                    \path[->]
                    (1) edge[above] node{a} (2)
                    (2) edge[above] node{b} (3)
                    (2) edge[loop above] node{a} ()
                    (1) edge[loop above] node{b} ()
                    (3) edge[loop above] node{a} ()
                    (3) edge[loop below] node{b} ();
                
                \end{tikzpicture}
            \end{figure}
        \end{center}

    On représente de façon doublement cerclée les états acceptants. Les états sont les sommets du graphe. 
    Les arcs valués sont les antécédants/images de la fonction $T$. 

    Enfin, la \textbf{table de transition} de l'automate est représentée par le tableau suivant :
    \begin{multicols}{2}
        \begin{center}
            \begin{tabular}{c|c|c}
                Q/$\Sigma$ & a & b \\ \hline 
                1 & 2 & 3 \\ \hline 
                2 & 2 & 3 \\ \hline 
                \textbf{3} & 3 & 3 
            \end{tabular}
        \end{center}

        La première ligne présente les lettres de l'alphabet et la première colonne les différents états. 
        Pour chaque état, le tableau donne l'état obtenu en fonction de la lettre suivante lue. 
    \end{multicols}
\end{example}

\subsection{Mot et Langage Automatique}


\begin{definition}[Lecture d'un mot]
    Soit $\Sigma$ un dictionnaire, $l \in \Sigma$ et $\mathcal{A}$ un automate. 
    On lit la lettre $l$ en dérivant d'un état $q \in Q$ vers un état $q' \in Q$ et si $T(q,l) = q'$. 
    On notera la lecture d'un mot de longueur $p$ par la lecture successive de ses lettres :
        \[ q_1 \;  \overset{l_1}{\longrightarrow} \; q_2 \; \dots \; q_{p-1} \; \overset{l_p}{\longrightarrow} \; q_p \] 
\end{definition}

Concrètement, pour la lecture d'un mot, on va partir de l'état initial et en fonction des valeurs de l'état courant et de 
la lettre lue, on va "bouger" d'un état (sommet) à un autre.

\begin{definition}[Mot refusé]
    Un mot $w \in \Sigma^*$ est dit refusé par un automate $\mathcal{A}$ si sa lecture à partir de l'état initial 
    se termine sur un état refusant ou ne se termine pas. Dans le cas contraire, $w$ est dit accepté. 
\end{definition}

\begin{definition}[Langage d'un Automate]
    Le langage d'un automate $\mathcal{A}$ est l'ensemble des mots acceptés par l'automate. 
    On le note $L(\mathcal{A})$. 
\end{definition}

On parle de \textbf{langage automatique} si il est reconnaissable par un automate. 
Deux automates sont dits \textbf{équivalents} si ils reconnaissent le même langage. 

\begin{definition}[Automate Complet]
    Un automate $\mathcal{A}$ est dit complet si 
        \[ \forall i \in Q, \forall l \in \Sigma, \quad T(i,l) \text{ est défini} \] 
    Autrement dit, un automate est dit complet si pour toute lettre et pour tout état fixés, il est possible de 
    changer d'état dans l'automate.  
\end{definition} 

\begin{definition}[Puit]
    Un état $q \in Q$ est un puit ssi 
        \[ \forall l \in \Sigma, \quad T(q,l) = q \]
    Un puit est un état duquel on ne peut sortir. 
\end{definition}

On définit aussi la notion de piège comme un puit refusant (i.e un puit dont l'état est refusant). 

Puisque la notion de complémentaire existe pour les langages et que les automates semblent très étroitement liés 
aux langages, on peut se demander si un automate peut admettre un complémentaire... 

Soit $\mathcal{A}$ un automate associé à un langage L. On cherche $\mathcal{A}' = \overline{\mathcal{A}}$. 
    \[ w \in \overline{\mathcal{A}} \iff w \not \in L \iff w \not \in L(\mathcal{A}) \] 
Il semble falloir que $\mathcal{A}$ soit complet. Si c'est le cas, on pourrait inverser $\mathcal{A}$ en inversant 
les sommets accpetants/refusants. 

\begin{proposition}
    Tout automate fini peut être complété par des puits refusants en un automate complet. 
\end{proposition}

\begin{theorem}[Complémentarité]
    L'ensemble des automates est stable par complémentarité. 
\end{theorem}

% ==================================================================================================================================
% Automates fini non déterministes

\section{Automates fini non déterministes}

Dans la section précédente nous avons vu un modèle très efficace pour vérifier l'appartenance d'un mot à un langage. 
En plus d'être facilement représentable en mémoire (i.e Python), il est facile à utiliser à la main et 
hérite de toute la théorie des graphes vue précédement ce qui en fait un très beau modèle mathématiquement parlant. 

Malgré tout cela, nous ne savons pas comment, à partir de plusieurs langages simples, constituer un automate 
reconnaissant la somme de ces langages. Nous n'avons pas défini de somme/union d'automate et celles-ci semblent 
assez difficiles vu la rigidité de notre modèle. 

Nous allons donc construire un modèle d'automate, appelé non déterministe, nous permettant de faire ces opérations 
d'union (que nous appelerons juxtaposition). Elles nous permettrons de construire des automates complexes à partir de 
somme de langages. 


\subsection{Généralités}

\begin{definition}[AFN]
    Un automate fini non déterministe $ \mathcal{A}$ est un quintuplet : 
        \[ \mathcal{A} := (Q,\Sigma, T,I,A) \] 
    tel que :
    \begin{itemize}
        \item $Q$ est l'ensemble des états de l'automate 
        \item $\Sigma$ est un alphabet 
        \item $T : Q \times \Sigma \longrightarrow \mathcal{P}(Q)$ est une application 
        \item $I \subseteq Q$ est l'ensemble des états initiaux 
        \item $A \subseteq Q$ est l'ensemble des états acceptants. 
    \end{itemize}
\end{definition}

\begin{remark}
    Plusieurs remarques concernant ce nouveau modèle. Premièrement, on remarque que l'on peut maintenant définir 
    des transitions multiples entre les états de l'automate. Deuxièmement, il existe plusieurs états initiaux. 
\end{remark}

\begin{definition}[Arbre de lecture]
    Soient $w \in \Sigma^*$, $L$ un langage sur $\Sigma$ et $ \mathcal{A}$ un automate reconnaissant $L$.
    L'arbre de lecture de $w$ par $ \mathcal{A}$ est l'arbre résultat du parcours de $ \mathcal{A}$ en fonction 
    des lettres de $w$. 
    
    Autrement dit dans l'arbre de lecture G de $w$, un noeud $a$ est le fils d'un 
    noeud $b$ si il existe une lettre $l$ de $w$ telle que $T(b,l) = a$. Une feuille de cet arbre est un état acceptant 
    ou refusant de l'automate. 
\end{definition}

\begin{definition}[Lecture acceptante]
    Soient $w \in \Sigma^*$, $L$ un langage sur $\Sigma$ et $ \mathcal{A}$ un automate reconnaissant $L$. 
    Une lecture de $w$ par $ \mathcal{A}$ est dite acceptante si \underline{il existe} un chemin 
    d'un état initial vers un état acceptant dans l'arbre de lecture de $w$ par $ \mathcal{A}$. 
\end{definition}

\begin{proposition}
    On peut dire plusieurs choses de la lecture d'un mot $w \in \Sigma^*$ par un automate $ \mathcal{A}$ :
    \begin{itemize}
        \item Si l'automate possède plusieurs états initiaux, la lecture produit un arbre de lecture pour chaque 
        état initial, nous auront donc une \underline{forêt d'arbres de lecture}. 
        \item Une lecture sera donc acceptante ssi il existe un arbre de la forêt dont au moins une des feuilles
        est un état acceptant. 
    \end{itemize}
\end{proposition}

\begin{example}
    Voyons tout cela sur un exemple. Soit $ \mathcal{A} := (Q,\Sigma, T,I,A)$ un automate fini non déterministe $ \mathcal{A}$
    sur l'alphabet $\Sigma : \{a,b\}$. Représentons notre automate sous forme de graphe orienté valué et sa table de transition :
    \begin{center}
        \begin{minipage}{0.45\textwidth} % Colonne pour le dessin
            \begin{tikzpicture}[shorten >=1pt, node distance=3cm, on grid, auto]
    
                % Définitions des styles d'états
                \node[state, initial] (1) {1};
                \node[state, accepting, right=of 1] (2) {2};
                \node[state, below=of 1] (3) {3};
                \node[state, initial, right=of 3] (4) {4};
            
                % Définition des transitions
                \path[->]
                (1) edge[above] node{$a$} (2)
                (1) edge[left] node{$a$} (3)
                (2) edge[right] node{$a$} (4)
                (3) edge[below] node{$b$} (4)
                (4) edge[sloped, below left] node{$b$} (1)
                (4) edge[loop below] node{$b$} (4);
            
            \end{tikzpicture}
        \end{minipage}%
        \hfill 
        \begin{minipage}{0.45\textwidth} % Colonne pour le tableau
            \begin{tabular}{c|c|c}
                T & a & b \\ \hline 
                $1$ & $\{2,3\}$ & X \\ \hline 
                \textcircled{2} & $4$ & X \\ \hline 
                $3$ & X & $4$ \\ \hline 
                $4$ & X & $\{1,4\}$ 
            \end{tabular}
        \end{minipage}
    \end{center}
    C'est un automate non déterministe puisqu'il contient deux transitions mutliples et deux états initiaux. 

    Posons $w := abba$, déterminons si ce mot appartient au langage $L( \mathcal{A})$. Nous allons construire 
    un seul arbre permettant d'avoir une condition suffisante de validation du mot. 

    \begin{center}
        \begin{minipage}{0.45\textwidth}
            \begin{tikzpicture}[scale=0.8, transform shape, shorten >=1pt, node distance=3cm, on grid, auto]
                
                % Racine
                \node[state, initial] (1) {1};
    
                % 1er niveau
                \node[state] (2) [below right=2cm and 2cm of 1] {2};
                \node[state] (3) [below left=2cm and 2cm of 1] {3};
    
                % 2nd niveau
                \node[state] (x1) [below=2cm of 2] {X};
                \node[state] (4) [below=2cm of 3] {4};
                
                % 3eme niveau 
                \node[state] (42) [below left=2cm and 2cm of 4] {4};
                \node[state] (12) [below right=2cm and 2cm of 4] {1};
    
                % 4 eme niveau 
                \node[state] (22) [below right=2cm and 2cm of 12] {2};
                \node[state] (32) [below left=2cm and 2cm of 12] {3};
                \node[state] (x2) [below=2cm of 42] {X}; 
    
                % Branches
                \path[->]
                (1) edge[left] node{a} (2)
                (1) edge[left] node{a} (3)
                (2) edge[left] node{b} (x1)
                (3) edge[left] node{b} (4)
                (4) edge[left] node{b} (12)
                (4) edge[left] node{b} (42)
                (12) edge[left] node{a} (22)
                (12) edge[left] node{a} (32)
                (42) edge[left] node{a} (x2);
    
            \end{tikzpicture}
        \end{minipage}
        \hfill 
        \begin{minipage}{0.45\textwidth}
            L'arbre de lecture du mot $abba$ contient un état acceptant comme feuille. 

            \vspace{1cm}

            Autrement dit, il existe un chemin menant d'un état initial à un état acceptant dans la forêt de lecture 
            de $abba$. Donc $abba \in L( \mathcal{A})$. 
        \end{minipage}        
    \end{center}
\end{example}

\subsection{Juxtaposition et construction d'un AFD}

\subsubsection{Juxtaposition}

Rappelons la problématique principale du chapitre. On cherche un modèle dérivant des AFD nous permettant de définir 
des opérations dessus et qui puisse être convertit algorithmiquement vers un AFD pour construire des automates 
d'un langage complexe à partir de langages plus simple. 

Autrement dit, on veut pouvoir appliquer l'opération de somme de langages sur les automates fini déterministes. 

\begin{definition}[Juxtaposition d'AFN]
    Soient $L_1$ et $L_2$ deux langages reconnus par deux automates $ \mathcal{A}_1$ et $ \mathcal{A}_2$. 
    Le langage $L_1 + L_2$ est reconnu par la \textbf{juxtaposition disjointe} de $ \mathcal{A}_1$ et $ \mathcal{A}_2$. 
\end{definition}

\begin{theorem}[Langages automatiques et stabilité]
    L'ensemble des langages automatiques est stable par somme. 
\end{theorem}

On peut maintenant, à partir de deux langages automatiques, définir le nouveau langage résultant de 
la somme des deux qui sera lui aussi automatique. Il suffit de faire la juxtaposition disjointe des deux automates
de départ. 

\subsubsection{Déterminisation}

\begin{theorem}[Existence et équivalence]
    Pour tout automate fini non déterministe, il existe un automate déterministe équivalent. 
\end{theorem}

Ce théorème est peut être un peu obscur mais permet de dire qu'il est toujours possible de passer d'un 
automate fini non déterministe (obtenu par exemple par juxtaposition) à un automate fini déterministe
qui reconnaisse le même langage. En tout cas, il nous dit qu'il en existe un...


L'intérêt de vouloir repasser chez les automates fini déterministes vient du fait que la lecture d'un mot par 
un AFN est de complexité exponentielle alors que la lecture d'un mot par un AFD est polynômiale... 
Lors de la vérification syntaxique de très long mots pour des langages très complexes, cela fait une différence 
cruciale pour la compilation. 

Ce processus est appelé \textbf{déterminisation} d'un AFN. 

\begin{proposition}
    Soit $ \mathcal{A} = (Q, \Sigma, T, I, A)$ un AFN. On cherche à construire un AFD $ \mathcal{A}'$ équivalent à $ \mathcal{A}$. 
    L'idée est de raisonner sur l'application $T : Q \times \Sigma \longrightarrow \Q$. Dans un AFN, cette application 
    n'est pas injective, on va donc poser une nouvelle application dans l'espace quotient de $Q \times \Sigma$ par le noyau de $T$. 
    Nous obtiendrons donc une application injective et donc un AFD. 
\end{proposition}


\begin{example}[Déterminisation]
    Soit $ \mathcal{A}$ l'automate défini sur l'alphabet $\Sigma = \{ a,b \}$ non déterministe et sa table 
    de transition suivants : 

    \begin{center}
        \begin{minipage}{0.45\textwidth} % Colonne pour le dessin
            \begin{tikzpicture}[shorten >=1pt, node distance=3cm, on grid, auto]
    
                % Définitions des styles d'états
                \node[state, initial] (1) {1};
                \node[state, accepting, right=of 1] (2) {2};
                \node[state, initial, below=of 1] (3) {3};
                \node[state, right=of 3] (4) {4};
            
                % Définition des transitions
                \path[->]
                (1) edge[above] node{$a$} (2)
                (1) edge[left] node{$a$} (3)
                (2) edge[right] node{$a$} (4)
                (3) edge[below] node{$b$} (4)
                (4) edge[sloped, below left] node{$b$} (1)
                (4) edge[loop below] node{$b$} (4)
                (2) edge[loop above] node{$b$} (2)
                (3) edge[loop below] node{$a$} (3);
            
            \end{tikzpicture}
        \end{minipage}%
        \hfill 
        \begin{minipage}{0.45\textwidth} % Colonne pour le tableau
            \begin{tabular}{c|c|c}
                T & a & b \\ \hline 
                $1$ & $\{2,3\}$ & X \\ \hline 
                \textcircled{2} & $4$ & $2$ \\ \hline 
                $3$ & $3$ & $4$ \\ \hline 
                $4$ & X & $\{1,4\}$ 
            \end{tabular}
        \end{minipage}
    \end{center}

    Déterminisons cet automate. Pour cela, nous allons renommer tous les états de l'automate en prenant en 
    compte les ensembles. 
    L'algorithme consiste donc à construire la table de transition du nouvel automate. 
    Pour chaque itération (i.e ajout d'une ligne dans la table), on effectue un parcours en largeur du nouvel automate 
    pour "découvrir" de nouveau état. On créé ainsi un "automate des parties". 

    \begin{figure}[h]
    \begin{center}
        \begin{minipage}{0.45\textwidth}
            \begin{tabular}{c|c|c}
                & a & b \\ \hline 
                $I = \{1,3\}$ & $II$ & $III$ \\ 
                $II = \{2,3\}$ & $IV$ & $V$ \\ 
                $III = \{4\}$ & - & $VI$ \\ 
                $IV = \{3,4\}$ & $VII$ & $VI$ \\ 
                $V = \{2,4\}$ & $III$ & $ VIII$ \\ 
                $VI = \{1,4\}$ & $ II$ & $VI$ \\ 
                $VII = \{3\}$ & $VII$ & $III$ \\ 
                $VIII = \{1,2,4\}$ & $IX$ & $VIII$ \\ 
                $IX = \{2,3,4\}$ & $IV$ & $VIII$ 
            \end{tabular}
            \caption{Table de l'AFD}
        \end{minipage}
        \hfill 
        \begin{minipage}{0.45\textwidth}
            \begin{tikzpicture}[scale=0.8, transform shape, shorten >=1pt, node distance=3cm, on grid, auto]
                % 1er niveau
                \node[state] (I) {I};

                % 2nd niveau
                \node[state] (II) [below left=2cm and 2cm of I] {II};
                \node[state] (III) [below right=2cm and 2cm of I] {III};

                % 3eme niveau
                \node[state] (IV) [below left=2cm and 2cm of II] {IV};
                \node[state] (V) [below right=2cm and 2cm of II] {V};
                \node[state] (VI) [below=2cm of III] {VI};

                % 4eme niveau
                \node[state] (VII) [below=2cm of IV] {VII};
                \node[state] (VIII) [below=2cm of V] {VIII}; 

                % 5eme niveau
                \node[state] (IX) [below=2cm of VIII] {IX};
                
                % Branches 
                \path[-]
                (I) edge[left] node{a} (II)
                (I) edge[left] node{b} (III)
                (II) edge[left] node{a} (IV)
                (II) edge[left] node{b} (V)
                (III) edge[left] node{b} (VI)
                (IV) edge[left] node{} (VII)
                (V) edge[left] node{} (VIII)
                (VIII) edge[left] node{} (IX);
            \end{tikzpicture}
            \caption{Automate des parties}
        \end{minipage}
    \end{center} 
    \end{figure}
\end{example}

% ==================================================================================================================================
% Automates fini à espilon transitions

\section{Automates fini à $\varepsilon$-transitions}

Définissons un nouveau type d'automates non déterministes. Les automates non déterministes à $\varepsilon$-transition. 
Il diffèrent des premiers puisque l'on va permettre le changement d'état sans lecture de lettres lors 
de la lecture d'un mot. Pour cela, nous allons définir une transition $\varepsilon$. 
Cela peut se voir comme une transition via le mot vide. 

\subsection{Généralités}

\begin{definition}[Automate Fini à $\varepsilon$-transitions (AFN$\varepsilon$)]
    Un automate fini à $\varepsilon$-transitions est un quintuplet: 
        \[ \mathcal{A} = (Q,\Sigma, T, I, A) \] 
    définit de la même façon que les automates précédents mais où: 
        \[ T : Q \times \Sigma \cup \{\varepsilon\} \longrightarrow \mathcal{P}(Q) \] 
\end{definition}

Ici, le changement spontané d'état sans lecture de lettre sera donc caractérisé par une nouvelle 
entrée dans la table de transition $\varepsilon$. 

\begin{example}
    Soit le langage $L := a^*b^*$. Un automate reconnaissant ce langage peut être écrit avec une 
    $\varepsilon$-transition. Ecrivons aussi sa table de transition: 
    \begin{center}
        \begin{minipage}{0.45\textwidth} % Colonne pour le dessin
            \begin{tikzpicture}[shorten >=1pt, node distance=3cm, on grid, auto]
    
                % Définitions des styles d'états
                \node[state, initial] (1) {1};
                \node[state, accepting, right=of 1] (2) {2};
            
                % Définition des transitions
                \path[->]
                (1) edge[above] node{$\varepsilon$} (2)
                (1) edge[loop below] node{$a$} (1)
                (2) edge[loop below] node{$b$} (2); 

            \end{tikzpicture}
        \end{minipage}%
        \hfill 
        \begin{minipage}{0.45\textwidth} % Colonne pour le tableau
            \begin{tabular}{c|c|c|c}
                T & a & b & $\varepsilon$ \\ \hline 
                1 & 1 & - & 2 \\ \hline 
                \textcircled{2} & - & 2 & - \\ \hline 
            \end{tabular}
        \end{minipage}
    \end{center}
\end{example}

Lors de la lecture d'un mot, les transitions peuvent être très aléatoires en fonction du nombre 
d'$\varepsilon$-transitions possibles de l'état courant. Un tel automate est donc hautement non déterministe. 


\begin{definition}[Lecture d'un mot]
    Soit $w = l_1 l_2 \dots l_n $ un mot sur $\Sigma$. 
    Soit $w' = a_1 a_2 \dots a_p$ le mot $w$ $\varepsilon$-complété (rembourré par des $\varepsilon$) tel que :
    \begin{itemize}
        \item $p \geqslant n$ 
        \item $ \forall i \in \llbracket 1, n \rrbracket, \; a_i \in \{l_1, \dots, l_n\} \cup \{\varepsilon\}$ 
        \item $l_1 l_2 \dots l_n = a_1 a_2 \dots a_p$ (du point de vue du produit de concaténation)
    \end{itemize}
    Une lecture du mot $w$ par $ \mathcal{A}$ est une lecture par $ \mathcal{A}$ de n'importe quel $w'$, 
    un $\varepsilon$-complété de $w$. 
\end{definition}

\begin{remark}
    Tout comme pour les automates précédants, un mot appartient au langage d'un automate ssi 
    la lecture de ce mot par celui-ci se finit sur au moins un état acceptant de l'automate. 

    \vspace{0.3cm}

    La lecture d'un mot par un AFN$\varepsilon$ conduira donc à la construction d'une forêt de lecture de ce mot 
    par l'automate. 
\end{remark}

\begin{example}[Lecture d'un mot]
    Soit l'automate fini non déterministe à $\varepsilon$-transitions précédent. 
    Soit le mot $abb$. Construisons la forêt de lecture de $abb$ par $ \mathcal{A}$. 

    \begin{center}
        \begin{tikzpicture}[shorten >=1pt, node distance=3cm, on grid, auto, scale=0.7]
    
            % Définitions des styles d'états
            \node[state, initial] (1) {1};
            \node[state, accepting, below=of 1] (2) {2};
            \node[state, right=of 1] (11) {1};
            \node[state, right=of 2, xshift=-1.1cm] (X1) {X};
            \node[state, right=of 11] (X2) {X};
            \node[state, below=of 11] (22) {2};
            \node[state, right=of 22] (222) {2};
            \node[state, accepting, right=of 222] (2222) {2};
        
            % Définition des transitions
            \path[->]
            (1) edge[above] node{$a$} (11)
            (1) edge[right] node{$\varepsilon$} (2)
            (2) edge[above] node{} (X1)
            (11) edge[above] node{} (X2)
            (11) edge[right] node{$\varepsilon$} (22)
            (22) edge[above] node{$b$} (222)
            (222) edge[above] node{$b$} (2222);
    
        \end{tikzpicture}
    \end{center}

    Il existe un chemin d'un racine vers une feuille acceptante dans la forêt de lecture: 
        \[ 1 \overset{a}{\longrightarrow} 1 \overset{\varepsilon}{\longrightarrow} 2 \overset{b}{\longrightarrow} 2 \overset{b}{\longrightarrow} 2 \] 
    Donc par définition, $abb \in L( \mathcal{A} )$. 
\end{example}

\subsection{Déterministation}

Dans le chapitre précédent, un théorème nous permet de dire que pour tout automate fini non déterministe, 
il existe un automate fini déterministe équivalent. Ainsi, à chaque fois que l'on considère un AFN$\varepsilon$, 
on est sûr qu'il existe un AFD équivalent. 

Pour la déterminisation d'un AFN$\varepsilon$, nous allons avoir besoin d'une définition supplémentaire... 

\begin{definition}[Clôture]
    Soit $\mathcal{A} = (Q,\Sigma, T, I, A)$ un AFN$\varepsilon$. 
    Pour tout $q \in Q$, on appelle clôture de $q$ l'ensemble des états accessibles à partir de $q$ sans lecture de 
    lettre lors de la lecture d'un mot dans l'automate. 

    Autrement dit, la clôture de $q$ est l'ensemble des états accessibles depuis $q$ dans le sous-graphe de $ \mathcal{A}$
    restreint aux $\varepsilon$-transitions. 
    
    On note $cl(q)$ la clôture de $q$. 
\end{definition}


L'idée de la déterministation d'un AFN$\varepsilon$ est, non plus de regrouper des états, mais d'étendre les transitions 
de l'automate à tous les états accessibles par $\varepsilon$-transitions. 

De plus, si un état acceptant est accessible uniquement par $\varepsilon$-transtion depuis un état, alors celui-ci 
hérite du caractère acceptant de l'état atteint. 

\begin{proposition}[Algorithme de Déterministation]
    Soit $\mathcal{A} = (Q,\Sigma, T, I, A)$ un AFN$\varepsilon$. On construit l'automate fini déterministe $ \mathcal{A}'$ 
    équivalent à $ \mathcal{A}$ en :
    \begin{enumerate}
        \item Calculer les clôtures de $ \mathcal{A}$ 
        \item Héritage : Tous les états dont la clôture contient un état acceptant sont acceptants. 
        \item Calculer les transitions étendues 
        \item Déterminisation de $ \mathcal{A'}$ par l'automate des parties (voir chap précédent)
    \end{enumerate}
\end{proposition}

\begin{example}[Déterminisation]
    Soit $ \mathcal{A}$ l'AFN$\varepsilon$ suivant : 
    
    \begin{center}
        \begin{minipage}{0.7\textwidth} % Colonne pour le dessin
            \begin{tikzpicture}[shorten >=1pt, node distance=3cm, on grid, auto]
    
                % Définitions des styles d'états
                \node[state, initial] (1) {1};
                \node[state, right=of 1] (2) {2};
                \node[state, accepting, above right=of 2] (3) {3}; 
                \node[state, below=of 3] (4) {4};
                
                % Définition des transitions
                \path[->]
                (1) edge[above] node{$a$} (2)
                (2) edge[bend left, above] node{$\varepsilon$} (3)
                (3) edge[bend left, above] node{$b$} (2) 
                (3) edge[right] node{$\varepsilon$} (4) 
                (4) edge[above] node{$a$} (2) 
                (4) edge[loop below] node{$b$} (4);

            \end{tikzpicture}
        \end{minipage}%
        \hfill 
        \begin{minipage}{0.25\textwidth} % Colonne pour le tableau
            \begin{tabular}{c|c|c|c}
                T & a & b & $\varepsilon$ \\ \hline 
                1 & 2 & - & - \\ \hline 
                2 & - & - & 3 \\ \hline 
                \textcircled{3} & - & 2 & 4 \\ \hline 
                4 & 2 & 4 & - 
            \end{tabular}
        \end{minipage}
    \end{center}

    \begin{enumerate}
        \item \textbf{Calcul des clôtures et héritage} 
            \begin{center}
                \begin{tabular}{c|c|c|c|c}
                    T & a & b & $\varepsilon$ & $cl(q)$ \\ \hline 
                    1 & 2 & - & - & $\{1\}$ \\ \hline 
                    \textcircled{2} & - & - & 3 & $\{2,3,4\} $ \\ \hline 
                    \textcircled{3} & - & 2 & 4 & $\{3,4\}$ \\ \hline 
                    4 & 2 & 4 & - & $\{4\}$
                \end{tabular}
            \end{center}
        \item \textbf{Calcul des transitions étendues}
            \begin{center}
                \begin{tabular}{c|c|c|c}
                    $\overset{\sim}{T}$ & a & b & $cl(q)$ \\ \hline 
                    1 & 2 & - & \\ \hline 
                    \textcircled{2} & 2 & $\{2,4\}$ & $\{2,3,4\}$ \\ \hline 
                    \textcircled{3} & 2 & $\{2,4\}$ & $\{3,4\}$ \\ \hline 
                    4 & 2 & 4 & \\
                \end{tabular}
            \end{center}
        \item \textbf{Déterminisation par l'automate des parties}
            \begin{center}
                \begin{minipage}{0.45\textwidth}
                    \begin{tabular}{c|c|c}
                        & a & b \\ \hline 
                        I $= \{1\}$ & II & X \\ \hline 
                        II $= \{2\}$ & II & III \\ \hline 
                        III $= \{2,4\}$ & II & III 
                    \end{tabular}
                \end{minipage}
                \hfill 
                \begin{minipage}{0.45\textwidth} % Colonne pour le dessin
                    \begin{tikzpicture}[shorten >=1pt, node distance=2cm, on grid, auto]
            
                        % Définitions des styles d'états
                        \node[state, initial] (1) {1};
                        \node[state, accepting, right=of 1] (2) {2};
                        \node[state, accepting, below=of 2] (3) {3}; 
                        
                        % Définition des transitions
                        \path[->]
                        (1) edge[above] node{$a$} (2)
                        (2) edge[loop above] node{$a$} (2) 
                        (2) edge[bend left, left] node{$b$} (3)
                        (3) edge[bend left, left] node{$a$} (2) 
                        (3) edge[loop below] node{$b$} (3);
        
                    \end{tikzpicture}
                \end{minipage}
            \end{center}
    \end{enumerate}
\end{example}

La déterminisation d'un AFN$\varepsilon$ nous permet donc de passer de la lecture d'un mot de complexité 
exponentielle (voire infinie) à un automate permettant de lire tous les mots avec une complexité linéaire. 


===================================================================================================
% Opérations entre automates

\section{Opérations entre automates}

L'objectif de cette section est de nous permettre de construire un automate reconnaissant le langage $L_1 \; T \; L_2$ 
où $T$ est une opération entre langages à partir des automates $ \mathcal{A}_1$ et $ \mathcal{A}_1$ reconnaissant 
respectivement les langages $L_1$ et $L_2$.
Pour cela, nous aurons besoin de définir formellement les opérations entre langages et les processus 
nous permettant de "passer aux automates".      

\subsection{Langages Elémentaires}

Toutes les opérations entre langages (par extension entre automates) se feront à partir de langages de 
élémentaires. 
Ces langages seront reconnus par des automates fixés, que nous connaissons d'avance. Nous en choisissons un nombre fini 
pour pouvoir les stocker en mémoire. Ils vont représenter les briques de base nous permettant de construire des automates plus 
complexes par la suite. 

\begin{proposition}[Langages Elémentaires]
    Soit $\Sigma = \{a,b\}$ un alphabet. On définit les langages élémentaires de $\Sigma$ comme :
    \begin{itemize}
        \item $L = \emptyset$ reconnu par l'automate : 
            \begin{center}
                \begin{tikzpicture}[shorten >=1pt, node distance=3cm, on grid, auto]
                    \node[state, initial] (1) {$1$};
                \end{tikzpicture}
            \end{center}
        \item $L = \{\varepsilon\}$ reconnu par l'automate :
            \begin{center}
                \begin{tikzpicture}[shorten >=1pt, node distance=3cm, on grid, auto]
                    \node[state, initial, accepting] (1) {$1$};
                \end{tikzpicture}
            \end{center}
        \item $L = \{a\}$ reconnu par le langage :
            \begin{center}
                \begin{tikzpicture}[shorten >=1pt, node distance=3cm, on grid, auto]
                    \node[state, initial] (1) {$1$};
                    \node[state, accepting, right of =1] (2) {$2$}; 
                    \path[->]
                    (1) edge[above] node{$a$} (2);
                \end{tikzpicture}
            \end{center}
        \item $L = \{a_1 \dots a_n\}$ reconnu par le langage :
            \begin{center}
                \begin{tikzpicture}[shorten >=1pt, node distance=3cm, on grid, auto]
                    \node[state, initial] (1) {$1$};
                    \node[state, right of =1] (2) {$2$}; 
                    \node[state, right of=2] (3) {$a_n$};
                    \node[state, accepting, right of=3] (4) {$n$};
                    \path[->]
                    (1) edge[above] node{$a_1$} (2)
                    (2) edge[above] node{$a_2$} (3)
                    (3) edge[above] node{$\dots$} (4);
                \end{tikzpicture}
            \end{center}
    \end{itemize}
\end{proposition}


\subsection{Automate Complémentaire}

A partir d'un langage, on peut définir son langage complémentaire. De même, on peut définir l'automate complémantaire 
reconnaissant ce langage. 

\begin{definition}[Automate Complémentaire]
    Soit $L$ un langage reconnu un automate $ \mathcal{A} = (Q,\Sigma,T,q_0,A)$ \textbf{complet}. 
    L'automate reconnaissant le langage complémentaire de $L$ est $ \overline{ \mathcal{A}} = (Q,\Sigma,T,q_0,Q \backslash A)$. 
\end{definition}

\begin{remark}
    Attention, pouvoir "passer au complémentaire" pour un automate, il faut que celui-ci soit complet. 
\end{remark}


\subsection{Somme d'automates}

\begin{definition}[Somme d'automates]
    Soient $L_1$ et $L_2$ deux langages respectivement reconnus par $ \mathcal{A}_1$ et $ \mathcal{A}_2$. 
    L'automate reconnaissant $L_1 + L_2$ est l'automate fini non déterministe construit par \textbf{l'union disjointe}
    de $ \mathcal{A}_1$ et $ \mathcal{ A}_2$. On l'appelle \textbf{automate somme} des automates $ \mathcal{A}_1$ 
    et $ \mathcal{A}_2$. 
\end{definition}

Cet automate n'est pas déterministe puisqu'il contient deux états initiaux mais que l'on peut déterminiser. 

\begin{example}
    Soient les langages suivants sur $\Sigma = \{a,b\}$ :
    \begin{align*}
        L_1 &= \{ \text{mots de } \Sigma \text{ terminant par } a \} \\ 
        L_2 &= \{ \text{mots de } \Sigma \text{contenant un nombre pair de } a \}
    \end{align*}

    Ces langages sont reconnus par les automates suivants :

    \begin{center}
        \begin{minipage}{0.45\textwidth}
            \begin{tikzpicture}[shorten >=1pt, node distance=3cm, on grid, auto]
                \node[state, initial] (1) {$1$};
                \node[state, right of=1, accepting] (2) {$2$};
                
                \path[->]
                (1) edge[above, bend right] node{$a$} (2)
                (2) edge[above, bend right] node{$b$} (1)
                (2) edge[loop right] node{$a$} (2)
                (1) edge[loop above] node{$b$} (1); 
            \end{tikzpicture}
        \end{minipage}%
        \hfill 
        \begin{minipage}{0.45\textwidth}
            \begin{tikzpicture}[shorten >=1pt, node distance=3cm, on grid, auto]
                \node[state, initial] (1) {$1$};
                \node[state, right of=1, accepting] (2) {$2$};
                
                \path[->]
                (1) edge[above, bend right] node{$a$} (2)
                (2) edge[above, bend right] node{$a$} (1)
                (2) edge[loop right] node{$b$} (2)
                (1) edge[loop above] node{$b$} (1); 
            \end{tikzpicture}
        \end{minipage}
    \end{center}

    On construit l'automate $ \mathcal{A}$ comme la juxtaposition disjointe des deux automates :
    \begin{center}
        \begin{minipage}{0.45\textwidth}
            \begin{tikzpicture}[shorten >=1pt, node distance=3cm, on grid, auto]
                \node[state, initial] (1) {$1$};
                \node[state, right of=1, accepting] (2) {$2$};
                
                \path[->]
                (1) edge[above, bend right] node{$a$} (2)
                (2) edge[above, bend right] node{$b$} (1)
                (2) edge[loop right] node{$a$} (2)
                (1) edge[loop above] node{$b$} (1); 
            \end{tikzpicture}
        \end{minipage}%
        \hfill 
        \begin{minipage}{0.45\textwidth}
            \begin{tikzpicture}[shorten >=1pt, node distance=3cm, on grid, auto]
                \node[state, initial] (3) {$3$};
                \node[state, right of=3, accepting] (4) {$4$};
                
                \path[->]
                (3) edge[above, bend right] node{$a$} (4)
                (4) edge[above, bend right] node{$a$} (1)
                (4) edge[loop right] node{$b$} (4)
                (3) edge[loop above] node{$b$} (3); 
            \end{tikzpicture}
        \end{minipage}
    \end{center}

    Que l'on doit ensuite déterminiser... 
\end{example}

Ces définitions de langages élémentaires nous permettent de savoir que tout langage réduit à un mot est automatique. 
On en déduit le théorème suivant : 

\begin{theorem}[Langage Fini]
    Tout langage fini est automatique. 
\end{theorem}


\subsection{Intersection d'Automates}

\begin{proposition}
    Soient $L_1$ et $L_2$ reconnus par les automates fini déterministes suivants $ \mathcal{A}_1$ et $ \mathcal{A}_2$. 
    On a alors : 
        \[ \overline{L_1 \cap L_2} = \overline{L_1} + \overline{L_2} \]
    On peut donc en conclure que : 
        \[ \boxed{L_1 \cap L_2 = \overline{\overline{L_1} + \overline{L_2}}} \]  
    Ainsi, l'intersection de deux automates peut être construite par somme et complémentarité. 
\end{proposition}

En pratique nous utuliseront plutôt l'automate des couples : 

\begin{definition}[Automate des Couples]
    Soient $ \mathcal{A}_1 = \{Q_1, \Sigma, T_1, Q_0, A_1\}$ reconnaissant le langage $L_1$ et 
    $ \mathcal{A}_2 = \{Q_2, \Sigma, T_2, q_0', A_2\}$ reconnaissant le langage $L_2$. On définit 
    l'automate des couples : 
        \[ A = \{Q_1 \times Q_2, \Sigma, T, (q_0,q_0'), A_1 \times A_2\} \] 
    reconnaissant le langage $L_1 \cap L_2$ où 
        \[ \forall i \in \Sigma, \forall q_1, q_2 \in Q_1, \forall q_1', q_2' \in Q_2 \text{ tels que } q_2 = T_1(q_1,j) \text{ et } q_2' = T(q_1',i) \]
        \[ \text{alors } T'((q_1, q_1'), i) = (q_2, q_2') \] 
\end{definition}

\begin{proposition}
    Si $ \mathcal{A}_1$ possède $n \in \N$ états et que $ \mathcal{A}_2$ possède $p \in \N$ états, alors 
    $ \mathcal{A}_1 \cap \mathcal{A}_2$ possède $n \times p$ états. 
    Pour de gros automates, cette méthode peut donc engendrer des très gros. 
    Même si la façon de les construire est assez simple et ressemble beaucoup à la déterminisation. 

    L'automate obtenu est, de plus, déterministe. 
\end{proposition}


\subsection{Différence d'Automates}

\begin{proposition}
    Soient $ \mathcal{A}_1$ et  $ \mathcal{A}_2$ deux automates reconnaissant respectivement les langages $L_1$ et $L_2$. 
    Pour reconnaître le langages $L_1 \backslash L_2$, on peut simplement construire l'automate reconnaissant :
        \[ L_1 \backslash L_2 = L_1 \cap \overline{L_2} \] 
\end{proposition}

\subsection{Langages Automatiques}

Essayons maintenant de déduire des conditions nécessaire pour qu'un langage soit automatique. 
D'après ce que l'on a vu grâce aux opérations, on peut déjà énoncer la proposition suivante : 

\begin{proposition}
    Les langages réguliers sont tous automatiques. 
\end{proposition}

Proposition que nous élargirons plus tard grâce au \emph{théorème de Kleene}. 

\begin{theorem}[Pompage (faible)]
    Soit $L$ un langage. Supposons $L$ automatique. 
    Soit $N \in \N$, alors pour tout $w \in L$ tel que $ |w| \leqslant N$, $w$ admet une décomposition 
    de la forme : 
        \[\exists w_1, w_2, w_3 \in L, w = w_1 w_2 w_3 \quad \text{avec } w_2 \not  = \varepsilon \]
    telle que cette décomposition soit gonflable :
        \[ \text{ i.e } \forall k \in \N, w_1 w_2 ^k w_3 \in L \]
\end{theorem} 

Ce théorème n'est pas idéal pour montrer qu'un langage est automatique. En revanche, sa contraposé de la forme :
    \[ \boxed{\text{Non Pompable } \Longrightarrow \text{Non Automatique}} \]  
Est en pratique très utilisée pour montrer qu'un langage n'est pas automatique. 

\begin{proposition}
    Ainsi, soit $L$ un langage. Pour montrer que $L$ n'est pas pompable, on montrera que 
    pour tout $N \in \N$, il existe $w \in L$ tel que $ |w| \geqslant N$ qui admette une décomposition de la forme : 
        \[ w = w_1 w_2 w_3 \quad \quad \text{avec } w_2 \not  = \varepsilon \] 
    telle que : 
        \[ \exists k \in \N, w_1 w_2^k w_3 \not \in L \] 
\end{proposition}

