% ==================================================================================================================================
% Introduction

\minitoc  % Affiche la table des matières pour ce chapitre

Vous voyez ce qu'est une Twingo ? Maintenant essayez d'y faire rentrer une équipe de rugby entière dedans...
On pourrait dire que l'intérieur de la Twingo est compact. Voilà ce que l'on va essayer de définir dans ce chapitre, les ensembles compacts. 

\vspace{0.3cm}

On nomme ici $E$ un espace métrique muni d'une distance $d$. 

% ==================================================================================================================================
% Points d'accumulation, recouvrements

\section{Points d'accupulation et recouvrement}

Avant de définir la notion de compact, il nous faire un effort théorique en définissant de nouveaux objects
qui vont nous aider à caractériser les compacts. 

\begin{definition}[Point d'accumulation]
    Soient $A \subset E$ et $x \in E$. On dit que $x$ est un point d'accumulation de $A$ si toute boule de rayon non nul 
    centrée en $x$ contient une infinité de points de $A$. On remarquera qu'il suffit seulement que cette boule contienne 
    un seul point de $A$ différent de $x$. 
\end{definition}

\begin{remark}
    Un point d'accumulation est un point adhérent. La réciproque est fausse en général. 
    On remarquera que pour qu'une partie admette un point d'accumulation, elle doit contenir un nomre infini de points. 
\end{remark}

\begin{definition}[Recouvrement]
    Soient $A \subset E$ et $(A_n)_{n \in \N}$ une suite de parties de $E$. On dit que $(A_n)_{n \in \N}$ constitue 
    un recouvrement de $A$ si $ A \subset \bigcup_{n \in \N} A_n $. 
\end{definition}

\begin{proposition}
    Soit $A \subset E$, on a les trois propriétés suivantes :
    \begin{itemize}
        \item Toute suite d'éléments de $A$ contient une suite partielle qui converge vers un élément de $A$. 
        \item Tout ensemble infini d'éléments de $A$ admet un point d'accumulation dans $A$. 
        \item De tout recouvrement de $A$ par des ensembles ouverts, on peut en extraire un recouvrement fini. 
    \end{itemize}
\end{proposition}


% ==================================================================================================================================
% Ensembles compacts

\section{Compacts}

\subsection{Définition et caractérisations}

\begin{definition}[Ensemble Compacts]
    Soit $K \subset E$, on dit que $K$ est compact si il satisfait au moins l'une des propriétés précedentes. 
\end{definition}

\begin{proposition}
    Un ensemble compact est borné et fermé. 
\end{proposition}

\begin{corollary}[Caractérisation des compacts de $\R^n$]
    Dans $\R^n$ les compact sont exactement les fermés bornés. 
\end{corollary}

\subsection{Propriétés}

\begin{corollary}[Théorème de Bolzano-Weierstraß]
    Dans $\R^n$ toute suite bornée possède une suite partielle convergente. 
\end{corollary}

\begin{corollary}[Théorème de Bolzano-Weierstraß]
    Dans $\R^n$ tout ensemble infini borné admet un point d'accumulation. 
\end{corollary}


