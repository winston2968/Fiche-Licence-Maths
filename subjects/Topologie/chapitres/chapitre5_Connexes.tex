% ==================================================================================================================================
% Introduction

\minitoc  % Affiche la table des matières pour ce chapitre

1793, Place de la Révolution, déconnexification de Louis XVI... 

\vspace{0.3cm}

Blague à part, nous allons ici définir la notion de connexité pour un ensemble. 
Conceptuellement, un ensemble connexe est un ensemble "en une seule partie". 
Il reste à le définir proprement. 

% ==================================================================================================================================
% Définition 

\section{Connexité}

\begin{definition}[Connexité]
    Soit $A \subset E$, on dit que $A$ est connexe (i.e "en une seule partie") si il est impossible de trouver 
    deux ouverts $B$ et $C$ de $E$, disjoints, tels que leur intersection respective avec $A$ soit non vide 
    et que leur union soit égale à $A$. 
\end{definition}





% ==================================================================================================================================
% Connexité et fonctions 

\section{Connexité et fonctions}
