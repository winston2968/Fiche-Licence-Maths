% ==================================================================================================================================
% Introduction 

\minitoc


% ==================================================================================================================================
% Ouverts et Fermés

\section{Ouverts et Fermés}

Une fois définies les notions de boules, d'intérieur et d'adhérence, on peut maintenant "caractériser" 
des ensembles/parties en fonction des propriétés de leur adhérene/intérieur/frontière. 
Cela va nous permettre de définir les ouverts et les fermés, deux "catégories" d'ensembles essentielles pour 
la plupart des raisonnements analytiques de topologie. 

\subsection{Définitions et Conventions}

\begin{definition}[Ensemble Ouvert]
    Soit $A \subseteq E$, on dit que $A$ est ouvert si $A = \int(A)$. 
    Autrement dit si pour tout élément de $A$, il existe une boule autour de cet élément entièrement 
    contenue dans $A$. 
\end{definition}

\begin{definition}[Ensemble Fermé]
    Soit $A \subseteq E$, on dit que $A$ est ouvert si $adh(A) = A$. 
\end{definition}

\begin{remark}
    Quelques conventions sur les ouverts et les fermés. 
    \begin{itemize}
        \item $ \forall a \in E, \forall r > 0 \quad B(a,r)$ est un ouvert et $\overline{B}(a,r)$ est un fermé. 
        \item $\emptyset$ et $\R^n$ dans $\R^n$ sont à la fois ouverts et fermés. 
        \item $[a,b[ \subset \R$ n'est ni ouvert, ni fermé.  
    \end{itemize}
\end{remark}

\begin{proposition}
    Soit $A \subset \R^n$, $A$ est ouvert ssi son complémentaire dans $\R^n$ est fermé. 
\end{proposition}

\begin{prop}[Réunion et Intersection]
    \begin{itemize}
        \item Une réunion quelconque d'ouverts est ouverte.
        \item Une intersection finie d'ouverts est ouverte.
        \item Une réunion finie de fermés est fermée.
        \item Une intersection quelconque de fermés est fermés
    \end{itemize}
\end{prop}

\begin{remark}[Moyen Mnémotechnique]
    Pour aider à la mémorisation, on peut s'aider de ces phrases :
    \begin{itemize}
        \item Les ouverts aiment s’étaler (réunion infinie), mais ils sont timides à se croiser (intersection finie).
        \item Les fermés aiment se serrer (intersection infinie), mais ne se dispersent pas trop (réunion finie).
    \end{itemize}
\end{remark}

\begin{example}[Réunion et Intersection]
    Quelques exemples pour retenir les propriétés :
    \begin{itemize}
        \item \textbf{Réunion infinie d'ouverts : } Soit $(A_n)_{n \in \N} = \left. \right] -\frac{1}{n}; \frac{1}{n} \left[ \right. $ une suite d'intervalles ouverts. 
        Alors la réunion infinie de tout ces intervalles reste ouverte :
            \[ \bigcup_{n = 1}^\infty  \left. \right] -\frac{1}{n}; \frac{1}{n} \left[ \right.  \quad \text{ouvert}\] 
        \item \textbf{Intersection Infinie de fermés : } De même, soit $(B_n)_{n \in \N} =  \left[-\frac{1}{n}; \frac{1}{n} \right]$
        une suite d'intervalles fermés. Alors leur intersection infinie reste fermée :
            \[ \bigcap_{n = 1}^\infty  \left[-\frac{1}{n}; \frac{1}{n} \right] \quad \text{fermé}\]
    \end{itemize}
\end{example}

\begin{proposition}
    Soit $A \subset E$, on dit que l'intérieur de $A$ est le plus grand ouvert contenu dans $A$ et 
    l'adhérence de $A$ est le plus petit fermé contenant $A$. 
\end{proposition}


\subsection{Ouverts/Fermés relativement}

\begin{definition}[Ouvert/Fermé relativement]
    Soient $A \subset E$ et $B \subset A$. On dit que $B$ est \textbf{ouvert relativement} à $A$ si il existe 
    une ouvert $V$ de $E$ tel que $B = A \cap V$. 

    D'autre part, on dit que $B$ est \textbf{fermé relativement} à $A$ si il existe un fermé 
    $U$ de $E$ tel que $B = A \cap U$. 
\end{definition}



% ==================================================================================================================================
% Ensembles Compacts

\section{Ensembles Compacts}

\begin{definition}[Recouvrement]
    Soit $(E,d)$ un espace métrique et $A \subset E$. Soit $I$ un ensemble quelconque et $(A_i)_{i \in I}$ une famille de 
    sous-ensembles de $E$. On dit que la famille $(A_i)_{i \in I}$ est un \emph{recouvrement de $A$} si : 
        \[ A \subset \bigcup_{i \in I} A_i \] 
    Lorsque les $A_i$ sont des ouverts, on parlera de recouvrement ouvert. 
\end{definition}

\begin{definition}[Compact]
    Soit $K \subset E$. On dit que $K$ est \emph{compact dans $E$} si : 
    \begin{itemize}
        \item[$\quad$] Toute suite à valeurs dans $K$ admet une sous-suite convergente dans $K$. 
        \item[$ \iff$] De tout recouvrement ouvert de $K$ on peut en extraire un recouvrement fini. 
    \end{itemize}
\end{definition}

\begin{proposition}
    Un ensemble compact est \emph{fermé et borné}. 
\end{proposition}

\begin{theorem}[Cas $\R^n$]
    Dans $\R^n$, les ensembles compacts sont exactement les fermés bornés. 
\end{theorem}

\begin{corollary}[Théorème de Bolzano-Weierstraß]
    Dans $\R^n$, toute suite bornée possède une suite partielle convergente. 
\end{corollary}

% ==================================================================================================================================
% Ensembles Connexes

\section{Ensembles Connexes}

Dans cette section, nous allons détailler la notion de connexité chez les ensembles. 
Intuitivement, un ensemble connexe se résumera à un ensemble "en un seul morceau". 

\begin{definition}[Connexité]
    Soit $A \subset E$. On dit que $A$ est connexe s'il est impossible de trouver $B,C \subset E$ tels que : 
    \begin{itemize}
        \item $B \cap C = \emptyset$ 
        \item $E = B \cup C$ 
        \item $E \cap B \not = \{\emptyset\} $ 
        \item $ E \cap C \not = \{\emptyset\} $ 
    \end{itemize}
\end{definition}

