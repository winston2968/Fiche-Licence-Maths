% ==================================================================================================================================
% Introduction

\minitoc  % Affiche la table des matières pour ce chapitre

Rappellons les propriétés du principal espace que nous allons considérer dans ces chapitres, $\R$. 

% ==================================================================================================================================
% Supremum, Infimum 

\section{Majorant, Minorant, Supremum, Infimum}

\subsection{Définitions}

Formellement, $\R$ est un corps totalement ordonné muni de 4 opérations compatibles avec cet ordre. 
Considérons ici un ensemble $E$ et $A \subseteq E$ une partie de $E$. 

\begin{definition}[Majorant]
    On appelle majorant de $A$, un élément de $M \in E$ supérieur à tous les éléments de $A$. 
    Plus formellement :
        \[ M \in E \text{ est un majorant de } A \iff \forall x \in A, M \geqslant x \] 
\end{definition}

\begin{definition}[Minorant]
    De même que pour les majorants, on appelle minorant de $A$ un élément $m \in E$ inférieur à tous les éléments de $A$. 
    Plus formellement :
        \[ m \in E \text{ est un minorant de } A \iff \forall x \in A, m \leqslant x \] 
    Autrement dit, tous les éléments de $A$ majorent $m$.
\end{definition}

\begin{example}
    Soit $E = \R$ et $A = [0,1] \subset E$. Alors $0,-1$ et $-\pi$ sont des minorants de $A$
    et $1, 7$ et $ e$ sont des majorants de $A$. 
\end{example}

\begin{definition}[Maximum/Minimum]
    Soit $A \subseteq E$. On appelle \textbf{maximum} de $A$ un élément $x \in A$ qui majore tous les éléments de $A$. 
    De même, un \textbf{minimum} de $A$ est un élément $x \in A$ qui minore tous les éléments de $A$. 
    On les notes généralement $\max(.)$ et $\min(.)$.
\end{definition}

\begin{remark}
    On repère rapidement la différence entre un maximum et un majorant. Un maximum a la propriété d'appartenir à la partie qu'il majore. 
    Idem pour un minimum. Ces objets sont quand même limités. Si on prends $A = ]0,1[ \subset \R$. 
    
    On remarque facilement 
    que l'on ne peut pas trouver de minimum à cette partie. Il existe une infinité de minorants mais si l'on souhaite minimiser 
    $A$ de façon "plus fine", cela risque de ne pas suffire. On va donc définir les supremum et infimum. 
\end{remark}

\begin{definition}[Supremum]
    Soit $A \subseteq E$, on appelle supremum de $A$ le plus petit des majorants de $A$. 
    Plus formellement :
        \[ x \in E \text{ est un supremum de } A \iff 
            \begin{cases}
                \forall a \in A, x \geqslant a \\
                m = \min(\{m \in E, \forall a \in A, m \geqslant a\})
            \end{cases}
        \]
    On le note généralement $sup$ et on parle de "borne sup". 
\end{definition}

\begin{definition}[Infimum]
    Soit $A \subseteq E$, on appelle infimum de $A$ le plus grand des minorants de $A$. 
    Plus formellement :
        \[ x \in E \text{ est un infimum } de A \iff 
            \begin{cases}
                \forall a \in A, x \leqslant a \\
                x = \max(\{m \in E, \forall a \in A, m \leqslant a\})
            \end{cases}
        \] 
    On le note $inf$ et on parle de "borne inf". 
\end{definition}

\begin{proposition}
    S'il exite, un supremum ou un infimum est unique. 
\end{proposition}

\begin{remark}
    Moins formellement, les bornes inf et sup permettent de résoudre beaucoup de problèmes de majoration/minoration
    fine en nous permettant de "regarder" de l'autre côté de notre partie $A \subseteq E$. 

    Les bornes inf et sup sont surtout utilisées dans des espaces tels que $\R$ et $\Q$ où les éléments sont "très proches"
    (nous définirons cette notion plus tard). Intuitivement dans des ensembles tels que $\N$ ou $\Z$ nous n'avons pas besoin 
    de tels objets. 
\end{remark}

\begin{example}
    Il peut arriver que l'on considère des parties qui n'admettent pas de majorants/minorants. 
    Par exemple, $A = \N \subset \R$. 
\end{example}

\subsection{Propriétés et caractérisations}

\begin{theorem}[Existence]
    Dans $\R$ toute partie non vide et majorée admet un supremum. 

    De même, {Toute partie non vide et minorée admet un infimum.}
\end{theorem}

\begin{proposition}
    Soit $A \subseteq \R$. $x \in \R$ est un supremum de $A$ ssi 
    \begin{itemize}
        \item $x$ majore $A$
        \item pour tout $\varepsilon > 0, \exists a \in A, \quad a > x - \varepsilon $
    \end{itemize}
    En français, un supremum de $A$ est un élément $x \in E$ qui majore $A$ et tel que pour tout réel 
    positif $\varepsilon$, on peut trouver un élément $a \in A$ entre $x$ et $x - \varepsilon$. 

    On a la même propriété pour les infimum. 
\end{proposition}

\subsection{Densité des rationnels dans les réels}

\begin{prop}[Archimède]
    Pour tout réel $x \in \R$, il existe un entier $n \in \N$, tel que $n > x$. 
\end{prop}

\begin{quote}
    \begin{footnotesize}
        \begin{proof}
            La démonstration se fait par l'absurde. 
        \end{proof}
    \end{footnotesize}
\end{quote}

On peut donc démontrer la proprosition principale de la densité de $\Q$ dans $\R$. 
On veut montrer qu'entre deux réels distincts, il existe une infinité de rationnels. 
Soient $x, y \in \R$ distincts. Il suffit juste de montrer qu'il existe un rationnel $r$ entre $x$ et $y$ et, par suite, 
puisqu'un rationnel est aussi un réel, on pourra trouver un autre rationnel entre $x$ et $r$ puis entre $r$ et $y$ et 
ainsi de suite... 

\begin{proposition}[Densité]
    Entre deux réels distincts, il existe une infinité de rationnels. 
\end{proposition}






