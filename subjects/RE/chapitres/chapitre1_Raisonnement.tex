% ==================================================================================================================================
% Introduction

\minitoc

Dans ce chapitre, nous allons définir les principaux quantificateurs et connecteurs logiques utilisés en mathématiques. 
Nous aborderons différents raisonnements très utilisés pour démontrer des propriétés. 
Nous présenterons aussi la structure la plus primitive des mathématiques qui, à elle seule, peut être développée en 
toute une théorie. 

% ==================================================================================================================================
% Assertions 

\section{Assertions}

En mathématiques, syntaxiquement, on utilise ce que l'on appelle des assertions composées de quantificateurs, de noms 
et de connecteurs logiques permettant d'affirmer des choses. Une assertion (phrase) peut être vraie ou fausse. 
On note généralement une assertion $\mathcal{A}$. 

\begin{example}
    \emph{"26 est plus petit que 50"} est un assertions vraie 
\end{example}

\subsection{Quantificateurs}

Un quantificateur est un symbole mathématique permettant de donner une quantité d'un objet. 

\begin{definition}[Lettre Muette]
    Comme son num l'indique, une lettre muette est un lettre de l'alphabet (généralement $x$) muette. 
    Elle peut être remplacée par n'importe quel objet en fonction de la définition de $x$, des propriétés 
    que l'on décide qu'elle doit respecter. 
\end{definition}

\begin{example}
    Une lettre muette $a$ peut représenter la valeur $2$ ou $1$. 
\end{example}

On utilise les lettres muettes en mathématiques pour démontrer des résultats généraux. 
En manipulant des lettres muettes et en montrant des propriétés sur ces dernières, on n'a pas besoin de regarder chaque 
cas particulier. 

\begin{definition}[Quantificateur Universel]
    Le quantificateur universel $\forall$ nommé "pour tout" permet d'évoquer cette notion de généralité. 
    Dans l'assertion $\forall x \; \mathcal{A}$, on veut dire "pour toute substitution de $x$ par un objet donné, l'assertion 
    $\mathcal{A}$ est vraie". 
\end{definition}

\begin{example}
    \emph{"Pour toute voiture, celle-ci possède des roues".}
\end{example}

\begin{definition}[Quantificateur Existenciel]
    Le quantificateur existenciel $ \exists$ nommé "il existe" permet d'énoncer une propriété 
    valable pour \textbf{au moins} un éléments. 
    Ainsi, dans l'assertion $ \exists x \; \mathcal{A}$ on veut dire que "il existe au moins un élément $x$ qui vérifie la condition $ \mathcal{A}$". 
\end{definition}

\begin{example}
    \emph{L'assertion "il existe une voiture bleue est assurément vraie" tandis que l'assertion "toutes les voitures sont bleu 
    ou rouge" n'était valable qu'en URSS. }
\end{example}







