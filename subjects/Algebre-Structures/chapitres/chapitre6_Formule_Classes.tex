% ==================================================================================================================================
% Introduction

\minitoc  % Affiche la table des matières pour ce chapitre

Maintenant que nous avons vu en détail les actions de groupes, nous allons d'abord essayer de trouver des relations entre les différents ensembles qu'elles 
nous permettent de définir. 
Dans un second temps nous définirons proprement les groupes quotient du point de vue des actions de groupes, puis 
nous finirons par présenter les théorèmes d'isomorphismes. 

% ==================================================================================================================================
% Formule des classes 

\section{Formule des classes}
\subsection{Relation Orbite/Stabilisateur}

Quand on manipule des actions de groupes, on s'apperçoit vite que l'orbite d'un élément 
est étroitement lié à son stabilisateur... 

\begin{proposition}
	Soient $(G,.)$ un groupe, $X$ un ensemble et $G_x$ le stabilisateur de $x \in X$ pour l'action de G sur X. 
	Alors l'application 
		\begin{align*}
			G/G_x &\longrightarrow \text{Orb}_G(x) \\
			g G_x &\longmapsto g.x 
		\end{align*}
	Est une bijection entre l'ensemble des classes à gauche du stabilisateur de $x$ et l'orbite de x sous G. 
\end{proposition}

A partir de cette application et à l'aide du formidable outil qu'est le théorème de Lagrange, on peut en déduire le corollaire suivant... 

\newpage

\begin{corollary}[Relation Orbite/Stabilisateur]
	Soit X un  G-ensemble, $x \in X$. On utilise les mêmes notations que précédement. On a alors :
	\begin{itemize}
		\item $ | \text{Orb}_G(x) | = (G : G_x) $
		\item $ |G| = |G_x| \times \text{Orb}_G(x) $ 
	\end{itemize}
\end{corollary}

\subsection{Formules des classes}

\begin{corollary}[Formule des classes]
	Soient X G un groupe et X un G-ensemble. Alors on peut \textbf{partitionner X en orbites disjointes} sous l'action de G, d'où :
	\[ X = \bigsqcup_{i=1}^r \text{Orb}_G(x_i) \quad \text{et} \quad |X| = \sum_{i=1}^{r} |\text{Orb}_G(x_i)| = \sum_{i=1}^{r} (G : G_x) = \sum_{i=1}^{r} \frac{ |G| }{ |G_{x_i}| } \]
	où $r$ est le nombre d'orbites distinctes de X sous l'action de G. 
\end{corollary}


% ==================================================================================================================================
% Groupes Quotient 

\section{Groupe Quotient}

Nous savons déjà que le noyau d'un morphisme de groupes est un sous-groupe distingué. 
Nous cherchons ici, à partir d'un sous-groupe distingué, à construire un morphisme dont il serait le noyau. 
Pour cela, nous devons définir les groupes quotients. 

\subsection{"Quotientons !"}

Soit G un groupe et H un sous-groupe de G. On définit l'ensemble $G/H = \{gH \; | \; g \in G\}$ comme l'ensemble quotient de G par la relation d'équivalence "appartenir à la même orbite"
pour l'action par translation de H sur G. 

\begin{definition}[Projection Canonique]
	Soient G un groupe et H un sous-groupe distingué de G. La projection canonique : 
	\[\pi : 
		\begin{cases}
			G \longrightarrow G/H \\ 
			g \longmapsto gH 
		\end{cases}
	\]
	est  \textbf{morphisme de groupes} qui envoie chaque élément de G sur sa classe d'équivalence modulo la relation "appartenir à la même orbite" évoquée précédement. 
\end{definition}

\begin{theorem}[Sous-groupe distingué et projection canonique]
	Soit G un groupe. Sous H un sous-groupe de G. 
	\begin{itemize}
		\item[] H \textbf{est distingué} dans G ssi la formule $g_1H \ast g_2H = (g_1g_2)H, \; \forall g_1,g_2 \in G$ 
		définit une loi de composition interne pour l'ensemble $G/H$ telle que la projection canonique sur G est un morphisme de groupes. 
		\item[]$\iff$  $G/H$ est un groupe pour la loi $\ast$  
	\end{itemize}
\end{theorem}

\newpage 

\begin{corollary}[Ordre du quotient]
	Si H est distingué dans G, alors l'ordre du quotient de G par H est égal au quotient de l'ordre de G par l'ordre de H. 

	D'où la formule suivante :
		\[ |G/H| = \frac{|G|}{|H|} \]
\end{corollary}

\begin{corollary}[Condition nécessaire et suffisante]
	Soit G un groupe et H un sous-groupe de G. H est distigué dans G ssi H est le noyau d'un morphisme de source G. 
\end{corollary}

\begin{remark}
	Pour montrer qu'un sous-groupe est distigué, on peut donc simplement vérifier que c'est le noyau de la projection canonique $\pi$ sur G. 
\end{remark}

\subsection{Propriétés du quotient}

\begin{theorem}
	Soient G et K deux groupes et $H \triangleleft G$. Soient $\phi : G \longrightarrow K$ un morphisme de groupes 
	et $\pi$ la projection canonique sur G. Alors :
	\begin{itemize}
		\item[] $H \subset \ker(\phi)$ 
		\item[]$\iff$  $\phi (H) = \{e_K\}$ 
	\end{itemize}
	Autrement dit, on peut factoriser le morphisme $\phi$ "en passant" par $G/H$. 
	Il existe donc un morphisme de groupes $\overline{\phi} : G/H \longrightarrow K$ tel que $\phi = \overline{\phi} \circ \pi$. 
	Ainsi, $\overline{\phi}$ est unique et définit par : 
		\[\overline{\phi} : 
			\begin{cases}
				G/H \longrightarrow K \\ 
				gH \longmapsto \overline{\phi}(gH) = \phi(g)
			\end{cases}
		\]
	L'image de $\overline{\phi}$ est celle de $\phi$ et son noyau est ker($\phi$)/H.
\end{theorem}

On peut représenter cette "factorisation" par le diagramme commutatif suivant :

\begin{multicols}{2}
	\[
		\begin{tikzcd}
			G \arrow[rr, "\phi"] \arrow[dr, "\pi" swap, bend right=20] & & K \\
			& G/H \arrow[ur, "\overline{\phi}" swap, bend right=20] &
		\end{tikzcd}
	\]

	Où $H = \ker(\pi)$. 

	\vspace{0.3cm}

	Dire que le diagramme est commutatif revient à dire que $\phi = \overline{\phi} \circ \pi$

\end{multicols}

% ==================================================================================================================================
% Théorèmes d'isomorphismes 

\section{Théorèmes d'isomorphismes}

Admettons que l'on ait un morphisme $\phi : G \longrightarrow K$ entre deux groupes G et K. 
Mais $\phi$ n'est pas injectif, notons alors son noyau $H = \ker(\phi)$. Comme vu précédement, 
H est un sous-groupe de G. On va alors pouvoir construire un isomorphisme de groupe de $G/H$ dans Im$(\phi)$. 

\vspace{0.5cm}

Cela revient à assimiler tous les éléments de G comme des classes d'équivalences et ainsi, ne plus avoir le problème 
du noyau qui "envoie" plusieurs éléments sur $e_K$. 
Pour obtenir un isomorphisme il nous faut donc aussi restreindre le domaine d'arrivée aux éléments atteignables par $\phi$. 

\begin{theorem}[Premier théorème d'isomorphisme]
	Soit $\phi : G \longrightarrow K$ un morphisme de groupes. Alors il existe un isomorphisme $\overline{\psi}$ tel que :
	\[ \psi : 
		\begin{cases}
			G/\ker(\phi) \longrightarrow \text{Im}(\phi) \\ 
			g\ker(\phi) \longmapsto \phi(g)
		\end{cases}
	\]
	De plus, \textbf{si $\phi$ est surjectif}, $\psi$ est un isomorphisme entre les groupes $G/\ker(\phi)$ et $K$.
	(si $\phi$ est surjectif, on peut étendre l'image de $\psi$ à l'image de $\phi$, donc à K tout entier)
\end{theorem}

\begin{remark}
	Soit G un groupe monogène engendré par $g \in G$. 
	D'après la propriété universelle de $\Z$, il existe un morphisme $\phi : \Z \to \Longrightarrow \langle g \rangle= G$ tel que $\ker(\phi) = n\Z$. 
	
	D'après le théorème 2.2, on peut quotienter l'espace de départ par le noyau du morphisme et ainsi obtenir le morphisme suivant :
		\[ \overline{\phi} :
			\begin{cases}
				\Z/n\Z \longrightarrow \langle g \rangle \\ 
				\overline{k} \longmapsto g^k \in G 
			\end{cases}
		\] 
	Enfin, d'après le premier théorème d'isomorphisme, puisque $\phi$ est surjectif, $\overline{\phi}$ est isomorphisme. 
	D'où : $$ \langle g \rangle \simeq \Z/n\Z $$  
\end{remark}

\begin{theorem}[Troisième théorème d'isomorphisme]
	Soient $K \subset H \subset G$ trois groupes. Tels que $H \triangleleft G$ et $ K \triangleleft G$. On a alors :
		\[ (G/K)/(H/K) \simeq G/H \] 
	\emph{Autrement dit, le quotient de deux groupes quotientés par le même sous-groupe normal est isomorphe au quotient des autres groupes.}
\end{theorem}

Ce théorème mérite quelques petites explications. Notons $\sigma, \pi, \mu$ les projections canoniques associés aux groupes quotient $G/H, G/K$ et $(G/K)/(H/K)$. 

On a donc $K \subseteq H \Longrightarrow \sigma(H) = H/K$, un sous groupe de $G/K$. 
Or, d'après le théorème 2.2, il existe un unique morphisme $\overline{\sigma} : G/H \longrightarrow (G/K)/(H/K)$ tel que $\overline{\sigma} \circ \pi = \mu \circ \sigma$. 

\vspace{0.3cm}

On a donc le diagramme commutatif suivant : 

	\[
	\begin{tikzcd}[scale=1.5, row sep=3em, column sep=4em]
		G \arrow[r, "\sigma"] \arrow[d, "\pi" swap] & G/K \arrow[d, "\mu"] \\
		G/H \arrow[r, "\exists ! \overline{\sigma}" swap, dashed] & (G/K)/(H/K)
	\end{tikzcd}
	\]

$\mu$ et $\sigma$ sont surjectifs donc $\mu \circ \sigma$ est surjectif donc $\overline{\sigma}$ est surjectif. 
De plus, $\ker(\mu \circ \sigma) = H$ donc $\overline{\sigma}$ est injectif. On en conclut que $\overline{\sigma}$ est 

\begin{example}
	Quelques exemples pour mieux apprécier le théorème précédent :
	\begin{itemize}
		\item A-t-on $(\Z/10\Z)/(2\Z/10\Z) \simeq \Z/2\Z$ ? 
		\item A-t-on $(\mathcal{S}_4/\mathcal{A}_4) \simeq (\mathcal{S}_4/\mathcal{V}_4)(\mathcal{A}_4/\mathcal{V}_4)$ ?
	\end{itemize}
\end{example}