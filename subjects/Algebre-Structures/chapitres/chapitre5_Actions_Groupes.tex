% ==================================================================================================================================
% Introduction

\minitoc  % Affiche la table des matières pour ce chapitre

Depuis le début de l'étude des groupes nous voyons ces objets simplement comme des ensembles auxquels ont 
a donné une structure via une application et quelques autres propriétés. Ces objets, en apparence très simples, 
sont pourtant très riches et peuvent être très variés. 

Une autre approche de l'étude de la théorie des groupes est de les faire agir sur un ensemble. L'un des exemples 
le plus intuitif pour comprendre cette notion est le groupe symétrique. En effet, il est définit comme l'ensemble des 
bijections sur un ensemble quelconque. Mais lorsqu'on choisit comme ensemble les entiers de 1 à $n \in \N$, on s'apperçoit
que notre groupe permute tout simplement des suites d'entier. 

% Dans ce chapitre, nous allons essayer de généraliser cette notion, en définissant tout d'abord les actions de groupes, 
% puis en démontrant quelques unes de leurs propriétés, et enfin, nous verrons comment relier ces nouvelles applications
% aux classes d'équivalences vues précédement. 

% ==================================================================================================================================
% Actions de groupes

\section{Actions de groupes, premières définitions}

\begin{definition}[Actions de groupe]
	Soit G un groupe et X un ensemble quelconque. On définit l'action à gauche de G sur X, souvent noté $ G \curvearrowright X$,
	l'application : 
		\[ \bullet : 
			\begin{cases}
				G \times X \longrightarrow X \\
				(g,x) \longmapsto g.x 
			\end{cases} \] 
	qui vérifie :
	\begin{enumerate}
		\item[i)] \textbf{Associativité Mixte :} $ \forall g,h \in G, \; \forall x \in X, \; (gh).x = g.(h.x)$ 
		\item[ii)] \textbf{Invariance par le neutre :} $ \forall x \in X, \; e_G . x = x $ 
	\end{enumerate}
	On appelle ainsi X un G-ensemble. 
\end{definition}

\begin{definition}[Point fixe et partie stable]
	Soient G un groupe et X un G-ensemble :
	\begin{itemize}
		\item $x \in X$ est un point fixe de G si, $ \forall g \in G, \; g.x = x$ 
		\item $Y \subset X$ est une partie stable de X si, $ \forall y \in Y, \forall g \in G, g(Y) \subset Y$.
	\end{itemize}
\end{definition}


% ==================================================================================================================================
% Morphisme Structurel

\section{Morphisme Structurel}

\begin{theorem}[Morphisme structurel]
	Soient G un groupe et X un G-ensemble. Il existe une bijection entre les opérations au gauche de G sur X 
	et l'ensemble des bijections de X sur lui-même ($S(X)$). 
	\[ \text{i.e} \quad 
		\begin{cases}
			G \times X \rightarrow X \\
			(g,x) \mapsto g.x
		\end{cases}
		\; \overset{\sim}{\longmapsto} \; \phi :
		\begin{cases}
			G \rightarrow S(X) \\ 
			g \mapsto \sigma_g 
		\end{cases}
	\]
	où $ \forall g \in G, \forall x \in X, \; \sigma_g(x) = g.x $
\end{theorem}


\begin{proposition}
	Soient G un groupe et $\phi$ l'action de G sur le G-ensemble X. 
	\begin{itemize}
		\item Soit $\psi$ un morphisme de groupes tq $ \psi : K \rightarrow G$, alors X est aussi un $\psi \circ \phi$-ensemble 
		(i.e on peut composer une action de groupe par la droite par un morphisme de groupe dont l'image est le groupe opérant).
		\item Soit $Y \subset X$, alors Y est aussi un G-ensemble. 
		\item Soit $H \leq G$, alors X est aussi un H-ensemble. 
	\end{itemize}
\end{proposition}

\begin{remark}
	Notons qu'une action droite d'un groupe G sur un ensemble X correspond à un anti-morphisme de groupe 
	(i.e du point de vue du morphisme structurel $ \forall g,h \in G, \phi(gh) = \phi(h)\phi(g)$).
\end{remark}

% ==================================================================================================================================
% Orbites et stabilisateurs 

\section{Orbites et Stabilisateurs}
\subsection{Définitions}

\begin{definition}[Orbite et Stabilisateur]
	Soit G un groupe et X un ensemble tels que $G \curvearrowright X$, et soit $x \in X$, on peut alors définir les deux ensembles suivants :
	\begin{itemize}
		\item \textbf{L'orbite} de x est le \underline{sous-ensemble} Orb$(x) \subseteq X$ tel que :
			\[ \boxed{ \text{Orb}_G(x) = \{g.x \; | \; g \in G\} } \] 
		\item \textbf{Le stabilisateur} de x est le \underline{sous-groupe} de G tel que :
			\[ \boxed{ \text{Stab}(x) = \{ g \in G \; | \; g.x = x \} }\] 
	\end{itemize}
	\emph{L'orbite d'un élément x de X correspond à tous les éléments de X que l'on peut atteindre sous l'action de G sur x.}
	\emph{Le stabilisateur de x de X correspond à tous les éléments de G qui laissent x invariant.}
\end{definition}

% \begin{prop}
%	Soit G un groupe et X un G-ensemble, alors le stabilisateur de $x \in X$ par G est un \textbf{sous-groupe} de G. 
% \end{prop}

\newpage 

\begin{definition}[Propriétés d'une action]
	Soit G un groupe et X un ensemble tels que $G \curvearrowright X$ on dit alors que G est :
	\begin{itemize}
		\item \textbf{Transitive} sur X s'il existe exactement une seule orbite dans X. 
		\item \textbf{Libre} sur X si tous les statbilisateurs sont triviaux (i.e réduits à l'élément neutre de G). 
	\end{itemize}
\end{definition}

\begin{prop}[Stabilité des orbites]
	Les orbites de X sont stables par G et toute union d'orbites de X est aussi stable par G. 
\end{prop}

\subsection{Orbites, stabilisateurs et relation d'équivalence}

\begin{proposition}
	Soit G un groupe et X un G-ensemble, alors :
	\begin{itemize}
		\item La relation $\sim$ définie sur X par 
			\[ x \sim y \; \text{si} \; x \in \text{Orb}(y) \] 
		est une relation d'équivalence sur X. 
		\item Les classes d'équivalences pour cette relation sont exactement les orbites de X pour l'action de G.
	\end{itemize}
\end{proposition}

\begin{prop}
	Les orbites de X par l'action de G forment une partition de X.
\end{prop}

\begin{prop}[Noyaux]
	Soient G un groupe et $\phi$ une action de groupe de G sur un ensemble X. 
	Le noyaux de $\phi$, ker$(\phi)$, est l'intersection de tous les stabilisateurs de X. 
	\[ \boxed{ \text{i.e} \quad \text{ker}(\phi) = \bigcap_{x \in X} \text{Stab}_G(x) }\] 
\end{prop}

\begin{definition}[Action fidèle]
	Soient G un groupe et X un G-ensemble pour une action $\phi$, on dit que $\phi$ est fidèle si ker$(\phi) = \{ e_G\}$. 

	\vspace{0.5cm}

	\emph{La fidélité d'une action de groupe peut être assimilé à l'injectivité.}
	
\end{definition}

% ==================================================================================================================================
% Actions Particulières

\section{Actions Particulières}
\subsection{Opération par translation}

Ici, nous allons étudier les propriétés des actions par translations. 
Jusqu'à maintenant, nous avons étudié des actions de certains groupes sur des ensembles. 
Or on peut aussi considérer un groupe comme un ensemble et donc faire agir un groupe \textbf{sur lui-même}\dots

\newpage 

\begin{definition}[Action par translation]
	Soit $(G,.)$ un groupe noté multiplicativement. On définit l'action à gauche de G sur lui-même :
		\begin{align*}
			G \times G &\longrightarrow G \\ 
			(g,h) &\longmapsto g.h = gh 
		\end{align*}
\end{definition}

\subsubsection{Théorème de Cayley}

\begin{theorem}[Cayley, 1878]
	Tout groupe fini d'ordre $n \in \N$ est isomorphe à un sous-groupe de $\mathcal{S}_n$. 
\end{theorem}

Ce théorème, bien que puissant n'est en réalité pas très utile. 
En effet pour un groupe d'ordre 6, on peut ainsi montrer qu'il est isomorhe à $\mathcal{S}_6$.
On connaît bien le groupe symétrique mais $\mathcal{S}_6$ est d'ordre $6! = 720$, au final, on ne récolte pas spécialement plus d'informations. 

\subsection{Opération par conjugaison}

Lorsque l'on a étudié le groupe symétrique, on a définit la conjugaison entre deux permuations. 
Ici, nous allons définir une action de groupe permettant de dire que deux permutations conjuguées
sont dans la même orbite..intéressant non ?

\begin{definition}[Action par conjugaison]
	Soit $(G,.)$ un groupe noté multiplicativement. On définit l'action par conjugaison de G sur lui-même :
		\begin{align*}
			G \times G &\longrightarrow G \\
			(g,h) &\longmapsto g.h = g h g^{-1}
		\end{align*}	
	On peut remarquer qu'elle est bien définie puisque $g^{-1}$ appartient bien à G. 
	Et $g h g^{-1}$ est bien dans G par stabilité. 
\end{definition}

\begin{remark}
	Permettons nous de faire quelques remarques...
	\begin{itemize}
		\item Si $G$ est abélien, alors l'action par conjugaison de G sur lui-même en triviale. 
		\item Ici, \textbf{l'orbite} d'un élément $h \in G$ est appelé \textbf{classe de conjugaison} de $h$. 
	\end{itemize}
\end{remark}

On peut ainsi définir de nouveaux objects... 

\begin{definition}[Centralisateur]
	Soit $(G,.)$ un groupe noté multiplicativement. On définit le centralisateur de $h \in G$ comme l'ensemble :
		\[ Z_G(h) = \{g \in G \; | \; ghg^{-1} = h\} \] 
	Le centralisateur d'un élément $h \in G$ peut se voir comme l'ensemble des éléments $g \in G$ qui \textbf{commutent} avec $h$.
	
	\vspace{0.3cm}

	Autrement dit, le centralisateur d'un élément est le stabilisateur du même élément pour l'action par conjugaison. 
\end{definition}