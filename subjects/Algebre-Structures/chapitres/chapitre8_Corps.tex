% ==================================================================================================================================
% Introduction

\minitoc  % Affiche la table des matières pour ce chapitre

Résumons, nous avons un groupe avec une loi supplémentaire qui peut être commutative et posséder un neutre. 
Grâce à cela, nous avons pu définir les idéeaux, un nouveau type de morphisme, etc... 

Seulement, on souhaiterai pouvoir définir des structures plus complexes comme avec les groupes tels que les anneaux 
quotient, avoir une notion de divisibilité, bref, de l'arithmétique. 

Pour cela, nous avons besoin d'étendre la notion d'anneaux aux corps...

% ==================================================================================================================================
% Corps 

\section{Corps, définition et propriétés}

\begin{definition}[Corps]
    On appelle $(K, +, \times)$ un corps si $(K,+,\times)$ est un anneau et $(K^*, \times)$ un groupe. 
    Si la loi $\times$ est commutative, on parlera alors de corps commutatif. 
\end{definition}

On peut bien évidement trouver une condition nécessaire et suffisante pour qu'un anneau soit un corps...

\begin{proposition}
    Soit $(K, +, \times)$ un anneau unitaire. $K$ est un corps si tout élément non nul est inversible pour $\times$. 
    Plus formellement ssi : 
        \[ \forall x \in K^*, \; \exists y \in K, \; xy = yx = 1_K \quad \text{ssi} \; \mathcal{U}(K) = K^*\] 
\end{proposition}

\begin{proposition}
    Un corps commutatif est aussi un anneau intègre. 
\end{proposition}

Tout comme pour les groupes et les anneaux, on peut définir un sous-corps d'un corps. 

\newpage 

\begin{definition}[Sous-corps]
    Soit $K$ un corps. Un partie $L \subseteq K$ est un sous-corps de $K$ si :
    \begin{enumerate}[label=\roman*)]
        \item $L$ est un sous-anneau de $K$
        \item $1_K \in L$ 
        \item $ \forall x \in L, x^{-1} \in L$
    \end{enumerate}
\end{definition}

En pratique pour montrer qu'un anneau est un corps, on préfèrera montrer que c'est un sous-corps quand c'est possible. 

\begin{example}
    $(\C,+,\times)$ est un corps. $(\Q,+,\times)$ et $(\R,+,\times)$ en sont des sous-corps. 
\end{example}

% ==================================================================================================================================
% Anneau Quotient

\section{Anneau Quotient}

Dans cette section, on considère un anneau A unitaire, non trivial. 
On va quotienter A par un de ses idéaux I bilatère différent de A. 
On a donc le théorème suivant : 

\begin{theorem}[Structure de l'anneau quotient]
    Soit A un anneau unitaire, non trivial et I un idéal bilatère de A différent de A. 
    L'ensemble $A/I$ est définit comme l'ensemble des classes du groupe $(A,+)$ modulo le 
    sous-groupe $I$ de $A$. On définit des opérations dessus : 
    \begin{itemize}
        \item \textbf{Loi multiplicative : } $ \forall a, b \in A, \overline{a} \times \overline{b} = \overline{a \times b}$. 
        \item \textbf{Loi additive : } $ \forall a,b \in A, \overline{a} + \overline{b} \in A/I$. 
    \end{itemize}
    Alors le groupe $(A,+)$ peut être étendu en un anneau $(A,+,\times)$ unitaire et non nul appelé 
    \textbf{anneau quotient de A par I}. 
\end{theorem}

\begin{proposition}
    Si A est commutatif, alors $A/I$ est aussi commutatif. 
\end{proposition}

Comme pour les groupes, on peut définir la projection canonique, porte d'entrée pour les théorèmes d'isomorphismes... 

\begin{definition}[Projection Canonique]
    Soit A un anneau et $A/I$ son anneau quotient par $I$. 
    On définit la projection canonique par : 
        \[ \pi : 
            \begin{cases}
                A \longrightarrow A/I \\ 
                a \longrightarrow \overline{a} 
            \end{cases} \]
    qui à chaque élément de $A$ lui associe sa classe dans $A/I$. Elle "projette" les éléments de $A$ sur leur classe. 
    Cette application est : 
    \begin{itemize}
        \item un morphisme d'anneaux 
        \item surjective 
        \item de noyau $I$ 
    \end{itemize}
\end{definition}

\newpage 

\begin{theorem}[Isomorphisme]
    Soient $A,B$ deux anneaux et $\phi : A \longrightarrow B$ une morphisme d'anneaux. 
    On peut alors construire un isomorphisme d'anneaux $\overline{\phi}$ tel que : 
        \[ A / \ker \phi \simeq Im \phi \] 
    \[
        \begin{tikzcd}
            A \arrow[r, "\phi"] \arrow[d, two heads, "\pi"] & B \\
            A/\ker \phi \arrow[ur, "\tilde{\phi}", dashed] & 
        \end{tikzcd}
    \]
\end{theorem}


% ==================================================================================================================================
% Anneau Z/nZ

\section{L'anneau $\Z / n\Z$}

Définissons maintenant proprement l'anneau $\Z /n \Z$ en nous servant du fait que $\Z$ est un anneau 
et $ n \Z, \forall n \in \N$ est un de ses idéaux. 

\begin{definition}[Anneau $\Z / n \Z$]
    Soit $n \in \N, n \geqslant 2$. On appelle anneau des entiers modulo $n$ l'anneua quotient de $\Z$ 
    par son idéal $n\Z$ noté $\Z /n \Z$. 
\end{definition}

\begin{theorem}[Inversibles dans $\Z / n \Z$]
    Le groupe des inversibles de $\Z / n \Z$ est : 
        \[ \boxed{ \mathcal{U}(\Z / n \Z) = \{ \overline{x} \; | \; x \in \llbracket 1, n \rrbracket \text{ tq } pgcd(x,n) = 1\} } \] 
    $(\mathcal{U}(\Z / n \Z), \times)$ est un groupe abélien d'ordre $\phi(n)$ où $\phi$ est l'indicatrice d'Euler. 
\end{theorem}

\begin{theorem}[Fondamental]
    On a les équivalences suivantes : 
        \[ n \text{ premier } \iff \Z / n \Z \text{ est un corps } \iff \Z / n \Z \text{ est intègre} \] 
\end{theorem}

Ce théorème nous permet de rapidement pouvoir affirmer qu'un $\Z/n\Z$ est un corps. 
En effet, l'utilisation d'un corps permet d'obtenir des inverses pour la loi $\times$. 

\begin{remark}[Notation]
    Lors que $n$ est premier, le corps $\Z / n \Z$ est noté $\F_p$. 
\end{remark}

Enonçons maintenant quelques théorèmes en vrac très utiles dans l'étude de la théorie des corps. 

\begin{theorem}[Restes Chinois]
    Soient $n,m \in \Z$. Si $ pgcd(n,m) = 1$ alors l'application : 
        \[ \varphi : 
            \begin{cases}
                \Z /nm \Z \longrightarrow \Z /n \Z \times \Z / m \Z \\ 
                \overline{x} \longmapsto (x \mod n, x \mod m)
            \end{cases} \] 
    est un isomorphisme d'anneaux. En particulier : 
        \[ \boxed{ pgcd(n,m) = 1 \iff \Z /nm \Z \simeq \Z /n \Z \times \Z / m \Z  } \]
\end{theorem}

Comme vu dans le cadre de la théorie des groupes, nous pouvons aussi généraliser le petit théorème de Fermat. 

\begin{theorem}[Petit théorème de Fermat]
    Soit $ p \in \N$. Si $p$ est premier alors 
        \[ \forall x \in \Z, \quad x^p \equiv x \mod p \] 
\end{theorem}

% ==================================================================================================================================
% Caractéristique d'un anneau 

\section{Caractéristique d'un Anneau}

Introduisons maintenant la caractéristique d'un anneau, concept fondamental en théorie des anneaux et en algèbre 
commutative. 

\begin{definition}[Caractéristique d'un anneau]
    Soit $A$ un anneau unitaire non trivial. 
    La \textbf{caractérisitique} de $A$ est le plus petit entier naturel $ n \in \N$ tel que $ n \times 1_A = 0_A$. 
    On la note $car(A)$. 
    Si cet entier n'existe pas, alors $car(A) = 0_\N$. 
\end{definition}

\begin{remark}
    La caractéristique d'un anneau permet de distinguer les anneaux divisibles comme $\Q,\R$ ou $\C$ des anneaux 
    cycliques tels que les $\Z / n \Z$. 
    Intuitivement elle "mesure" combien de fois on doit additionner $1$ pour obtenir $0$. 
\end{remark}

\begin{proposition}
    On peut définir la caractéristique d'un anneau plus formellement. 
    Soit $A$ un anneau initaire non trivial. Soit $f$ le morphisme d'anneau suivant : 
        \[ f : 
            \begin{cases}
                \Z \longrightarrow A \\ 
                k \longmapsto k \times 1_A 
            \end{cases} \]
    La caractéristique de $A$ est donc l'unique entier naturel $n \in \N$ tel que : 
        \[ \ker f = n \Z \] 
    On a alors $ \forall x \in A$ que $ n x = 0_A$. 
\end{proposition}

\begin{theorem}[Caractéristique des anneaux usuels]
    \begin{itemize}
        \item $car (\Z) = 0 $ 
        \item $car (\Z / n \Z) = n $
        \item Un anneau de caractéristique nulle est infini. 
    \end{itemize}
\end{theorem}

\begin{theorem}[Caractérisation des anneaux intègres]
    Soit $A$ un anneau intègre. Alors $car(A) = 0$ ou $car(A) = p$ avec $p$ premier. 
\end{theorem}

% ==================================================================================================================================
% Corps des fractions 

\section{Corps des fractions}

Soit $A$ un anneau commutatif, on cherche à construire une extension de $A$ pour laquelle tout élément de $A$ 
contient un inverse pour la loi $\times$ usuelle de $A$. 
Pour cela, nous allons être amenée à construire le corps des fractions de $A$. 

\vspace{0.3cm}

En guise d'exemple introductif, on peut considérer l'anneau $(\Z,+,\times)$. 
Dans cet anneau, $2$ ne possède pas d'inverse pour la loi $\times$. 
Pour cela, on peut "construire" un nouvel anneau (ici $\Q$), dans lequel $ \frac{1}{2} \in \Q$ sera l'inverse de 2 
pour la loi $\times$. 

\begin{proposition}
    Soit A un anneau commutatif. On cherche à construire une structure pour dans laquelle tout élément de $A$ possède 
    inverse pour la loi $\times$. 
    On définit ainsi le \textbf{corps des fractions} de $A$ comme l'ensemble : 
        \[ \boxed{ Fr(A) = \left\{ \frac{a}{b} \; | \; a,b \in A \text{ et } b \not = 0_A \right\} } \]
    Dans cette structure, ces objets sont définis comme des classes d'équivalences sous la forme : 
        \[ \frac{a}{b} = \frac{c}{d} \iff ad = bc \] 
    Cette forme d'égalité garantit la compatibilité des opérations/propriétés arithmétiques déjà utilisées. 
    Ainsi tout élément $b \in A$ qui ne possédait pas d'inverse dans $A$ en possède maintenant un dans $Fr(A)$ tel que :
        \[ \forall b \in A, \exists \frac{1}{b} \in Fr(A), \quad b \times \frac{1}{b} = \frac{1}{b} \times b = 1_A \] 
    Chaque élément non nul de $A$ possède maintenant un inverse pour $\times$ ce qui fait de $Fr(A)$ un corps. 
\end{proposition}

\begin{prop}[Calculs]
    Abordons quelques propriétés de calculs dans $Fr(A)$. Soient $a,c \in A$ et $k,b,d,u,v \in A$ non nuls. 
    On a les propriétés suivantes dans $Fr(A)$ : 
    \begin{itemize}
        \item $ \frac{ka}{kb} = \frac{a}{b} $ 
        \item $ \frac{a}{b} \times \frac{c}{d} = \frac{ac}{bd} $ 
        \item $ \frac{a}{b} + \frac{c}{d} = \frac{ad + cb}{bd} $ 
        \item $ \left\{ \frac{u}{v} \right\} ^{-1} = \frac{v}{u} $
    \end{itemize}
\end{prop}

\begin{proposition}
    Depuis le début, nous supposons que $Fr(A)$ est bien définit que tout se passe bien. Or, pour que $Fr(A)$ 
    soit bien définit, il faut que l'on puisse "plonger" $A$ dans son corps des fractions. 
    On définit ainsi le morphisme : 
        \[ \varphi : 
            \begin{cases}
                A \longrightarrow Fr(A) \\ 
                a \longmapsto \frac{a}{1_A}
            \end{cases} \] 
    C'est un morphisme d'anneaux par propriétés sur les fractions. 
    
    Pour "plonger" $A$ dans $Fr(A)$, il faut que $\varphi$ soit injectif. 
            \[ \text{i.e } \forall a,b \in Fr(A), \quad \varphi(a) = \varphi(b) \Longrightarrow a = b \] 
    Autrement dit dans la cas des fractions, si $\frac{a}{1} = \frac{b}{1}$ alors $a = b$. 
    Or si $A$ n'est pas \textbf{intègre}, il peut contenir des diviseurs de zéro $a$ et $b$ tels que : 
            \[ a \not  = 0 \; \text{et} \; b \not  = 0 \; \text{ mais } \; ab = 0_A \] 
    En terme de fractions, cela se traduit par : 
        \[ \exists a,b \in A, \text{ tq } \; \frac{a}{b} = \frac{0}{b} = 0 \] 
    Il se pourrait donc que $\varphi$ ne soit pas injectif. Ainsi $Fr(A)$ ne serait pas définit. 
\end{proposition}

\begin{prop}[Intégrité et corps des fractions]
    Soit $A$ un anneau unitaire non nul et $Fr(A)$ son corps des fractions. 
    Si il existe une morphisme d'anneau injectif $ \varphi : A \longrightarrow Fr(A)$ injectif, 
    alors nécessairement, $A$ est intègre. 
\end{prop}

\begin{theorem}[Existence corps des fractions]
    Tout anneau commutatif intègre $A$ admet un corps de fractions : 
        \[ Fr(A) = \left\{ \frac{a}{b} \; | \; a,b \in A, b \not  = 0_A \right\} \] 
\end{theorem}

