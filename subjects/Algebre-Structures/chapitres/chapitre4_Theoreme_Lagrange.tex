% ==================================================================================================================================
% Introduction

\minitoc  % Affiche la table des matières pour ce chapitre

Joseph Louis de Lagrange était un mathématicien, astronome et mécanicien italien de la fin du XVIII° siècle. 
Né en 1736 à Turin en Sardaigne, il mourut en 1813 à Paris. Initiateur du Calcul des Variations, il étudia la mécanique de l'artillerie, 
l'astronomie, au travers des variations de l'orbite lunaire, l'analyse et se trouva très fort en arithmétique. 
On lui doit le Théorème des Quatre Carrés et le Théorème de Lagrange sur les groupes d'ordre fini. 

La théorie des groupes n'existant pas à son époque, puisqu'elle émergeat qu'au XIX°s sous l'impulsion de Cauchy, Gauss et Galois,
Lagrange n'énonça pas son théorème tel que nous allons le voir, mais il lui est quand même dû.

% ==================================================================================================================================
% Relation d'équivalence 

\section{Relation d'équivalence}

De légers rappels sur les relations et les classes d'équivalences ne sont jamais de trop... 

\begin{definition}[Relation d'équivalence]
    Une relation d'équivalence $\sim$ entre deux éléments $x$ et $y$ d'un ensemble $X$ est une relation (RST):
    \begin{itemize}
        \item \textbf{Réflexive : } $x \sim x$ 
        \item \textbf{Symétrique :} $x \sim y \Longleftrightarrow y \sim x$ 
        \item \textbf{Transitive : } $ \forall z \in X, x \sim y $ et $y \sim z \Longrightarrow x \sim z $
    \end{itemize}
\end{definition}

\begin{remark}
    Etre isomorphe est une relation d'équivalence sur les groupes. 
\end{remark}

\begin{definition}[Classes d'équivalence]
    Soit $\sim$ une relation d'équivalence sur un ensemble $X$. Une classe d'équivalence de $x \in X$, 
    souvent noté $\overline{x}$ est l'ensemble :
    \[ \overline{x} = \{y \in X, x \sim y\}\]
\end{definition}

\begin{remark}
    On remarquera que les classes d'équivalences sur un ensemble forment une partition de cet ensemble. 
\end{remark}

\begin{definition}[Projection Canonique]
    Soit $\sim$ une relation d'équivalence sur un ensemble $X$. On appelle \textbf{ensemble quotient de $X$ par $\sim$} l'ensemble :
        \[ X / \sim = \{\overline{x}, x \in X \} \]
    On définit alors la \textbf{projection canonique} comme l'application :
        \[ \pi :
            \begin{cases}
                X \longrightarrow X / \sim \\ 
                x \longmapsto \overline{x} 
            \end{cases}
        \]
    \emph{"C'est l'application, qui, à x lui associe sa classe d'équivalence pour la relation $\sim$."}
\end{definition}

\begin{remark}
    La projection canonique est surjective puisqu'une classe d'équivalence n'est jamais vide. 
\end{remark}

% ==================================================================================================================================
% Classes à gauche 

\section{Classes à gauche}

\begin{definition}[Classe à gauche]
    Soit G un groupe noté multiplicativement et H un sous-groupe de G. La relation $\sim_H$ est une relation d'équivalence sur G telle que :
        \[ \forall g_1,g_2 \in G, \quad g_1 \sim_H g_2 \Longleftrightarrow \exists h \in H, g_1 = h.g_2 \] 
    Ses classes d'équivalences sont les ensembles $gH$ notés $gH = \{ gh, h \in H \}$ de G. On les appelles \textbf{classes de $g$ à gauche} de
    G modulo H.
\end{definition}

\begin{remark}
    On peut définir de façon similaire les classes à droite de G modulo H, mais dans la pratique, nous n'utiliserons que les classes à gauche. 
\end{remark}

\begin{prop}[Classes à gauche et cardinal]
    Soit G un groupe et H un sous-groupe de G. Toutes les classes à gauche de G modulo H ont le même cardinal que H.
\end{prop}

\begin{quote}
	\footnotesize
	\begin{proof}
        Montrons que toutes les classes à gauche de G modulo H sont isomorphes à H.
		Soit $g \in G$. Considérons $\phi_g$ l'application telle que  :
            \[ \phi_g :
                \begin{cases}
                    H \longrightarrow gH \\
                    h \longmapsto gh 
                \end{cases}
            \] 
        \begin{enumerate}
            \item \textbf{Injectivité :} Soient $h_1,h_2 \in H$ tels que :
                \begin{align*}
                    & \phi(h_1) = \phi(h_2) \\
                    \Longleftrightarrow \; & g h_1 = g h_2 \\ 
                    \Longleftrightarrow \; & h_1 = h_2
                \end{align*}
            \item \textbf{Surjectivité :} Soit $x \in gH$, $ \exists h \in H$ tel que $\phi(h) = gh = x$
        \end{enumerate}
        Donc $\phi$ est bijective. D'où le résultat. 
    \end{proof}
	\normalsize
\end{quote}

\newpage 
\subsection{Ensemble Quotient et Indice}

\begin{definition}[Ensemble Quotient]
    Soit $(G,.)$ un groupe noté multiplicativement et $H \leq G$. L'ensemble quotient de $G$ par la relation d'équivalence 
    $\sim_H$, noté $G/H$ est l'ensemble $\{gH, \; g \in G\}$ des classes à gauche de G modulo H. 

    \vspace{0.5cm}

    \textbf{L'indice} de H dans G, noté $(G:H)$ est le cardinal de l'ensemble quotient $G/H$.
    Il correspond au nobre de classes d'équivalences différentes pour la relation $\sim_H$ dans G. 
\end{definition}

\begin{theorem}[Cardinal du groupe et ensemble quotient]
    Soit $(G,.)$ un groupe et $H \leq G$ alors : 
        \[ |G| = |H| \times (G:H) \]
\end{theorem}

\subsection{Théorème de Lagrange appliqué aux groupes finis}

\begin{theorem}[Théorème de Lagrange]
    Soit G un groupe fini et $H \leq G$, alors {l'ordre de H divise l'ordre de G}.
\end{theorem}

\begin{corollary}[Ordre d'un élément]
    Soit G un groupe d'ordre fini et $g \in G$ alors $ \text{ord}(g) \mid  \text{Card}(G)$. 
\end{corollary}

\begin{proposition}[Sous-groupe d'indice 2]
    Soit G un groupe et $H \leq G$, si H est d'indice 2 alors il est distingué dans G. 
\end{proposition}

\begin{quote}
	\footnotesize
	\begin{proof}
        Soit $g \in G$. Distinguons deux cas :
        \begin{itemize}
            \item si $g \in H$ alors comme H est un sous-groupe $ \forall h \in H, ghg^{-1} \in H$ 
            \item si $g \in G \backslash H$. H est d'indice 2 donc G n'a que deux classes modulo H, 
            $H = eH = He$ et puisque les classes forment une partition de G, $G \backslash H$ forme la deuxième classe 
            à gauche et à droite. 

            Donc $G \backslash H = gH = Hg$ et $ \forall h \in H, \; \exists h' \in H$ tel que $gh = h'g$. 

            C'est à dire que $ghg^{-1} = h' \in H$. 
        \end{itemize}
        Donc $H \triangleleft G$. 
    \end{proof}
	\normalsize
\end{quote}