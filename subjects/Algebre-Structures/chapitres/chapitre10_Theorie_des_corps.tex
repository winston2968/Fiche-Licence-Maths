% ==================================================================================================================================
% Introduction

\minitoc  % Affiche la table des matières pour ce chapitre

Dans ce chapitre, nous allons plonger plus profondément dans la théorie des corps en abordant les extensions de corps. 

Dans tout ce chapitre, sauf mention contraire, nous considérons $\K$ comme un corps commutatif. 

% ==================================================================================================================================
%  Extension de corps 

\section{Extension de corps}

\begin{definition}[Extension de corps]
    Soit $\K$ un corps. On dit que $\K$ est une extension du corps $\Fc$ si $\K$ est un sous corps de $\Fc$. 
    On la note $\K/\Fc$. 
\end{definition}

\begin{example}
    $\C$ est donc une extension de corps de $\R$ de même que $\R$ est une extension de corps de $\Q$. 
\end{example}

% ==================================================================================================================================
%  Éléments algébriques et transcendants

\section{Éléments algébriques et transcendants}

\begin{definition}[Éléments algébrique]
    Soit $\K/\Fc$ deux corps. Soit $ \alpha \in \K$, on dit que $\alpha$ est algébrique dans $\Fc$ s'il 
    est racine d'un polynôme $P$ à coefficients dans $\Fc$ (i.e $P(\alpha) = 0$). 
\end{definition}

\begin{example}
    Ainsi, $i$ est algébrique dans $\C$ puisqu'il est racine de $X+1 \in \R[X]$ où $\C/\R$. 
\end{example}

On peut remarquer que tout éléments d'un corps $\K$ est algébrique sur $\K$ car il est racine du polynôme $X - \alpha$. 

\newpage 

\begin{definition}[Élément Transcendant]
    Soit $\K/\Fc$ deux corps. Soit $\alpha \in \K$, on dit que $\alpha$ est transcendant sur $\Fc$ s'il n'est pas algébrique sur $\Fc$. 
    Autrement dit, $\alpha$ est transcendant sur $\Fc$ ssi 
        \[ \forall P \in \Fc[X], P(\alpha) \not = 0 \] 
\end{definition}

Il est important de toujours spécifier dans quel corps on se trouve. En effet, un corps peut avoir une infinité de sous-corps, 
dont chacun a des propriétés différentes. Ainsi un élément transcendant/algébrique dans l'un peut ne pas l'être dans l'autre. 

% ==================================================================================================================================
%  Polynômes et isomorphismes 

\section{Polynômes et isomorphismes}

\subsection{Polynôme Minimal}

Soient deux corps $\K/\Fc$. Soit $ \alpha \in \K$. Soit le morphisme de substitution suivant :
    \[ \phi_\alpha : 
        \begin{cases}
            \Fc[X] \longrightarrow \K \\ 
            P \longmapsto P( \alpha)
        \end{cases} \]

Distinguons deux cas :
\begin{itemize}
    \item Supposons que $ \alpha$ est transcendant sur $\Fc$ , alors $ \ker \phi_\alpha = \{O_{\Fc[X]}\}$ et $\phi_\alpha$ est injective. 
    \item Supposons que $ \alpha $ est algébrique sur $\Fc$. 
    $\Fc[X]$ est principal et $\ker \phi_\alpha$ est un de ses idéaux donc il existe $P \in \Fc[X]$ tel que $ P(\alpha) = 0 $. 
\end{itemize} 

\begin{definition}[Polynôme Minimal]
    Soit $\K$ une extension d'un corps $\Fc$. Soit $ \alpha \in \K$. Les conditions suivantes sont équivalentes. 
    $P_\alpha \in \Fc$ est appelé \textbf{polynôme minimal} de $ \alpha$ sur $\Lc$ si : 
        \begin{itemize}
            \item[] $P_\alpha$ est l'unique polynôme unitaire de $\Fc[X]$ admettant $ \alpha$ comme racine. 
            \item[$ \iff$] $P_\alpha$ est irréductible dans $\Fc[X]$ et $ \alpha$ est une racine de $P_\alpha$.
            \item[$ \iff$] L'idéal de $\Fc[X]$ engendré par $P_\alpha$ est maximal. 
            \item[$ \iff$] Si $Q \in \Lc[X]$ admet $ \alpha$ comme racine, alors $P_\alpha | Q$. 
        \end{itemize}
    Le degré de $P_\alpha$ est appelé \textbf{degré de $\alpha$}. 
\end{definition}

\begin{proposition}[Polynôme Minimal et noyau]
    Soit $\K$ une extension d'un corps $\Fc$. Soit $ \alpha \in \K$.
    Soit $P_\alpha \in \Fc[X]$ le polynôme minimal de $ \alpha$. 
    On peut montrer que : 
        \[ \ker \phi_\alpha = (P_\alpha) \] 
\end{proposition}

\begin{quote}
    \begin{footnotesize}
        \begin{proof}
            Soit $\K$ une extension d'un corps $\Fc$. Soit $ \alpha \in \K$. Soit le morphisme d'évaluation suivant : 
                \[ \phi_\alpha : 
                    \begin{cases}
                        \Fc[X] \longrightarrow \K \\ 
                        P \longmapsto P(\alpha) 
                    \end{cases} \] 
            Supposons que $ \alpha$ est algébrique. Soit $P$ le polynôme minimal de $\alpha$.
            Montrons que $ \ker \phi_\alpha = (P)$. 
            \begin{itemize}
                \item[$\boxed{\subseteq}$] Soit $ (P) = \{QP \; | \; Q \in \Fc[X]\}$. Soit $R \in (P)$, on a donc : 
                    \[ 
                        R(\alpha) = Q(\alpha) \times P(\alpha) = 0 
                    \] 
                    donc $R \in \ker \phi_\alpha$. On a donc $(P) \subseteq \ker \phi_\alpha$. 
                \item[$\boxed{\supseteq}$] Soit $Q \in \ker \phi_\alpha$, alors $Q(\alpha) = 0$. 
                    \begin{itemize}
                        \item[\textbf{si}] $Q = 0_{\Fc[X]}$ alors $Q \in (P)$ 
                        \item[ \textbf{sinon}] on a : 
                                \[ (Q) := \{RQ \; | \; R \in \Fc[X]\} \] 
                            or $ \forall G \in (Q), G(\alpha) = 0$ donc $(Q) \subseteq (P) \Longrightarrow P | Q$. 
                            
                            d'où $ \ker \phi_\alpha \subseteq (P)$. 
                    \end{itemize}
            \end{itemize}
            Par double inclusion, on a donc $\ker \phi_\alpha = (P)$. 
        \end{proof}
    \end{footnotesize}
\end{quote}

\subsection{Sous-corps engendré}

\begin{definition}[Sous-corps engendré]
    Soient $\K/\Fc$ une extension de corps et $ \alpha \in \K$. On définit le sous-corps engendré par $\Fc$ et $ \alpha$, 
    noté $\Fc( \alpha)$, comme le plus petit sous-corps de $\K$ contenant $ \alpha$ et $\Fc$. 
\end{definition}

On peut définir les sous-corps engendré de manière plus globale de la même façon en prenant une famille 
$( \alpha_1, \dots, \alpha_k) \in \K$. De plus, comme pour les anneaux, $\Fc(\alpha)$ est exactement 
l'intersection de tous les sous-corps de $\K$ contenant $\alpha $ et $\Fc$. 

\begin{proposition}
    Reprenons le morphisme d'évalusation précédant : 
        \[ \phi_\alpha : 
        \begin{cases}
            \Fc[X] \longrightarrow \K \\ 
            P \longmapsto P( \alpha)
        \end{cases} \]
    Supposons que $\alpha \in \K$ est algébrique. On a alors $ \ker \phi_\alpha = (P_\alpha)$.  
    Or, d'après le théorème d'isomorphisme, on a : 
    \begin{multicols}{2}
        \[
            \begin{tikzcd}
                \Fc[X] \arrow[r, "\phi_\alpha"] \arrow[d, two heads, "\pi"'] & \phi_\alpha(\Fc[X]) \\
                \Fc[X] / \ker f \arrow[ur, dashed, "\overline{\phi_\alpha}"']
            \end{tikzcd}
        \]
        Donc :  
            \[ \Fc[X] / (P_\alpha) \simeq \phi_\alpha(\Fc[X]) = \Fc[\alpha] \]  
        Or $P_\alpha$ est irréductible sur $\Fc[X]$ donc $ \Fc[X] / (P_\alpha)$ est un corps. Par isomorphisme, $\Fc[\alpha]$ 
        est aussi un corps. 
    \end{multicols}

    Or $\Fc(\alpha)$ est le plus petit sous-corps de $\K$ contenant $\alpha$ et $\Fc$ et $\Fc[\alpha]$ est 
    un sous-corps de $\K$ contenant $\alpha$ et $\Fc$. Donc c'est le plus petit, d'où $ \Fc[\alpha] =  \Fc(\alpha) $.  

    D'autre part, si $\alpha$ est transcendant, le quotient $\Fc[X] / (P_\alpha)$ n'est pas un corps. 
    Donc $\Fc[\alpha]$ n'en est pas un. D'où $ \Fc[\alpha] \not \subseteq \Fc(\alpha)$. 
\end{proposition}

\begin{theorem}[Cas transcendant]
    Soit $\alpha \in \K$ et $\Fc \subset \K$ un sous-corps de $\K$. On a :
    \begin{itemize}
        \item[] $\alpha$ est transcendant sur $\Fc$
        \item[$\iff$] $\Fc[X] \simeq \Fc[\alpha]$ 
        \item[$\iff$] $ \Fc(\alpha) \not = \Fc[\alpha]$   
    \end{itemize}
\end{theorem}

\begin{theorem}[Cas algébrique]
    Soit $\alpha \in \K$ et $\Fc \subset \K$ un sous-corps de $\K$. On a :
    \[ \boxed{ \alpha \text{ est algébrique sur } \Fc \iff \Fc[X] \simeq \Fc[\alpha] } \] 
\end{theorem}


% ==================================================================================================================================
% Degré d'une extension

\section{Degré d'une extension}

\begin{definition}[Degré]
    Soit $ \K / \Fc$ une extension de corps. On considère $\Fc$ comme un $\K$ espace vectoriel et on définit 
    le degré de l'extension noté $ [\K : \Fc]$ tel que : 
        \[ [\K : \Fc] = \dim_\K (\Fc) \] 
    On dira qu'une extension de corps est \textbf{finie} si son degré est fini. 
\end{definition}

Le degré de l'extension de corps sera donc la dimension de $\Lc$ en tant que $\K$ espace vectoriel. 

\begin{example}
    Si on prend $\Fc = \Q(\sqrt{2})$ et $ \K = \R$ on a donc $ [\Q(\sqrt{2}) : \R] = \dim_\R (\Q(\sqrt{2})) = 2 $ 
    car une base de $\Q(\sqrt{2})$ sur $\R$ est $(1, \sqrt{2})$. 
\end{example}

\begin{lemma}
    Soit $\K / \Fc$ une extension de corps. On a les propriétés suivantes : 
    \begin{itemize}
        \item $ [\K : \Fc] = 1 $ si et seulement si $ \K = \Fc$ 
        \item $ \alpha \in \K$ est de degré 1 sur $\Fc$ ssi $ \alpha \in \Fc$ 
    \end{itemize}
\end{lemma}

\begin{prop}[Degré et éléments algébriques]
    Soit $ \K / \Fc$ une extension de corps. Alors : 
    \begin{itemize}
        \item $ \alpha \in \K$ est algébrique sur $\Fc$ ssi $[\Fc(\alpha) : \Fc]$ est le degré 
        de $\alpha$ sur $\Fc$. 
        \item $\alpha \in \K$ est algébrique est algébrique sur $\Fc$ ssi $[\Fc(\alpha) : \Fc] < \infty$. 
    \end{itemize}
\end{prop}

\begin{theorem}[Multiplicativité du degré]
    Soient $\Fc \subset \K \subset \Lc$ des corps, alors : 
        \[ \boxed{ [\Lc : \Fc] = [\Lc : \K] \times [\K : \Fc] } \] 
    En particulier : $[\Lc : \K] | [\Lc : \Fc]$ et $ [\K : \Fc] | [\Lc : \Fc]$. 
\end{theorem}






