% ==================================================================================================================================
% Introduction

\minitoc  % Affiche la table des matières pour ce chapitre

Grâce au cours de raisonnement et ensemble, nous connaissons bien la notion d'ensemble et les propriétés qu'ils ont. 
Cependant cette structure reste primitive et on commence à en faire le tour. 
Essayons maintenant de doter notre ensemble de quelques propriétés supplémentaires et d'une loi entre ses éléments. 

% ==================================================================================================================================
% Groupes

\section{Groupes}

\begin{definition}[Groupe]
    Un groupe est un couple $(G,*)$ où $G$ est un ensemble et $*$ une application telle que $* : G \times G \rightarrow G$, que l'on appelle loi de composition interne.
    Une groupe satisfait les conditions suivantes :
    \begin{enumerate}
        \item \textbf{$*$ est associative} : $ \forall x, y, z \in G, \; x * (y * z) = (x * y) * z $
        \item \textbf{G est muni du neutre pour $*$} : $ \exists e \in G, \forall x \in G, \; x * e = e * x = x $ 
        \item \textbf{G est inversible} : $ \forall x \in G, \exists y \in G, \; x * y = y * x = e $
    \end{enumerate}
\end{definition}

\begin{remark}
    \begin{itemize}
        \item Le groupe $(G,*)$ est dit \textbf{abélien} ou commutatif ssi $ \forall g,h \in G, g*h = h*g $.
        \item \textbf{L'ordre} de $G$, noté $|G|$, est le cardinal de l'ensmeble si celui-ci est fini. Sinon on dit que $G$ est d'ordre infini.
    \end{itemize}
\end{remark}

\begin{definition}[Groupe Produit]
    Soient $(G_1, *)$ et $(G_2, *)$ deux groupes. On appelle groupe produit le groupe $(G_1 \times G_2, *)$ de $G_1$ et $G_2$ en posant : \[ (x_1, x_2) * (y_1, y_2) = (x_1 * y_1, x_2 * y_2) \]
\end{definition}

% ==================================================================================================================================
% Sous-groupes

\newpage
\section{Sous-groupes}

Lorsque l'on a un groupe, on peut définir d'autre groupes à l'intérieur de celui-ci appelé sous-groupe. 
En général, pour montrer qu'une structure sur un ensemble connu est un groupe on essaye d'abord de montrer que c'est un 
sous-groupe d'un groupe connu. 

\begin{definition}[Sous-groupes]
    Un sous-groupe d'un groupe $G$ est un sous-ensemble $H$ de $G$ sur lequel la multiplication de $G$ induit une structure groupe.
    Il vérifie les propriétés suivantes :
    \begin{itemize}
        \item $ e_G \in H $
        \item $ \forall x, y \in H, \; x * y \in H $
        \item $ \forall x \in H, \; x^{-1} \in H $
    \end{itemize}
\end{definition}

\begin{theorem}[Neutre et inverse dans une sous-groupe]
    Si $H$ est un sous groupe de $G$, le neutre de $H$ est le même que le neutre de $G$ et l'opposé dans $H$ est le même que l'opposé dans $G$.
\end{theorem}

\begin{proposition}[CNS Sous-groupe]
    Soit $(G,*)$ un groupe. Une partie $H$ de $G$ est un sous-groupe de $G$ ssi :
    \begin{itemize}
        \item $H \not = \emptyset $
        \item $ \forall h_1, h_2 \in H, h_1 * h_2^{-1} \in H $
    \end{itemize}
\end{proposition}


\begin{definition}[Sous-groupe distingué]
    Un sous-groupe $H$ de $(G,*)$ est dit \textbf{distingué} ssi 
        \[ \forall g \in G, \forall h \in H, \quad ghg^{-1} \in H \] 
    On le notera $ H \triangleleft G $
\end{definition}

\begin{remark}(Cas particulier des sous-groupes distingués)
    \begin{itemize}
        \item Pour tout groupe G, $\{ e \} \triangleleft G$ et $G  \triangleleft G$
        \item Une sous-groupe d'un groupe abélien est toujours distingué. 
    \end{itemize}
\end{remark}

\begin{definition}[Groupe Simple]
    Un groupe $G \not = \{ e \}$ est appelé groupe simple si les seuls groupes distingués de $G$ sont les sous-groupes triviaux $\{ e \}$ et $G$.
\end{definition}

\begin{proposition}
    L'intersection de sous-groupes distingués (ou non) de $G$ est une sous-groupe distingué (ou non) de $G$.
\end{proposition}

\newpage

% ==================================================================================================================================
% Sous-groupe engendré

\section{Sous-groupe engendré}

Dans certains espaces, on peut définir un groupe à partir d'un unique élément (ou de plusieurs) ce qui donne des 
groupes très intéressant à étudier. 

\begin{definition}[Sous-groupe engendré]
    Soit $(G,*)$ un groupe et $X$ une partie de $G$. Il existe un plus petit sous-groupe de $G$ contenant $X$ appelé sous-groupe engendré par $X$ et noté $\langle X\rangle $.

    Un groupe fini est appelé \textbf{groupe cyclique} s'il existe $g \in G$ tel que $\langle g\rangle =G$.

    Nous appelons l'ordre d'un élément $g \in G$ l'ordre du sous-groupe $\langle g\rangle $, engendré par $g$.
\end{definition}

\begin{corollary}
    Soient $(G,*)$ un groupe et $g \in G$ un élément d'ordre fini $n \in \N$. Alors $n$ ets le plus petit entier strictement positif ayant la propriété $g^n = e$ et de plus : 
        \[ \langle g\rangle  = \{ g, g^2, g^3, \dots, g^n=e \} \] 
\end{corollary}

\begin{proposition}[Sous-groupe des entiers]
    Soit $H$ un sous-groupe de $(\Z,+)$, alors il existe un unique $n \in \Z$ tel que $ H = n\Z$.
\end{proposition}
