% ==================================================================================================================================
% Introduction

\minitoc  % Affiche la table des matières pour ce chapitre

Une fois les corps définis (voir chap précédent), on peut parler d'arithmétique dans les anneaux. 

On va essayer dans ce chapitre de généraliser aux anneaux toutes les propriétés que l'on connaît sur les 
nombres relatifs usuels. L'objectif sera donc ensuite de pouvoir appliquer cette théorie à n'importe quel anneau. 

L’étude des anneaux remonte aux travaux de Richard Dedekind (1831-1916) 
et David Hilbert (1862-1943), qui ont exploré les idéaux pour résoudre des problèmes en théorie des nombres. 
Ces concepts ont été développés pour comprendre la factorisation dans des extensions des entiers, 
notamment dans le cadre du dernier théorème de Fermat. L’approche moderne a été formalisée par Emmy Noether 
au début du XX° siècle.

\vspace{0.3cm}

Dans tout ce chapitre, on se place dans une anneau $A$ unitaire, intègre et commutatif. 

% ==================================================================================================================================
% Divisibilité

\section{Divisibilité}

Revenons aux fondamentaux...

\begin{definition}[Divisibilité]
    Soient $a,b \in A$ on dit que $a$ divise $b$ (noté $a | b$) si il existe $c \in A$ tel que $ac = b$. 

    On note $ \mathcal{D}(a)$ l'ensemble des diviseurs de $a \in A$ dans $A$ : 
        \[ \mathcal{D}(a) = \{d \in A \; \text{ tq } d | a\}  \] 
\end{definition}

\begin{proposition}[Unicité]
    Soient $a,b \in A$. Si $b | a$ et $ b \not  = 0_A$ alors il existe un unique $q \in A$ appelé \textbf{quotient}
    tel que $a = bq$. 
\end{proposition}

\begin{prop}[Divisibilité et Idéaux]
    Soient $a,b \in A$. On a alors l'implication suivante : 
        \[ a | b \Longrightarrow (b) \subseteq (a) \] 
\end{prop}

\begin{definition}[Éléménts associés]
    Soient $a,b \in A$ on dit que $a$ et $b$ sont \textbf{associés} si : 
        \[ a | b \quad \text{et} \quad b | a \] 
    L'association pour la divisibilité est une relation d'équivalence noté $a \mathcal{R} b$. 
\end{definition}

\begin{proposition}
    Soient $a,b \in A$, on a alors : 
        \[ a \mathcal{R} b \iff \exists u \in \mathcal{U}(A), a = ub \iff (a) = (b) \] 
\end{proposition}


% ==================================================================================================================================
% PGCD, PPCM et éléments irréductibles 

\section{PGCD, PPCM et éléments irréductibles}

\begin{definition}[PGCD]
    Soit $(a_i)_{i \in I} \in A$ une famille dénombrable d'éléments de $A$. 
    On définit le PGCD de $(a_i)$ dans $A$ comme le plus grand diviseur commun de tous les $a_i, i \in I$. 
    Plus formellement, s'il existe : 
        \[ \boxed{ d = pgcd((a_i)_{i \in I}) \iff 
        \begin{cases}
            \forall i \in I, d | a_i \\ 
            \forall \delta \in A, \forall i \in I, \delta | a_i \Longrightarrow d | \delta 
        \end{cases} } \] 
    \textbf{Attention :} dans notre cadre le PGCD d'un famille d'éléments de $A$ n'existe pas forcément. 
\end{definition}

\begin{proposition}
    Soit $(a_i)_{i \in I} \in A$ une famille dénombrable d'éléments de $A$. 
    Le PGCD de la famille $(a_i)_{i \in I}$ est unique à un inversible près. 
    Autrement dit,
        \[ \forall d,d' \in A, 
        \begin{cases}
            d = pgcd((a_i)_{i \in I}) \\ 
            d' = pgcd((a_i)_{i \in I}) \\ 
        \end{cases}
        \Longrightarrow d \mathcal{R} d' \] 
\end{proposition}

\begin{proposition}
    Soient $a,b,d \in A$. On a l'implication suivante : 
        \[ (a) + (b) = (d) \Longrightarrow d = pgcd(a,b) \] 
    Ainsi, si on trouve un générateur de la somme de deux idéaux, alors on trouve un PGCD à un inversible près. 
\end{proposition}

\begin{definition}[Éléments premiers entre eux]
    Soit $(a_i)_{i \in I} \in A$ une famille dénombrable d'éléments de $A$. 
    On dit que tous les $a_i$ sont premiers entre eux si 
        \[ \forall i \in I, \forall j \in I, i \not  = j, \quad pgcd(a_i, a_j) \text{ existe et } pgcd(a_i, a_j) \in \mathcal{U}(A) \] 
\end{definition}

\begin{definition}[PPCM]
    Soit $(a_i)_{i \in I} \in A$ une famille dénombrable d'éléments de $A$. 
    On dit que $(a_i)_{i \in I}$ admet un plus petit commun multiple dans $A$ s'il existe un élément dans $A$ 
    multiple de tous les $a_i$ et étant le plus petit multiple. 
    Plus formellement : 
        \[ \boxed{ m = ppcm((a_i)_{i \in I}) \in A \iff 
        \begin{cases}
            \forall i \in I, a_i | m \\ 
            \forall \mu \in A, \forall i \in I, a_i | \mu  \Longrightarrow m | \mu 
        \end{cases} } \] 
    Tout comme le PGCD, le PPCM est définit à un inversible près. 
\end{definition}

\begin{proposition}[Caractérisation des PPCM]
    On peut caractériser le PPCM de deux éléments à partir des idéaux de ces éléments. 
    Ainsi, soient $a,b, m \in A$. On a alors : 
        \[ \boxed{ m = ppcm(a,b) \iff (m) = (a) \cap (b) } \] 
\end{proposition}

\newpage

\begin{definition}[Élémént irréductible]
    Soit $a \in A$. On dit que $p$ est irréductible dans $A$ si il n'existe aucune décomposition de $p$ dans 
    $A$ en éléments non inversibles. Plus formellement : 
        \[ p \in A \text{ est irréductible } \iff 
        \begin{cases}
            p \not = 0_A \text{ et } p \not \in \mathcal{U}(A) \\ 
            \forall d_1, d_2 \in A,  p = d_1 \times d_2 \Longrightarrow d_1 \in \mathcal{U}(A) \text{ \textbf{ou }} d_2 \in \mathcal{U}(A) 
        \end{cases} \] 
\end{definition}


\begin{example}
    Les irréductibles de $(\Z, +, \times)$ sont exactement les nombres premiers de $\Z$. 
\end{example}


% ==================================================================================================================================
% Divisibilité dans les anneaux principaux 

\section{Divisibilité dans les anneaux principaux}

Depuis le début de ce chapitre nous nous plaçons dans un cadre relativement général. 
Ce cadre, quoique très intéressant, ne nous permet pas d'avoir de "bonnes" propriétés pour la divisibilité. 
Ainsi, nous allons ici nous placer dans des anneaux principaux. 

\begin{remark}[Rappel]
    Un anneau $A$ est dit principal si tout idéal peut être engendré par un seul élément de $A$. 
\end{remark}

\begin{theorem}[Existence PGCD et PPCM]
    Soit $A$ un anneau \textbf{principal}. Soit $(a_i)_{i \in I}$ une famille d'éléments de $A$. 
    Alors le PGCD et le PPCM de $(a_i)_{i \in I}$ existe et de plus : 
        \begin{itemize}
            \item $ pgcd((a_i)_{i \in I}) = d \text{ tel que } d \in A \text{ et } (d) = (\{a_i \; | \; i \in I \}) $ 
            \item $ ppcm((a_i)_{i \in I}) = m \text{ tel que } m \in A \text{ et } (m) = \bigcap_{i \in I} (a_i) $
        \end{itemize}
\end{theorem}

\begin{theorem}[Bézout]
    Soit $A$ un anneau principal et $a,b \in A$. On a alors : 
        \[ \boxed{ pgcd(a,b) = d \in A \iff \exists u, v \in A, \; au + bv = d } \] 
    En particulier : 
        \[ pgcd(a,b) = 1 \iff \exists u, v \in A, \; au + bv = 1 \] 
\end{theorem}

\begin{corollary}[Bézout]
    Soit $A$ un anneau principal et $a,b, \alpha, \beta \in A$. Alors : 
        \[ a = d \alpha \text{ et } b = d \beta \text{ et } pgcd(\alpha, \beta) = 1 \Longrightarrow pgcd(a,b) = d \] 
\end{corollary}

\begin{theorem}[Gauss]
    Soit $A$ un anneau principal et $a,b,c \in A$. On a :
    \begin{itemize}
        \item $ pgcd(a,b) = 1 \text{ et } a | bc \Longrightarrow a | c $ 
        \item $p$ irréductible dans $A$ et $ p | ab \Longrightarrow p | b $ 
    \end{itemize}
\end{theorem}

\begin{definition}[Anneau Euclidien]
    On dit que $A$ est un anneau euclidien s'il est possible d'y définir une division euclidienne. 
\end{definition}

\begin{theorem}[Anneau euclidien, conséquences]
    Tout anneau euclidien est principal. 
\end{theorem}


% ==================================================================================================================================
% Anneaux Factoriels

\section{Anneaux Factoriels}

\begin{definition}[Anneau Factoriel]
    Un anneau $A$ est dit factoriel s'il est intègre et si tout élément se factorise en éléments inversibles. 
\end{definition}

\begin{theorem}[Factorialité des anneaux principaux]
    Tout anneau principal est factoriel. 
\end{theorem}

