% ==================================================================================================================================
% Introduction

\minitoc  % Affiche la table des matières pour ce chapitre

En mathématiques, dès que l'on découvre une nouvelle structure particulière, on essaye de définir des applications dessus 
pour essayer de voir si certaines structures se ressemblent ou pas. C'est ce que l'on va faire avec les groupes. 

% ==================================================================================================================================
% Morphisme, Image et Noyaux 

\section{Morphismes, Image et Noyaux}

\subsection{Définitions Générales et premières propriétés}

\begin{definition}[Morphisme de groupes]
    Soient $(G,\ast )$ et $(K,\bullet )$ deux groupes. 
    Un morphisme de groupes $\phi$ entre $G$ et $K$ est une application 
        \[ \phi : 
        \begin{cases}
            G \longrightarrow K \\ 
            g \longmapsto \phi(g)
        \end{cases} \] 
        telle que : 
        \[ \forall (g_1,g_2) \in G \times G, \quad \phi (g_1 \ast g_2) = \phi(g_1) \bullet \phi(g_2) \]
    \emph{"Un morphisme de groupes est une application qui respecte la structure des groupes."}
\end{definition}

\newpage 

\begin{definition}[Noyau et Image]
    Soit $\phi$ un morphime entre deux groupes $(G,*)$ et $(K,\bullet)$. 
    \begin{itemize}
        \item Le \textbf{noyau} de $\phi$ est l'ensemble des éléments de $G$ envoyés par $\phi$ sur $e_{K}$ 
            \[ \text{i.e } \ker (\phi) = \{ g \in G, \quad  \phi(g) = e_{K} \} \] 
        \item \textbf{L'image} de $\phi$ est l'ensemble des $\phi(g), g \in G$ dans $K$. 
            \[ \text{i.e }  \phi(G) = \{ h \in K, \quad \exists g \in G \text{ tq } \phi(g) = h \} \] 
    \end{itemize}
\end{definition}

\begin{definition}[Morphismes et bijection]
    \begin{itemize}
        \item Un \textbf{isomorphime} de $G$ dans $K$ est un morphisme de groupes \textbf{bijectif}. 
        \item Un \textbf{endomorphisme} de $G$ est un morphisme de $G$ dans lui-même. 
        \item Un \textbf{automorphisme} de $G$ est un morphisme bijectif de $G$ dans lui-même. 
    \end{itemize}
\end{definition}

\begin{prop}[Morphismes]
    Soit $\phi : (G,*) \longrightarrow (K,+)$ un morphisme de groupes. 
    \begin{itemize}
        \item $\phi (e_g) = e_{K} $
        \item $\phi(g^{-1}) = \phi(g)^{-1}$
        \item $\phi(g^n) = \phi(g)^n$
    \end{itemize}
\end{prop}

\subsection{Image et Noyaux}

\begin{lemma}[Image et Image réciproque d'un morphisme]
    Soit $\phi : (G,*) \longrightarrow (K,\bullet)$ un morphisme de groupes. 
    \begin{itemize}
        \item L'image d'un sous-groupe $H$ de $G$ est un sous-groupe de $G$. 
        \item Si $H \vartriangleleft G$ alors $\phi(H) \vartriangleleft \phi(G)$ \\ 
        En particulier, si $\phi$ est surjectif, alors $\phi(H) \vartriangleleft K$ 
        \item L'image réciproque d'un sous-groupe de $K$ est un sous-groupe de $G$.
        \item Si $H' \vartriangleleft G'$ alors $\phi^{-1}(H') \vartriangleleft G$ 
    \end{itemize}
    De plus, l'image de $\phi$ est un sous-groupe de $K$ et $\ker(\phi)$ est un sous-groupe distingué de G.
\end{lemma}

\begin{corollary}[Morphisme et sous-groupe engendré]
    Soit $\phi : (G,*) \longrightarrow (K,+)$ un morphisme de groupes et $X \subset G$, alors 
        \[ \boxed{ \phi(\langle  X \rangle ) =  \langle \phi(X) \rangle } \] 
    \emph{"L'image d'un générateur est le générateur de l'image."}
\end{corollary}

\begin{prop}[Stabilité]
    Les morphismes de groupes sont stables par composition et réciproque (dans le cas d'un isomorphisme).
\end{prop}

\begin{lemma}
    Un morphisme de groupes $\phi : G \longrightarrow K$ est injectif si et seulement si $\ker(\phi) = \{e_{G}\}$.
\end{lemma}

\subsection{Le cas des entiers}

\begin{proposition}
    Il existe un unique groupe cyclique d'ordre $n \in \N$  (à isomorphisme près). 
\end{proposition}

\begin{theorem}[Ordre Fini]
    Soit $G$ un groupe et $g \in G$. 
    \begin{enumerate}
        \item $g$ est d'ordre \textbf{infini} ssi $\langle g \rangle \sim (\Z,+)$, dans ce cas :
            \[ \langle g \rangle = \{\dots, g^{-3}, g^{-2}, g^{-1}, e, g, g^2, g^3, \dots \} \] 
        \item $g$ est d'ordre \textbf{fini} $n \in \N$ si et seulement si l'ensemble $\langle g \rangle = \{e, g, g^2, \dots, g^{n-1}\}$ 
        est bien défini et si $g^n = e$.
    \end{enumerate}
\end{theorem}

% ==================================================================================================================================
% Automorphisme 

\section{Automorphismes}

Pour rappel, un automorphisme d'un groupe G est un morphisme de groupe bijectif de G dans lui-même.

\begin{definition}[Centre d'une groupe]
    Le centre d'un groupe $G$ est l'ensemble $Z(G)$ des éléments de $G$ qui commutent avec tous les éléments de $G$.
\end{definition}

\begin{theorem}[Automorphismes]
    L'ensemble $(Aut(G),\circ)$ des automorphismes d'un groupe $G$ muni de la composition, est un groupe.
    \begin{itemize}
        \item C'est un sous-groupe de l'ensemble des permutations de $G$. 
        \item Si $G$ est d'ordre fini $n$, alors $Aut(G)$ est un groupe d'ordre au plus $(n-1)!$.
    \end{itemize}
\end{theorem}

\begin{definition}[Automorphisme intérieur]
    Soit $(G,\ast)$ un groupe noté multiplicativement, on définit l'automorphisme intérieur associé à $g \in G$ le morphime
    de groupes tel que :
    \[  Int_g
        \begin{cases}
            G \longrightarrow G \\
            x \longmapsto gxg^{-1}
        \end{cases}
    \]
    On définit $Int(G)$ comme l'ensemble des automorphismes intérieurs de $G$.
\end{definition}

\begin{remark}
    \begin{itemize}
        \item Tout automorphisme intérieur est un automorphisme de groupes. 
        \item Deux éléments d'un groupes dont l'un est image de l'autre par automorphisme intérieur sont dits conjugués. 
        \item On remarquera qu'un sous-groupe H de G est distingué dans G ssi H est stable par tous les automorphismes intérieurs de G. 
    \end{itemize}
\end{remark}
