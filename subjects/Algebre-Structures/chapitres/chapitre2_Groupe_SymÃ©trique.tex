% ==================================================================================================================================
% Introduction

\minitoc  % Affiche la table des matières pour ce chapitre

Maintenant que nous avons vu en détail la notion de groupe, sous-groupe et groupe engendré, 
attardons nous sur un des groupes les plus connus, le groupe symétrique. 
Ce groupe est un exemple presque canonique en algèbre des structures, il faut donc bien le connaître. 

% ==================================================================================================================================
% Groupe Symétrique

\section{Le groupe symétrique}

\subsection{Définitions}

\begin{definition}[Groupe Symétrique]
	On appelle \textbf{groupe symétrique} le groupe des bijections de $\{1, \dots ,n\}$ dans lui-même doté de la loi de composition, noté $(\mathcal{S}_n,\circ)$.
	Un élément de $\mathcal{S}_n$ quelconque est appelé une \textbf{permutation}.
\end{definition} 

\begin{prop}[Ordre et Bijection]
	Soit $n \in \N^*$ alors :
	\begin{enumerate}
		\item $ \forall n \in \N^*, |\mathcal{S}_n| = n ! $
		\item si $ |X| = n $ alors $ \mathcal{S}(X) \cong \mathcal{S}_n $
	\end{enumerate}
\end{prop}

\begin{quote}
	\footnotesize
	\begin{proof}
		Soit $\mathcal{S}_n$ l'ensemble des bijections de $\llbracket 1,n \rrbracket, n \geq 3$ dans lui-même.
		Initilisation triviale. \\
		Supposons que HR : $ \forall n \geq 3, | \sigma_n | = n ! $

		Soient $A_{n+1}$ et $B_{n+1}$ deux ensembles à $n+1$ éléments deux à deux distincts tels que :
			\[ A_{n+1} = \{a_1, \dots, a_{n+1} \} \]
		Soit $\sigma$ une bijection telle que $ \sigma : A_{n+1} \Longrightarrow B_{n+1}$.
		Combinatoirement, on a $n+1$ choix pour l'image de $a_{n+1}$ par $\sigma$.
			\[ i.e \quad \sigma(a_{n+1}) \in B = \{b_1, \dots, b_{n+1} \} \]
		Posons $ \sigma(a_{n+1}) = b_{n+1}$. Notons que $a_{n+1}$ n'est pas forcément égal à $b_{n+1}$.
		Pour respecter l'injectivité de $\sigma$, on n'a que $n !$ possibilités pour le choix de $\sigma(a_n)$ par hypothèse de récurrence.
		D'où, ici :
			\[ |\mathcal{S}_{n+1}| = |\mathcal{S}_n| \times (n+1) = n! \times (n+1) = (n+1)! \]
	\end{proof}
	\begin{proof}
		Soit $X$ un ensemble de cardinal $n \in \N^*$ et $\mathcal{S}(X)$ l'ensemble des bijections de $X$ sur lui-même où :
			\[ X_n = \{x_1, \dots, x_n \} \]
		Soit $\mathcal{S}_n$ l'ensemble des bijections de $\{1, \dots, n\}$ dans lui-même, alors :
			\[ i :  
				\begin{cases}
					\{1,\dots,n\} \longrightarrow X \\
					i \longmapsto x_i 
				\end{cases}  \text{ est une bijection} \]
		Soit $\sigma$ une bijection quelconque de $\mathcal{S}_n$ et $\tau$ une bijection quelconque de $ X \longrightarrow X$ 
		$$
		\begin{tikzcd}
			X \arrow[r, "\tau"]
				& X \arrow[d, "i^{-1}"] \\
			\{1,\dots,n\} \arrow[r, "\sigma"] \arrow[u, "i"] 
				& \{1,\dots,n\}
		\end{tikzcd}
		$$
		Posons :
				\[ \Phi : 
					\begin{cases}
						\mathcal{S}_n \longrightarrow \mathcal{S}(X) \\
						\tau \longmapsto i^{-1} \circ \sigma \circ i 
					\end{cases} \]
		Alors $\Phi$ est une bijection de $\mathcal{S}(X)$ dans $\mathcal{S}_n$ et c'est aussi un morphisme de groupe car :
					\[ \Phi(\tau_1 \circ \tau_2) = i^{-1} \circ (\tau_1 \circ \tau_2) \circ i = i^{-1} \tau_1 i i^{-1} \tau_2 i = \Phi(\tau_1) \circ \Phi(\tau_2) \]
	\end{proof}
	\normalsize
\end{quote}

\subsection{Structure du groupe symétrique}

\begin{theorem}[Centre d'un groupe]
	On désigne $Z$ comme étant le centre d'un groupe. Ici, $\forall n \geq 3, Z(\mathcal{S}_n) = \{e\}$.
\end{theorem}

\begin{quote}
	\footnotesize
	\begin{proof}(Par l'absurde)
		Soit $n \geq 3$ et $\sigma \in \mathcal{S}_n$ telle que $ \sigma \not = e $. \\ 
		Soit $ x \in \{1,\dots,n\}, \sigma(x) \not = x$. Posons $y = \sigma(x)$ et $z = \sigma(y)$ et soit $\tau = (x \; y) \in \mathcal{S}_n$.
		Alors :
		\[
			\begin{cases}
				\sigma \circ \tau (x) = \tau(y) = x \\
				\tau \circ \sigma(x) = \sigma(y) = z 
			\end{cases} \]
		Donc $\exists x \in \{1,\dots,n\}$ tq $ \sigma \circ \tau \not = \tau \circ \sigma $ donc $ \sigma \not \in Z(\mathcal{S}_n)$.
		Donc nécessairement, $Z(\mathcal{S}_n) = \{e\}$.
	\end{proof}
	\normalsize
\end{quote}

\begin{theorem}[Cayley]
	Tout groupe fini $G$ d'ordre $n$ est isomorphe à un sous-groupe de $\mathcal{S}_n$.	
\end{theorem}

\subsection{Propriétés d'une permutation}

\begin{definition}[Points fixes, Support]
	Soient $n \in \N^*$ et $\sigma \in \mathcal{S}_n$.
	\begin{itemize}
		\item Les éléments $i \leq n$ tels que $\sigma(i) = i$ sont appelés \textbf{points fixes} de $\sigma$ 
		\item Une partie $A$ de $\{1,\dots,n\}$ est dite stable par $\sigma$ si $\sigma(A) \subseteq A$
		\item On définit le \textbf{support} de $\sigma$ l'ensemble : 
			\[ \boxed{ \text{ Supp} (\sigma) := \{k \in \{1,\dots,n\}, \sigma(k) \not = k \} }\]
	\end{itemize}
\end{definition}

\newpage 

\begin{prop}[Supports, Inclusion et Produit]
	Soient $ n \in \N$ et $\sigma,\tau \in \mathcal{S}_n$ alors :
	\begin{enumerate}
		\item $ \text{ Supp} (\tau \circ \sigma) \subset \text{ Supp}(\sigma) \cup \text{ Supp}(\tau)$
		\item si $\text{ Supp}(\sigma) \cap \text{ Supp}(\tau) = \emptyset$ alors :
				\begin{enumerate}
						\item $\text{ Supp}(\pi \circ \tau) = \text{ Supp}(\sigma) \cup \text{ Supp}(\tau)$ 
						\item $
							\begin{cases}
			   					\sigma \circ \tau = \tau \circ \sigma \\
								\sigma \circ \tau = e \Longrightarrow \sigma = \tau = e
							\end{cases} $
				\end{enumerate}
	\end{enumerate}
\end{prop}

\begin{quote}
	\footnotesize
	\begin{proof}
		Soient $n \in \N$ et $\sigma, \tau \in \mathcal{S}_n$.
		\begin{enumerate}
			\item Soit $k \in \{1,\dots,n\}$ tel que $ k \in \text{ Supp}(\sigma\circ\tau)$.
			Par définition, $\sigma \circ \tau (k) \not = k$. Distinguons deux cas :
				\begin{enumerate}
					\item $\tau(k) = k$ alors $\sigma (\tau(k)) = \sigma (k) \not = k$ donc $k \in \text{ Supp}(\sigma)$
					\item $\sigma(k) = k$ alors puisque $\sigma \circ \tau (k) \not = k$, nécessairement, $\tau (k) \not = k$ donc $k \in \text{ Supp}(\tau)$
				\end{enumerate}
				D'où $ k \in \text{ Supp}(\tau) \cup \text{ Supp}(\sigma)$
			\item Supposons que $\text{ Supp}(\tau) \cap \text{ Supp}(\sigma) = \emptyset$ \\ 
			\textbf{L'union des supports est égale au support du produit :}
			\begin{enumerate}
					\item $\boxed{\subseteq}$ soit $k \in \text{ Supp}(\sigma \tau)$ alors $\sigma \circ \tau (k) \not = k$
							d'où $k \in \text{ Supp}(\sigma) \cup \text{ Supp}(\tau)$
					\item $\boxed{\supseteq}$ soit $k \in \text{ Supp}(\sigma) \cup \text{ Supp}(\tau)$ alors $\sigma (k) \not = k$ ou $ \tau(k) \not = k$ 
							si $ \tau(k) \not = k $ alors $ \sigma(\tau(k)) \not = k$
							D'où $ k \in \text{ Supp}(\sigma \tau)$
			\end{enumerate}
			\item Soient deux permutations $\sigma$ et $\tau$ à supports disjoints et $k \in \{1,\dots,n\}$.
			Montrons que $\sigma$ et $\tau$ commutent. \\
			Si $\sigma(k)=k$ alors $\tau(\sigma(k))= \tau(k)$ et $\sigma(\tau(k)) = \tau(k)$
			D'ou $ \sigma \circ \tau = \tau \circ \sigma$. (On a le même raisonnement pour le second cas) \\
			Si $\sigma \circ \tau = e$, en raisonnant par l'absurde sur les supports, on a que $\sigma = \tau = e$.
		\end{enumerate}
	\end{proof}
	\normalsize
\end{quote}


\begin{definition}[Cycle]
	Soit $p \in \N$ tel que $ 2 \leq p \leq n $ et soit $s \in \mathcal{S}_n$.
    On dit que $s$ est un $p$-cycle de \textbf{support} $\{a_1, a_2, \dots , a_p \}$ et de \textbf{longueur} $p$ ssi $ \exists a_1, a_2, \dots , a_p \in \{1, 2, \dots , n\}$ deux à deux différents tels que :
    
    \[
        \left \{
            \begin{array}{cccc }
                s(a_i) & = & a_{i+1} & \forall i < p \\
                s(a_p) & = & a_1 & \\
                s(k) & = & k & \forall k \not \in \{a_1, \dots , a_p\}
            \end{array}
        \right.
    \]
	Si $s$ est un $p$-cycle alors on a: $ { s^p = Id} $.
	Un cycle de longueur $2$ est appelé \textbf{transposition}.
\end{definition}

\begin{theorem}[Décomposition de permutations]
		Toute permutation $\sigma \in \mathcal{S}_n$ peut s'écrire comme produit (i.e composition) 
		de cycles $c_1,\dots,c_l$ à supports deux à deux disjoints. 
		Cette décomposition est unique à l'ordre près.
\end{theorem}

\begin{quote}
\footnotesize
\begin{proof}
	Soit $\sigma \in \mathcal{S}_n$. Soient $a,b \in \text{ Supp}(\sigma)$ tels que :
		\[ \mathcal{O}(a) = \{\sigma^k(a); k \in \N\} \quad \mathcal{O}(b) = \{\sigma^k(b); k \in \N\} \] 
	\begin{lemma}
		Montrons que $\mathcal{O}(a) = \mathcal{O}(b)$ ou $\mathcal{O}(a) \cap \mathcal{O}(b) = \emptyset $

		Supposons que $\mathcal{O}(a) \cap \mathcal{O}(b) \not = \emptyset $.
		Soit $ x \in \mathcal{O}(a) \cap \mathcal{O}(b)$ alors $\exists k,l \in \N$ tels que :
			\[ \sigma^k(a) = x \; \text{et} \; \sigma^l(b) = x \Longrightarrow \sigma^k(a) =\sigma^l(b) \]
		or $\mathcal{O}(a)$ est un ensemble fini donc il existe $n \in \N$ tel que  $\sigma^{k-l} = \sigma ^n$ 
		d'où $ b \in \mathcal{O}(a)$ donc $\mathcal{O}(b) \subseteq \mathcal{O}(a)$
		Symétriquement, on obtient facilement que $\mathcal{O}(a) \subseteq \mathcal{O}(b)$
	\end{lemma}
	Donc il existe $\mathcal{F} = (a_i, \dots, a_l), l \leq n$ tel que :
		\[ \text{Supp}(\sigma) = \bigcup_{i \leq l} \mathcal{O}(a_i) \] 
	Soit $c_i \in \mathcal{S}_n, i \leq l $ un cycle de longueur $ p_i = | \mathcal{O}(a_i) | $ tel que :
		\[ c_i = (a_i \dots) \] 
\end{proof}
\normalsize
\end{quote}

\begin{quote}
    \begin{footnotesize}
        D'après le lemme précédent $\forall i,j \leq l $ et $ i \not = j $ :
		\[ \boxed{ \text{ Supp}(c_i) \cap \text{Supp}(c_j) = \emptyset } \] 
        {Posons :} $\boxed{ \tau = c_1 \circ \dots \circ c_i \circ \dots \circ c_l }$ \\
        {Montrons que $\sigma = \tau$ :} Soit $k \in \llbracket 1, n \rrbracket $, distinguons deux cas :
        \begin{itemize}
            \item si $k \not \in \text{Supp}(\sigma)$ alors $\sigma(k) = k$ et par construction $\mathcal{O}(k) = \{k\}$ d'où $\tau(k) = k$ 
            \item si $k \in \text{Supp}(\sigma)$ alors $ \exists! i \leq l$ tel que $ \tau(k) = c_i(k) = \sigma(k)$
        \end{itemize}
        On a donc montré que $\forall k \leq n, \tau(k) = \sigma(k)$. Donc $\sigma = \tau$ 
    \end{footnotesize}
\end{quote}

\begin{prop}[Ordre d'une permutation]
		Soit $ \sigma \in \mathcal{S}_n$ et $c_1 \circ \dots \circ c_m, m \leq n $ sa décomposition en produit de cycles à supports deux à deux disjoints.
		L'ordre de $\sigma$ est égale au PPCM des longueurs des $c_i$.
\end{prop}

\begin{quote}
\footnotesize 
\begin{proof}[Ordre d'une permutation]
		Soit $\sigma$ une permutation de $\{1,\dots,n\}$, d'après la propriété précédente:
		\[ \sigma = c_1 \circ \dots \circ c_l \]
		Où les $c_i$ sont des cycles à supports deux à deux disjoints. Soient $k \ in \N, x \in \llbracket 1, n \rrbracket$, alors :
			\[ \sigma^k (x) = x \Leftrightarrow c_1^k \circ \dots \circ c_l^k(x) = x \]
		Donc $ \exists! i \leq l $ tel que :
		\begin{align*}
			\sigma^k (x) = x & \Longleftrightarrow c_i^k(x) = x \\
						& \Longleftrightarrow k \in \text{ord}(c_i) \times \N \\
						& \Longleftrightarrow k \in \bigcap_{i \in \llbracket 1, n \rrbracket} \text{ord}(c_i) \times \N 
		\end{align*}
		or $ \min \bigl\{ \bigcap_{i \leq l} \text{ord}(c_i) \times \N \bigr\} = PPCM(\text{ord}_{i \leq l}(c_i))$.
		D'où le résultat.
\end{proof}
\normalsize
\end{quote}

\subsection{Générateurs du groupe symétrique}

\begin{definition}[Conjugaison]
		Soient $\sigma, \tau$ deux permutations. On dit qu'elles sont conjuguées si :
		\[ \boxed{ \exists \omega \in \mathcal{S}_n, \quad \sigma = \omega \circ \tau \circ \omega ^{-1} } \]
\end{definition}

\begin{prop}[Conjugaison]
		Deux permutations sont conjuguées ssi elles sont de même type. \\ 
		En particulier, $\forall \sigma \in \mathcal{S}_n, \forall (i_1 \dots i_l)$ un l-cycle, on a :
			\[ \sigma \circ (i_1 \dots i_l) \sigma^{-1} = (\sigma(i_1) \dots \sigma(i_l)) \]
\end{prop}

\begin{theorem}[Générateurs]
		Le groupe symétrique est engendré par les transpositions.
\end{theorem}

\begin{quote}
\footnotesize
\begin{proof}
	Soit $ \sigma \in \mathcal{S}(n)$ alors $\exists c_1, \dots, c_l \in \mathcal{S}_n$ des cycles à supports disjoints tels que : 
		\[ \sigma = c_1 \circ \dots \circ c_i \circ \dots \circ c_l \] 
	où $c_i = (a_1 \dots a_p)$. C'est un cycle donc on peut le décomposer en produit de transpositions.
		\[ \text{i.e } c_i = (a_1 \; a_2) \circ \dots \circ (a_{p-1} \; a_p) \] 
	Soit $\tau \in \mathcal{S}_n$ le produit de la décomposition de chaque $c_i$ en produit de transpositions. 
	Soit $x \in \llbracket 1,n \rrbracket$. Distinguons deux cas :
	\begin{itemize}
		\item si $x \not \in \text{Supp}(\sigma)$ alors $\sigma(x) = x$ et par construction, $\tau(x) = x$ aussi.
		\item si $x \in \text{Supp}(\sigma)$ alors $ \exists! i \leq l, x \in \text{Supp}(c_i)$ où 
			$c_i = (a_1 \; a_2) \circ \dots \circ \underset{\omega \in \mathcal{S}_n}{(x \; \sigma(x))} \circ (\sigma(x) \dots) \circ \dots \circ (a_{p-1} \; a_p) $
			or $\omega$ est la seule permutation cd $c_i$ telle que $x \in \text{Supp}(\omega)$ donc $c_i(x) = \sigma(x) = \tau(x)$
	\end{itemize}
	Donc $\sigma = \tau$. D'où $\mathcal{S}_n$ est engendré par les transpositions.
\end{proof}
\normalsize
\end{quote}

% ==================================================================================================================================
% Signature et groupe alterné

\section{Signature et groupe alterné}
\subsection{Signature d'une permutation}

\begin{definition}[Signature]
		Soit $\sigma \in \mathcal{S}_n$, on appelle signature de $\sigma$ l'entier $\varepsilon (\sigma)$ telle que:
			\[ \varepsilon(\sigma) = \prod_{1\leq i \le j \leq n} \frac{\sigma(i) - \sigma(j)}{i-j} \]
\end{definition}

\begin{prop}[La signature est un morphisme de groupes]
		$\varepsilon : \mathcal{S}_n \longrightarrow (\Q^*, \times)$ est un morphisme de groupes.
\end{prop}

\begin{prop}[Calcul de la signature]
	Soit $z \in \mathcal{S}_n$ un cycle de longueur $p \geq 2$ alors:
		\[ \varepsilon(z) = {(-1)}^{p-1} \]
	(Ce résultat se montre en décomposant $z$ en produit de transpositions, puis en appliquant $\varepsilon$ en tant que morphisme de groupes.)
\end{prop}

\subsection{Groupe Alterné}

\begin{definition}[Noyau, Groupe alterné]
	On appelle groupe alterné le noyau de $\varepsilon$. C'est un sous-groupe distingué de $\mathcal{S}_n$.
	On le note $\mathcal{A}_n$.
\end{definition}

\begin{prop}[Groupe Alterné]
	Soit $n \geq 3$
	\begin{itemize}
		\item $\mathcal{S}_n$ est engendré par les permutations $(1,i), 2 \geq i \geq n$ 
		\item $\mathcal{A}_n$ est engendré par les cycles $(1,i,j), 2 \geq i \not = i \geq n $
	\end{itemize}
	En partiulier, $\mathcal{A}_n$ est engendré par les 3-cycles de $\mathcal{S}_n$.
\end{prop}
