

\justify 

\setlength{\parindent}{0pt}


% ==================================================================================================================================
% Introduction


Nous savons que l'ensemble des matrices carrées à coefficients dans un corps $\K$ (réel ou complexe) est 
muni d'une structure d'espace vectoriel. 
Ayant déjà étudié en détail les polynômes, on pourrait se demander si il est possible d'étudier des polynômes de matrices 
et par extension des polynômes d'endomorphismes. 
Pourrait-on, le cas échéant, en fonction des propriétés d'un tel polynôme, en déduire des propriétés sur des endomorphismes/matrices 
tels que des critères de diagonalisation ou de trigonalisation ?
C'est ce que nous allons étudier dans ce chapitre. 

\vspace{0.3cm}

Dans ce chapitre, on se place dans un $\K$-espace vectoriel $E$ où $\K$ désigne un corps. 
On notera $f$ ou $g$ des endomorphismes de $E$, $\lambda$ ou $\alpha$ des sclaires et $\mathcal{B}$ ou $\mathcal{B}'$ des bases de E. 


% ==================================================================================================================================
% Définition et propriétés


\section{Définition et premières propriétés}

Commençons par définir les principaux objects utilisés dans ce chapitre. 

\subsection{Evaluation d'un polynôme et morphisme}

\begin{definition}[Polynôme d'endomorphisme]
    Soit $P = \sum_{i=0}^{n} a_i X^i$ un polynôme et $f$ un endomorphisme, on peut évaluer P en f de la façon suivante :
        \[ P(f) = \sum_{i=0}^{n} a_i f^i \]
    Par convention, $f^0 = \text{Id}$. 
\end{definition}

\begin{definition}[Polynôme de matrice]
    Soit $P = \sum_{i=0}^{n} a_i X^i$ un polynôme et $A \in \mathcal{M}_n(\K)$ une matrice carrée. 
    On peut évaluer $P$ en $A$ de la façon suivante :
        \[P(A) = \sum_{i=0}^{n} a_i A^i\]
    De même que précédement, on pose $A^0 = \text{Id}$.
\end{definition}


\begin{proposition}
    On peut "résumer" ces deux définitions en posant une application qui, à un polynôme, lui associe ce même polynôme évalué 
    en un endomorphisme prédéterminé. 

    Plus formellement, $ \forall f \in \mathcal{L}(E)$, l'application 
        \[ \Phi_f : 
            \begin{cases}
                \K[X] \longrightarrow \mathcal{L}(E) \\ 
                P \longmapsto P(f)
            \end{cases}
        \]
    est un morphisme d'algèbre. Autrement dit, c'est une application entre deux algèbres qui respecte leur structure. 
    
    Elle vérifie donc $ \forall \alpha \in \K, \forall P,Q \in \K[X]$ :
    \begin{itemize}
        \item $\Phi(\alpha P) = \alpha \Phi(P)$ 
        \item $\Phi(P+Q) = \Phi(P) + \Phi(Q)$ 
        \item $\Phi(P \times Q) = \Phi(P) \circ \Phi(Q)$ 
    \end{itemize}
    Ces propriétés nous serviront tout au long du chapitre et durant les suivant et s'avèrent très utiles lors de calculs. 
\end{proposition}

\begin{remark}[Extension aux matrices]
    De même, on peut définir un morphisme d'algèbre pour évaluer des polynômes par des matrices :
        \[ \forall A \in \mathcal{M}_n(\K), \quad \Phi_A : 
            \begin{cases}
                \K[X] \longrightarrow \mathcal{L}(E) \\ 
                P \longmapsto P(A)
            \end{cases}
        \]
    Cette application possède les mêmes propriétés que la précédente. 
\end{remark}


\subsection{Propriétés Calculatoires}

Rapidement, on s'apperçoit que les polynômes d'endomorphismes possèdent beaucoup de bonnes propriétés calculatoires, 
notamment pour les relations de similitudes entre matrices, autrement appelées "changement de base". 

\begin{prop}[Matrices semblables et polynômes]
    Soient $A,B\in \mathcal{M}_{\K}$ deux matrices semblables liés par la matrice $M$. Soit $P \in \K[X]$, on a :
        $$ \forall k \in \N, A^k = M B^k M ^{-1} \quad \text{et} \quad  P(A) = M P(B) M^{-1} $$
\end{prop}

\begin{prop}[Commutation]
    Soit $f \in \mathcal{L}(E)$, deux polynômes en $f$ commutent. 
\end{prop}


% ==================================================================================================================================
% Annulateurs

\section{Polynômes Annulateurs}

Nous connaissons déjà bien le concept de racine d'un polynôme réel ou complexe. 
Qu'en est-il des matrices/endomorphismes annulant un polynôme ?

\subsection{Définition et remarques}

\begin{definition}[Polynôme Annulateurs]
    Soit $ f \in \mathcal{L}(E)$ un polynôme $ P \in \K[X]$ est dit annulateur de $f$ si $P(f) = 0_{\mathcal{L}(E)}$. 
    De même pour les matrices, on dit que $P$ annule $A \in \mathcal{M}_n(\K)$ si $P(A) = 0_{\mathcal{M}_n(\K)}$. 
\end{definition}

On remarquera facilement que le polynôme nul annule tout le monde...pas très intéressant. 

\begin{remark}[Pas le même 0 ?]
    Vous remarquerez qu'en fonction de l'espace sur lequel est défini un polynôme (endomorphisme ou matrice), 
    si le polynôme est un annulateur d'un élément de cet espace, le $0$ obtenu n'est pas le même. 
    En effet, pour un polynôme défini sur les endomorphismes, le 0 obtenu sera \textbf{l'application nulle}. 
    Alors que pour les matrices, on obtient la \textbf{matrice nulle}. 
    Il est important de bien différencier les ensembles de définition des applications que l'on utilise. 
\end{remark}

\subsection{Annulateurs et valeurs propres}

\begin{proposition}
    Tout endomorphisme de E admet un polynôme annulateur non nul. 
    Idem pour les matrices. 
\end{proposition}

\begin{prop}[Valeur propre et racine]
    Soit $P$ un polynôme annulateur de $f \in \mathcal{L}(E)$, alors toute valeur propre de $f$ est racine de $P$
    (la réciproque est généralement fausse). 
\end{prop}

\begin{remark}
    Dans le cas ou $f$ est diagonalisable, on peut facilement construire un annulateur. 
    
    Soit $P \in \K[X]$ et $f \in \mathcal{L}(E)$. Soit $\text{Sp}(f) = \{\alpha_1, \alpha_2, \dots, \alpha_p\}, p \in N$ 
    le spectre de $f$. D'après la propriété précédente, tous les $\alpha_i$ sont racine de $P$. 
    Autrement dit : 
        \[ \forall i \in \{1,\dots,p\}, \quad (X - \alpha_i) \; | \; P \]
    Posons :
        \[ M_f := (X - \alpha_1)(X - \alpha_2) \dots (X - \alpha_p)\]
    On a donc : 
        \[ M_f \; | \; P\]
    Autrement dit, \textbf{$P$ est un multiple de $M_f = (X - \alpha_1)(X - \alpha_2) \dots (X - \alpha_p)$}.
\end{remark}


% ==================================================================================================================================
% Théorèmes Fondamentaux 

\section{Théorèmes Fondamentaux}

Ci-dessous, un des théorèmes les plus forts d'algèbre linéaire cette année. 
La démonstration n'est absolument pas triviale...

\begin{theorem}[Arthur Cayley et William Hamilton]
    Pour toute matrice et endomorphisme, \textbf{le polynôme caractéristique est annulateur}. 
\end{theorem}

\subsection{Divisibilité}

Maintenant que nous avons vu pas mal de propriétés sur les polynômes annulateurs, le polynôme caractéristique et quelques
relations de divisibilité en fonction des valeurs propres d'un endomorphisme/matrice, intéressons nous plus profondément 
aux ordres de divibilités entre des polynômes. 

\vspace{0.3cm}

Saut mention contraire, on notera $M_f$ le polynôme minimal d'un endomorphisme $f$ et $P_f$ son polynôme caractéristique 
dans une certaine base. 

\begin{definition}[Polynôme Minimal]
    Soit $f \in \mathcal{L}(E)$. Il existe un unique polynôme, noté $M_f$, tel que l'ensemble des polynômes annulateurs 
    de $f$ soient des multiples de $M_u$. 

    \vspace{0.3cm}

    Si on interprète cette définition de façon ensembliste, le polynôme minimal d'un endomorphisme peut être vu 
    comme le minimum (au sens de la divisibilité/degré) de l'ensemble des annulateurs. 
\end{definition}

\begin{proposition}
    Soit $f \in \mathcal{L}(E)$, on a :
    \begin{itemize}
        \item Si $\text{Sp}(f) = \{\alpha_1, \alpha_2, \dots, \alpha_p\}, p \in N$, alors $(X - \alpha_1)(X - \alpha_2) \dots (X - \alpha_p)$ divise $M_f$.
        \item $M_f$ divise $P_f$
    \end{itemize}
\end{proposition}

\newpage 

\subsection{Lemme des noyaux}

\begin{prop}[Noyau et sev stable]
    Pour tout $P \in \K[X]$, et pour tout $f \in \mathcal{L}(E)$, $\ker P(u)$ est un sev de $E$ stable par $f$. 

    \vspace{0.3cm}

    Autrement dit, le noyau d'un polynôme évalué en un endomorphisme est un sev de l'espace de départ et, est de plus stable par cet endomorphisme. 
\end{prop}

\begin{lemma}[Noyaux]
    Soient $P$ et $Q$ deux polynômes premiers entre eux et $f \in \mathcal{L}(E)$, alors :
        \[ \boxed{\ker (PQ(f)) = \ker(P(f)) \oplus \ker(Q(f))} \] 
\end{lemma}

On peut donc en déduire un cas plus général... 

\begin{theorem}[Généralisation du Lemme des noyaux]
    Soit $P$ un polynôme annulateur de $f \in \mathcal{L}(E)$, et $P = P_1P_2 \dots P_k$ une décomposition de P en produit de polynômes 
    deux à deux premiers entre eux, on a donc :
        \[E = \bigoplus_{i \in \{1,\dots, k\}} \ker (P_i(f))\]
    Autrement dit, une factorisation d'un polynôme annulateur de $f$ en produit de polynômes deux à deux premiers entre eux 
    nous donne un décomposition de E en sous-espaces, d'une part stables par $f$ mais aussi en somme directe dans E. 
\end{theorem}

On peut donc en déduire le théorème suivant, nous donnant de nouveaux critères de diagonalisation. 

\begin{theorem}[Nouveau critère de diagonalisation]
    $f \in \mathcal{L}(E)$ est diagonalisable ssi 
    \begin{itemize}
        \item elle admet un \textbf{polynôme annulateur} scindé à racine simples. 
        \item ssi son \textbf{polynôme minimal} est scindé à racines simples. 
    \end{itemize}
\end{theorem}

\begin{criteria}[Ultime Critère]
    $f \in \mathcal{L}(E)$ est diagonalisable ssi son polynôme minimal est scindé à racines simples. 
\end{criteria}


% ==================================================================================================================================
% Endomorphismes Nilpotents

\section{Endomorphismes Nilpotents}

\begin{definition}[Endomorphisme/Matrice Nilpotent]
    Soient $f \in \mathcal{L}(E)$ et $A \in \mathcal{M}_n(\K)$, on dit que $f$ ou $A$ est nilpotent ssi 
        \[ \exists k \in \N^* \text{ tel que : }
            \begin{cases}
                f^k = 0_{\mathcal{L}(E)} \text{ et } L^{k-1} \not = 0_{\mathcal{L}(E)} \\ 
                A^k = 0_{\mathcal{M}_n(\K)} \text{ et } A^{k-1} \not = 0_{\mathcal{M}_n(\K)}
            \end{cases}
        \]
    On nomme alors l'entier $k$ \textbf{indice} de nilpotence. 
\end{definition}

\begin{prop}[Matrice strictement triangulaire et nilpotence]
    Tout matrice strictement triangulaire supérieure de $\mathcal{M}_n(\K)$ est nilpotente d'indice inférieur à $n$. 
\end{prop}

\begin{prop}[Nilpotence et polynôme minimal]
    Soient $f \in \mathcal{L}(E)$ $f$ est nilpotente d'indice $k$ ssi $M_f = X^k$. 

    \vspace{0.3cm}

    La même propriété est vraie pour n'importe quelle matrice $A \in \mathcal{M}_n(\K)$. 
\end{prop}




